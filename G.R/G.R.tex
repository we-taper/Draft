% The entire content of this work (including the source code
% for TeX files and the generated PDF documents) by 
% Hongxiang Chen (nicknamed we.taper, or just Taper) is
% licensed under a 
% Creative Commons Attribution-NonCommercial-ShareAlike 4.0 
% International License (Link to the complete license text:
% http://creativecommons.org/licenses/by-nc-sa/4.0/).
\documentclass{article}

\usepackage{float}  % For H in figures
\usepackage{amsmath} % For math
\usepackage{amssymb}
\usepackage{mathrsfs}
% Followings are for the special character: differential "d".
\newcommand*\diff{\mathop{}\!\mathrm{d}}
\newcommand*\Diff[1]{\mathop{}\!\mathrm{d^#1}}
\numberwithin{equation}{subsection} % have the enumeration go to the subsection level.
                                    % See:https://en.wikibooks.org/wiki/LaTeX/Advanced_Mathematics
\usepackage{graphicx}   % need for figures
\usepackage{cite} % need for bibligraphy.
\usepackage[unicode]{hyperref}  % make every cite a link
\usepackage{CJKutf8} % For Chinese characters
\usepackage{fancyref} % For easy adding figure,equation etc in reference. Use \fref or \Fref instead of \ref
\usepackage{braket} %http://tex.stackexchange.com/questions/214728/braket-notation-in-latex

% Following is for theorems etc environments
% http://tex.stackexchange.com/questions/45817/theorem-definition-lemma-problem-numbering && https://en.wikibooks.org/wiki/LaTeX/Theorems
\usepackage{amsthm}
\newtheorem{defi}{Definition}[section]
\newtheorem{thm}{Theorem}[section]
\newtheorem{lemma}{Lemma}[section]
\newtheorem{remark}{Remark}[section]
\newtheorem{prop}{Proposition}[section]
\newtheorem{coro}{Corollary}[section]
\theoremstyle{definition}
\newtheorem{ex}{Example}[section]

% A list of nomenclatures.
\usepackage{nomencl}
\makenomenclature

\usepackage{tensor}

\title{Notes for General Relativity}
\date{\today}
\author{Taper}


\begin{document}


\maketitle
\abstract{
    (None)
}
\tableofcontents
\section{The Principle of Relativity}
\label{sec:Principle_of_Relativity}
Landau's book \cite{landau} describes the speed of light, as a speed of the
maximum velocity of propagation of interaction. He perceives velocity as
describing the propagation of interaction.

In another aspect, I think that time does not exist. I perceive time as
a agent of the environmental effect, communicating information and 
coordinating movement between the system under consideration and its 
envrionment. Therefore, a single existance is eternal by nature. Perhaps 
before the universe originates, time is a meaningless concept. In my 
view, the light is the only reliable tool to serve function of
communication and coordination.

Fram either perspective, the speed of light is inherently constent.
However, from my view, it is not clear why this speed should be a maximum.

    \subsection{Invariance of Interval}
    \label{sec:Invariance of Interval}

    To do dynamics, we necessarily need a measure of distance. 
    Here introduces the distance in spacetime - interval. It is:
    \begin{align}
        \diff s^2 = \diff t^2 - \diff x^2 - \diff y^2 - \diff z^2
    \end{align}
    \textbf{Note} that from now on, I will try to use natural units
    as often as possible.

    Such a measure should be invariant in different Lorentz frames.
    Landau proves it by postulating a priori that:
    \begin{align}
        \diff s^2 = a \diff s'^2
    \end{align}
    This is suggested by $\diff s=0$ is invariant (invariance of the
    speed of light), and $\diff s$ and $\diff s'$ should be 
    infinitesimals of the same order. From this postulation, it is
    straightforward to argue that $a$ should be a constant and is
    equal to $1$.

    Using this property, one can classify interval between events
    as being \textit{timelike, spacelike,} and \textit{lightlike},
    by whether $ds>0$, $ds<0$ or $ds=0$. A mnemonic tip is that
    for timelike intervals, the "time difference" is dominant,
    and for spacelike intervals, the spatial difference is
    dominant.

    The following figure gives an indication of how the timelike,
    spacelike, lightlike classification is related to causality:
    \begin{figure}[H]
        \centering
        \includegraphics[width=0.6\linewidth]{pics/{ch1.spacetime_regions}.png}
        \caption{Spacetime regions}
    \end{figure}

    Only those events in the absolute future and in the absolute past
    regions can have a causal link with the people at $O$, due to the 
    limit of speed of propagation. And the future and past combined
    together as the timelike region.

    Another difference of timelike and spacelike regions lies in the
    prefix "absoluate". All events in timelike region cannot be
    simultaneous with $O$ in any reference frame, since the interval
    must have a nonzero time component. Similarly all events in
    spacelike region must happen in different place with the $O$.
    An additional requirement, which comes naturally from law of
    causality, is that all events in the future must remain absolutely
    in the future in any reference frame. Similarly we have an
    absolute past.

    Interestingly, the concept of being simultaneous with $O$, 
    being before or after $O$ in time, are relative for events in
    spacelike region. However, since there can be no causal link
    between $O$ and those events, this relativity does not pose
    a challenge to causality.

    Lastly, the cone formed by all events with $ds=0$ is called the
    \textit{light cone}. Events in it is very special that it deserve
    to devote a separate section to discuss it, which will be done
    later in this note.

    \subsection{Proper Time}
    \label{sec:Proper_Time}
    
    By
    \begin{align*}
        dt'^2 = dt^2-dx^2 = dt^2-(v\diff t)^2
    \end{align*}
    we have 
    \begin{align}
        \label{eq:}
        \Delta t' = \int_{t_1}^{t_2} \diff t \sqrt{1-v^2}
    \end{align}

    This shows the time dilation effect of a moving clock.
    Also, by this we can always calculate the time experienced by a clock
    by $t=\frac{1}{c}\int\diff s$ (SI unit).

    The Landau's book \cite{landau} explains a classical paradox about
    time dilation, which is omitted here.

    Interestingly, we have the property that, for all lines between two
    events, the longest one (i.e. the path that has the longest interval),
    is the straight line, contrary to the classical case. To see this,
    we note that any two events (assumed to be causally linked) can be
    connected by a flying clock with uniform speed.

    \subsection{Lorentz Transformation}
    \label{sec:Lorentz_Transformation}
    
    Next we have to consider the coordinate transformations in
    four-dimensional spacetime. The formula can be obtained solely by
    the restricting transformation to be an isometry.

    These transformations are categorize into two groups: parallel
    displacement and rotation. Only formula for rotation is worth
    consideration. Rotation in spacetime, simiilar to their
    counterparts in $\mathbb{R}^3$, can be decomposed according to
    six axes of rotation, each representing rotation in plane
    $xy,xz,yz,xt,yt,zt$. The first of this is only ordinary spatial
    rotation. The rest are called \textit{Lorentz Transformations}.
    And there is really only one important formula, one for 
    rotation around $xt$ plane:
    \begin{align}
        x = x'\cosh\psi + t \sinh\psi\text{, }
        t = x'\sinh\psi + t \cosh\psi
    \end{align}
    where $\psi$ is the "angle of rotation". However, this "angle"
    do not have much physical interpretation. Instead, the following
    formula is more useful, which relates the coordinate in frame
    $K$, to anther fram $K'$ moving in uniform velocity $v$ w.r.t
    $K$:
    \begin{align}
        x = \frac{x'+vt'}{\sqrt{1-v^2}}\nonumber\\
        y = y'\text{, }z = z' \nonumber\\
        t = \frac{t'+v x'}{\sqrt{1-v^2}}
    \end{align}

    Aided with this formula, Landau discusses the \textit{Lorentz
    contraction} phenomenon in special relativity and defines
    the proper length\nomenclature{proper length}{\nomrefpage}
    $l_0$, defined as the length in its rest frame. It is related
    to its length measured in another frame $K'$ by:
    \begin{align}
        l' = l_0 \sqrt{1-v^2}
    \end{align}

    Finally he mentions that, since Lorentz transformations are
    in fact rotations, they naturally do not commute, unless the
    rotation is fixed in only one plane.

    \subsection{Transformation of velocities}
    \label{sec:Transformation_of_velocities}
    
    Note that for convenience, I will adopt the popular Lorentz factor
    $\gamma$, defined as:
    \begin{align}
        \gamma \equiv \frac{1}{\sqrt{1-V^2}}
    \end{align}

    To get the similar transofmrations for velocities is quite easy,
    and the result is:
    \begin{align}
        v_x &= \frac{v'_x + V}{1+ v'_x V}\\
        v_y &= \frac{v'_y}{\gamma \left(1+v'x V\right)}\\
        v_z &= \frac{v'_z}{\gamma \left(1+v'x V\right)}\\
    \end{align}

    In the special case when the motion is parallel to the $x$-axis,
    we have (let $v_x$ just be $v$):
    \begin{align}
        v = \frac{v'+V}{1+v' V}
    \end{align}
    
    Landau mentions that when $V$ is significantly smaller than the
    velocity of light, we have the following approximate formula:
    \begin{align}
        \vec{v} = \vec{v'} + \vec{V} - (\vec{V}\cdot\vec{v'}) \vec{v'}
    \end{align}


    \subsection{Four-vectors}
    \label{sec:Four-vectors}
    
    This section is basically an recapitulation of important definitions
    and results in tensor analysis in spacetime. Since I am quite
    familiar with this topic, I will only cover something I felt new.

    \paragraph{Complete antisymmetric tensor $e^{iklm}$}
    It is defined such that any interchange of indices gives a minus
    sign. We requires that:
    \begin{align}
        e^{0123}= +1
    \end{align}
    Therefore:
    \begin{align*}
        e_{0123}= -1\;, e^{iklm}e_{iklm}= -24
    \end{align*}
    Also, $e^{iklm}$ is only a pseudotensor:
    \begin{proof}
        Under transformation (let $\Lambda$ be the transformation matrix)
        we have:
        $\tilde{e}^{0123}=
          \Lambda^{0}_{i}\Lambda^{1}_{k}\Lambda^{2}_{l}\Lambda^{3}_{m}
          e^{iklm}$
        Notice that $e^{iklm}=\mathrm{sgn}(iklm)$ by our definition.
        Therefore, the right hand side above is exactly the definition
        of determinant. Hence $\tilde{e}^{0123}=\mathrm{det}(\Lambda)$.

        Also, one can easily show that $\tilde{e}^{nqpr}$ is also
        totally antisymmetric, since $e^{iklm}$ is totally antisymmetric.

        With these we have:
        $$
            \tilde{e}^{nqpr}=\mathrm{det}(\Lambda)e^{nqpr}
        $$
        But $e^{iklm}$ is defined without reference to basis, therefore
        we have actually:
        \begin{align}
            \tilde{e}^{nqpr} \equiv e^{npqr}
        \end{align}
        Note that since $\Lambda$ must be orthonormal, we have 
        $\mathrm{det}(\Lambda)=\pm 1$. Therefore when
        $\mathrm{det}(\Lambda)=1$, the two requirement coincide and
        $e^{iklm}$ behaves like a tensor, and when 
        $\mathrm{det}(\Lambda)=-1$, it behaves like a pseudotensor.
    \end{proof}
    
    By similar arguments, we see that $e^{iklm}e^{npqr}$ forms a true
    tensor, so does $e^{iklm}e_{prst}$, etc. 
    
    Also, Landau says that the only tensor that is invariant in all 
    corrdinate system is the unit tensor $\sigma^i_k$. So all the 
    above tensor should be able to be expressed in terms of this.
    The author then provides the following formulae (they are too
    complicated to type that I just paste the screenshot here):
    \begin{figure}[H]
        \centering
        \includegraphics[width=0.8\linewidth]{pics/{ch1.four_vectors_formulae_4_ee}.png}
        \caption{Formulae for ee}
        \label{fig:Formulae-for-ee}
    \end{figure}

    And:
    \begin{align}
        e^{prst}A_{ip}A_{kr}A_{ls}A_{mt} &= -A e_{iklm} \\
        e^{iklm}e^{prst}A_{ip}A_{kr}A_{ls}A_{mt} &= 24 \;\mathrm{det}(A)
    \end{align}
    Next the autho introduces the dual tensors. Since this is important
    and the case of Lorentzian spacetime is somewhat different from
    the usual case, this topic deserves a separate treatment in the
    following section.

    \subsection{Dual Tensors}
    \label{sec:Dual_Tensors}
    % TODO See this post:http://physics.stackexchange.com/questions/37577/what-does-the-dual-of-a-tensor-mean-e-g-dual-stress-tensor-in-relativistic-ed
    \subsection{Light's life}
    \label{sec:Lights_life}
    The life of a proton must be miserable.
    
    Possible sources: \href{http://physics.stackexchange.com/questions/16018/does-a-photon-in-vacuum-have-a-rest-frame}{1},
    \href{http://physics.stackexchange.com/questions/29082/would-time-freeze-if-you-could-travel-at-the-speed-of-light}{2}.
    \href{https://www.quora.com/What-does-the-frame-of-reference-of-a-photon-look-like}{3}.
\begin{thebibliography}{1}
    \bibitem{landau} The Classical Theory of Fields
\end{thebibliography}
\printnomenclature
\section{License}
The entire content of this work (including the source code
for TeX files and the generated PDF documents) by 
Hongxiang Chen (nicknamed we.taper, or just Taper) is
licensed under a 
\href{http://creativecommons.org/licenses/by-nc-sa/4.0/}{Creative 
Commons Attribution-NonCommercial-ShareAlike 4.0 International 
License}. Permissions beyond the scope of this 
license may be available at \url{mailto:we.taper[at]gmail[dot]com}.
\end{document}
