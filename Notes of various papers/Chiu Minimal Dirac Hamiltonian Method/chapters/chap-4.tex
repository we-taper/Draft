\section{Looking Further}
\label{sec:Looking Further}
There has be great deal of progress made in classification of topological
insulator and superconductors. We first comment that our discussion could be
easily generalized to some topological superconductors because they naturally
admits a matrix like Hamiltonian using Nambu spinors.

Also, our discussion falls short in two respects. First, it cannot distinguish
between $\Z$ and $2\Z$ cases. Secondly, we do not give explicit formulae for
calculating topological invariants in different cases. The former problem can be
addressed in Ludwig's \cite{Schnyder2008}, while the latter requires
considerable work, and provides different topological invariant in different
cases (see sec. III.B of \cite{Chiu2016}).

There has been progress made both in incorporating defects into classification,
as well as incorporating unitary symmetries into the classification. The
defect's classification has been incorporated beautifully into the same table of
Tenfold Way. On the other hand, introduction of unitary symmetries gives a more
fruitful result. For example, introduction of crystal symmetries gives a new
class of topological insulators, dubbed topological crystalline
insulators.\cite{Ando2015} the case of reflection symmetry, some new topological
invariants denoted as $M\Z$, called mirror topological invariants, are needed to
classify topological matters\cite{Chiu2013}.

Lastly, the discussion of topological states in interaction picture extends to
the case of symmetry-protected phases (SPTs), of which I am not familiar with.
For all these new concepts, the review \cite{Chiu2016} is an excellent source of
information.

Note that due to unforeseenable problems, this document may be outdated. The
latest version may be emailed by request.
