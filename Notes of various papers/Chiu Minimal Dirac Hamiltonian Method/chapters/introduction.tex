\section{Outline}
\label{sec:Outline}
 
The classification of topological insulators in non-interacting system is in
effect a classification of $N\times N$ Hamiltonian matrix. %sym
The simple case considered here
is the classification done by considering how the Hamiltonian matrix responses
to those anti-unitary symmetries, when we effectively ignored all unitary
symmetries. Specifically, when we module out any unitary symmetries of the
Hamiltonian, the number of meaningful anti-unitary symmetries are limited, and
we only classify topological insulators based on how our Hamiltonian matrix
responses to these anti-unitary symmetries.

There are two classical approaches towards this classification. The first
\cite{Schnyder2008} is to examine the existence of Anderson delocalization at
the boundary of the insulator. The existence of robust surface conducting state
is the signature of topological insulators. When some random impurity potentials
could be present in these surfaces, the Anderson localization is the obstruction
that surface conducting states must overcome. This approach uses the Nonlinear
Sigma Models (NL$\sigma$Ms) to describe the surface Hamiltonian and consider the
when can a localization-breaking term can be added to NL$\sigma$Ms.

The second approach\cite{Kitaev2009a} is to use see the classification as an
extension problem of Clifford Algebras. The anti-unitary symmetries already
possess some behavior similar to generators of Clifford Algebras. And sometimes
Hamiltonians could be regarded as yet another generator of Clifford
Algebra\cite{Morimoto2013}. The $K$-theory approach is to put the symmetries in
the Clifford Algebra first, and consider adding the Hamiltonian that preserve
such symmetries to the Clifford Algebra. So the problem of classification of
Hamiltonians becomes a problem of extending the Clifford Algebra.

But the key to understand topological insulators, are not necessarily the
Hamiltonian of the whole bulk, but the physics happening when the bulk gap
closes. For example, the Chern number is the integration on a compact manifold
without boundary. Therefore, it should always be zero because of the Stokes
theorem, unless inside this manifold there is somewhere where the formula for
Chern number got "blown up", and it is just this locally "blown up" point
(caused by the crossing) that is responsible for the nonzero Chern number, and
for the non-triviality of topological insulators. More generally, we believe
that there is a bulk-boundary correspondence that the properties of the bulk is
reflected by the properties in the boundary.

Therefore, we can detect the different types of topological insulators when we
are focused in only the crossing point of the band, or, when we shifted the
energy zero level to the crossing point, in only the low energy physics. We
expect a locally linear crossing energy spectrum around this point, so we expect
$E=\pm\sqrt{\vb{p}^2+m^2}$, with $m$ being the parameter controlling the closing or
opening of this spectrum. One simple model with only first-order derivatives to
describe such a low energy physics is the Dirac Hamiltonian. Also, the
traditional discussion of Klein-Gordon equation all tells us to look at Dirac
Hamiltonian.

Using the Dirac Hamiltonian, it is discovered by Dr. Chiu \cite{Chiu2013a} that
the classification of topological insulators could be brought done into very
simple applications of classification and representation theory of Clifford
Algebras, which have been well studied with excellent resources provided online.
In his classification, he established a isomorphism from $G_{\text{symmetries 
class}}$, the algebra generated by symmetry operators and Dirac Hamiltonians, to
some Clifford Algebras. Consequently, the Hamiltonian and symmetry operators,
all become some specific representations of their corresponding Clifford
Algebras. Also, it will be argued later that the minimal matrix dimension of
such representation, i.e. the minimal dimension of the Dirac Hamiltonian, can
tell the specific type of this topological insulator. Therefore, as a whole all
the classification work is reduced to the consideration of minimal matrix
dimension of the representation of Clifford Algebra, in different spatial
dimension.

This work may be considered as a note to Chiu's dissertation\cite{Chiu2013a}.
However, I do not guarantee that my interpretation is the same as his as I have
modified and amended some parts to make the work more, in my opinion, congruous
and systematic.

\paragraph{Structure of this work} This work are divided into several parts.
\begin{enumerate}
    \item \textbf{Ten-fold Way} Introduces the result of Ten-fold way: why we
        care only about anti-unitary symmetries, why there are ten different
        classes.
    \item \textbf{Classification by Dirac Mass Hamiltonians} gives comprehensive
        account of how the Minimal Dirac Hamiltonian approach works, and how the
        ten-fold way is derived in this approach.
    \item \textbf{Looking Further} outlines the possible direction in classification
        under unitary symmetries, and some other results.
\end{enumerate}
