% The entire content of this work (including the source code
% for TeX files and the generated PDF documents) by 
% Hongxiang Chen (nicknamed we.taper, or just Taper) is
% licensed under a 
% Creative Commons Attribution-NonCommercial-ShareAlike 4.0 
% International License (Link to the complete license text:
% http://creativecommons.org/licenses/by-nc-sa/4.0/).
\documentclass{article}

\usepackage{float}  % For H in figures
\usepackage{amsmath} % For math
\usepackage{amssymb}
\usepackage{mathrsfs}
% Followings are for the special character: differential "d".
\newcommand*\diff{\mathop{}\!\mathrm{d}}
\newcommand*\Diff[1]{\mathop{}\!\mathrm{d^#1}}
\numberwithin{equation}{subsection} % have the enumeration go to the subsection level.
                                    % See:https://en.wikibooks.org/wiki/LaTeX/Advanced_Mathematics
\usepackage{graphicx}   % need for figures
\usepackage{cite} % need for bibligraphy.
\usepackage[unicode]{hyperref}  % make every cite a link
\usepackage{CJKutf8} % For Chinese characters
\usepackage{fancyref} % For easy adding figure,equation etc in reference. Use \fref or \Fref instead of \ref
\usepackage{braket} %http://tex.stackexchange.com/questions/214728/braket-notation-in-latex

% Following is for theorems etc environments
% http://tex.stackexchange.com/questions/45817/theorem-definition-lemma-problem-numbering && https://en.wikibooks.org/wiki/LaTeX/Theorems
\usepackage{amsthm}
\newtheorem{defi}{Definition}[section]
\newtheorem{thm}{Theorem}[section]
\newtheorem{lemma}{Lemma}[section]
\newtheorem{remark}{Remark}[section]
\newtheorem{prop}{Proposition}[section]
\newtheorem{coro}{Corollary}[section]
\theoremstyle{definition}
\newtheorem{ex}{Example}[section]

% A list of nomenclatures.
\usepackage{nomencl}
\makenomenclature

\title{Notes of Gravity and Entanglement}
\date{\today}
\author{Taper}


\begin{document}


\maketitle
\abstract{
This is my notes to \cite{utube_video}, a video lecture given by
professor Mark Van Raamsdonk. Also, it seems that this lecture is
unrelated to another paper of the same title, written by the same
author, on arXiv \cite{arxiv_paper}.
}
\tableofcontents

\section{Lecture}
\label{sec:Lecture}

We have reviewed ads/CFT, entanglement, some interesting connections
between entanglement structure of the CFT states, and the geometry
structure of the space time due to this state. If this is true, a natural
question is whether we can understand spacetime dynamics, i.e. gravitation,
from some physics related to entanglement.

Consider some CFT theory, assumning to satisfy the xxxxx (1'50) conjectur 
entanglement entropies are computed via:
xxxxx conjecture
2'17

\begin{align}
    S(A) = \frac{\text{Area}(\title{A})}{4G_N}
\end{align}

Some propertries that entanglement entropy must satisfy:
\begin{enumerate}
    \item sub-aditivity: $S(A and B)-S(A)-S(B) \geq 0$
\end{enumerate}
If there is some geometry due to the original states, will their entropy
statisfy the above constraints. Hence: \textbf{what do those constraints imply about
the spacetime geometry?}

Start with small perturbations to (?). 
\begin{thebibliography}{1}
    \bibitem{utube_video} Availalble on Youtube: \href{https://www.youtube.com/watch?v=ABJvMg5d9OY}{Mark Van Raamsdonk -Gravity and Entanglement}.
    \bibitem{arxiv_paper} \href{http://arxiv.org/abs/1609.00026v1}{arXiv:1609.00026v1}
\end{thebibliography}
\printnomenclature
\section{License}
The entire content of this work (including the source code
for TeX files and the generated PDF documents) by 
Hongxiang Chen (nicknamed we.taper, or just Taper) is
licensed under a 
\href{http://creativecommons.org/licenses/by-nc-sa/4.0/}{Creative 
Commons Attribution-NonCommercial-ShareAlike 4.0 International 
License}. Permissions beyond the scope of this 
license may be available at \url{mailto:we.taper[at]gmail[dot]com}.
\end{document}
