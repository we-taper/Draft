
Due to GRE physics preparation. Concepts that may have been covered:

\begin{itemize}
    \item Representation of Groups.
        \begin{itemize}
            \item Character of representation.
            \item Equivalence between representation.
        \end{itemize}
    \item Transformation of Fields
    \item Regular Representation.
\end{itemize}

Now let me cover these concepts quickly.

\begin{defi}[Representation]
\nomenclature{Representation}{\nomrefpage.}
    A representation of a group $G$ is a continuous homomorphism $D$
    from $G$ to the group of automorphisms of a vector space $V$:
    \begin{equation}
        D: G\mapsto \mathrm{Aut}V
    \end{equation}
    $V$ is called the \textit{represetation space}, and the
    \textit{dimension of the representation} is the dimension of $V$.
\end{defi}

The following is copied from \cite{Ludeling}.

\begin{itemize}
    \item There is always the representation $D(g) = 1$ for all $g$.
        If $\mathrm{dim} V = 1$, this is called the \textit{trivial
        representation} \nomenclature{trivial
        representation}{\nomrefpage}.
    \item The matrix groups, i.e. $\mathrm{GL}(n, K)$ and subgroups,
        naturally have the representation "by themselves", i.e.
        by$n\times n$ matrices acting on $K_n$ and satisfying the
        defining constraints (e.g. nonzero determinant). This is
        loosely called the \textit{fundamental or defining
        representation}.
    \item Two representations $D$ and $D'$ are called
        \textit{equivalent}\nomenclature{equivalent of
        representation}{\nomrefpage} if they are related by a
        similarity transformation, i.e. if there is an operator S such
        that
        \begin{equation}
            SD(g)S^{-1} = D'(g)
        \end{equation}
        for all $g$. Note that $S$ does not depend on $g$! Two equivalent
        representation can be thought of as the same representation in
        different bases. We will normally regard equivalent
        representations as being equal.

        Note also here $S$ is the transformation of basis. If we know
        the transformation of vectors $X$, then
        \begin{equation}
            X^{-1} D(g)X = D'(g)
        \end{equation}
    \item A representation is called
        \textit{faithful}\nomenclature{faithful
        representation}{\nomrefpage} if it is injective, i.e.
        $\mathrm{ker} D = \{e\}$, or in other words, if $D(g_1) \neq
        D(g_2)$ whenever $g_1 \neq g_2$.
    \item If $V$ is equipped with a (positive definite) scalar
        product, D is unitary if it preserves that scalar product, i.e
        if
        \begin{equation}
            \braket{u,v} = \braket{D(g)u,D(g)v}
        \end{equation}
        for all $g\in G$. (Here we assume that $V$ is a complex vector
        space, as that is the most relevant case. Otherwise one could
        define orthogonal representations etc.)
\end{itemize}

