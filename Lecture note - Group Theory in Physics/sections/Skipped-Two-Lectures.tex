
Due to GRE physics preparation. Concepts that may have been covered:

\begin{itemize}
    \item Conjugacy classes
    \item Representation of Groups.
        \begin{itemize}
            \item Character of representation.
            \item Equivalence between representation.
        \end{itemize}
    \item Transformation of Fields
    \item Group Algebra and Regular Representation.
\end{itemize}

\subsection{Conjugacy classes (continued)}
\label{sec:Conjugacy classes (continued)}

One important group is the symmetric group. It is important because
that every finite group of order $n$ can be embedded inside the
symmetric group $S_n$ (Cayley's theorem) \footnote{
    However, it is hard sometimes to find the smallest possible
    symmetric group to embed into. For example, $S_3$ has 6 elements
    and can thus be embedded into $S_6$. But obviously it can be
    embedded better just into itself $S_3$.
}.

The conjugacy classes in symemtric is pretty easy to find. What we
need to know is the following observation (from \cite{Ludeling})

\begin{key}
    Consider two permutation $\pi$ and $\sigma$:
    \begin{align}
        \pi =\left( \begin{array}{ccc}
             1 & \cdots & n \\
            \pi(1) & \cdots & \pi(n) \\
            \end{array} \right) ,\quad\quad
        \sigma = \left( \begin{array}{ccc}
             1 & \cdots & n \\
            \sigma(1) & \cdots & \sigma(n) \\
            \end{array} \right)
    \end{align}
    Then we have by direct calculation:
    \begin{align}
        \sigma\pi\sigma^{-1} = \left( \begin{array}{ccc}
                \sigma(1)      & \cdots & \sigma(n) \\
                \sigma(\pi(1)) & \cdots & \sigma(\pi(n)) \\
            \end{array} \right)
    \end{align}
    Therefore, the cycle structure of $\pi$ is unchanged under any
    "conjugacy transformation".
\end{key}
What do I mean by cycle structure? Let's see an example. Suppose we
have a cycle $(123)$:
\begin{center} \begin{tikzcd}[column sep=small]
    & 1 \arrow[dr] & \\
    2\arrow[ru] & & 3\arrow{ll} 
\end{tikzcd} \end{center}
Then the conjugated map $\sigma (123) \sigma^{-1}$ will have:
\begin{center} \begin{tikzcd}[column sep=small]
    & \sigma(1) \arrow[dr] & \\
    \sigma(2)\arrow[ru] & & \sigma(3)\arrow{ll} 
\end{tikzcd} \end{center}
So it has a cycle of $(\sigma(1)\sigma(2)\sigma(3))$.
We see that in general, the cycle type is unchanged under conjugacy
transformation.
\footnote{ For a permutation, its cycle type is determined by
    decomposing it into independent cycles. For independent cycles,
    the cycle type is unique. For example, $(135)(24)$ has a cycle
    type of $(3,2)$, or $(2,3)$ since the order is irrelevant. But if
    we decompose it into non-independent cycles, we cannot determine
    its cycle type. For example, $(135)=(15)(13)$, which is ambiguous
    if we were to tell the cycle type.
}

On the other hand, all elements of the same cycle type belong to the
same conjugacy class. For example, consider two permutations:
$(i_1 i_2 i_3)(i_4 i_5)$ and $\left(\pi_{i_1}
\pi_{i_2}\pi_{i_3}\right)\left(\pi_{i_4}\pi_{i_5}\right)$, where $\pi$
permutes the five numbers $i_1 \cdots i_5$.
Consider the map:
\begin{equation}
    \sigma \equiv \left( \begin{array}{ccc}
            i_1      & \cdots & i_5 \\
            \pi_{i_1} & \cdots & \pi_{i_5} \\
        \end{array} \right)
\end{equation}
Then \begin{align*}
    &\sigma (i_1 i_2 i_3)(i_4 i_5) \sigma^{-1} =  \\
    &\left( \begin{array}{ccc}
            i_1      & \cdots & i_5 \\
            \pi_{i_1} & \cdots & \pi_{i_5} \\
        \end{array} \right)
        (i_1 i_2 i_3)(i_4 i_5)
        \left( \begin{array}{ccc}
            \pi_{i_1} & \cdots & \pi_{i_5} \\
            i_1      & \cdots & i_5 \\
        \end{array} \right) \\
    & = (\pi_{i_1} \pi_{i_2}\pi_{i_3})(\pi_{i_4}\pi_{i_5})
\end{align*}
The proof for the general case is similar. So we have the theorem
below:
\begin{thm}[Cycle type determines conjugacy class]
    Two permutations are conjugate in the symmetric group if and only if they have the same cycle type.
\end{thm}
\begin{remark}
    Since a cycle type is just a set of unordered integer partition of
    number $n$. The above theorem means that the set of conjugacy
    classes in the symmetric group on a finite set is in bijection
    with the set of unordered integer partitions of the size of the
    set. 
\end{remark}

\subsection{Group Representation}
\label{sec:Group-Representation}

\begin{defi}[Representation]
\nomenclature{Representation}{\nomrefpage.}
    A representation of a group $G$ is a continuous homomorphism $D$
    from $G$ to the group of automorphisms of a vector space $V$:
    \begin{equation}
        D: G\mapsto \mathrm{Aut}V
    \end{equation}
    $V$ is called the \textit{represetation space}, and the
    \textit{dimension of the representation} is the dimension of $V$.
\end{defi}

Related concepts. (The following is copied from \cite{Ludeling}.)

\begin{itemize}
    \item There is always the representation $D(g) = 1$ for all $g$.
        If $\mathrm{dim} V = 1$, this is called the \textit{trivial
        representation} \nomenclature{trivial
        representation}{\nomrefpage}.
    \item The matrix groups, i.e. $\mathrm{GL}(n, K)$ and subgroups,
        naturally have the representation "by themselves", i.e.
        by$n\times n$ matrices acting on $K_n$ and satisfying the
        defining constraints (e.g. nonzero determinant). This is
        loosely called the \textit{fundamental or defining
        representation}.
    \item Two representations $D$ and $D'$ are called
        \textit{equivalent}\nomenclature{equivalent of
        representation}{\nomrefpage} if they are related by a
        similarity transformation, i.e. if there is an operator S such
        that
        \begin{equation}
            SD(g)S^{-1} = D'(g)
        \end{equation}
        for all $g$. Note that $S$ does not depend on $g$! Two equivalent
        representation can be thought of as the same representation in
        different bases. We will normally regard equivalent
        representations as being equal.

        Note also here $S$ represents a transformation of basis. If we
        know the transformation of vectors $X$, then
        \begin{equation}
            X^{-1} D(g)X = D'(g)
        \end{equation}
    \item A representation is called
        \textit{faithful}\nomenclature{faithful
        representation}{\nomrefpage} if it is injective, i.e.
        $\mathrm{ker} D = \{e\}$, or in other words, if $D(g_1) \neq
        D(g_2)$ whenever $g_1 \neq g_2$.
    \item If $V$ is equipped with a (positive definite) scalar
        product, D is unitary if it preserves that scalar product, i.e
        if
        \begin{equation}
            \braket{u,v} = \braket{D(g)u,D(g)v}
        \end{equation}
        for all $g\in G$. (Here we assume that $V$ is a complex vector
        space, as that is the most relevant case. Otherwise one could
        define orthogonal representations etc.)
\end{itemize}

\subsubsection{Regular Representation}
\label{sec:Regular-Representation}

Now we construce a simple representation for all groups. It utilizes a
vector space constructed from group itself called Group Algebra.

\begin{defi}[Group algebra/Monoid algebra]
\nomenclature{$A[G]$ group algebra/monoid algebra}{\nomrefpage.}

    Let $A$ be a commutative ring, $G$ be a monoid, written
    multiplicatively. Then the monoid ring $A[G]$ consists of those
    finite formal linear combinations $v$ of the form:
    \begin{equation}
        v = \sum_{g\in G} v_g g
    \end{equation}
    where $v_g \in A$. The $v_g$ are seen as coefficients and $g$ are
    seen as the basis vectors that can be multiplied. Soe the addition
    is defined as:
    \begin{equation}
        v+w = \sum_{g\in G} (v_g + w_g) g
    \end{equation}
    And the multiplication is
    \begin{equation}
        v w = \sum_{g,g'\in G} (v_g w_{g'}) gg'
    \end{equation}
    If $G$ is a group then the
    corresponding $A[G]$ is called a \textit{group algebra}.
\end{defi}
A rigorious definition would use the concept of function to define the
"formal linear combination". Please refer to page 105 (section II.3) of
\cite{lang-algebra}. 
\begin{remark}
    Note that both $A$ and $G$ can be naturally embeded into $A[G]$.
    This will helps us to define a representation of group $G$ later.
\end{remark}

\begin{ex}
    (From page 106 of \cite{lang-algebra}.)
    Polynomial rings are special cases. In $n$ variables, consider a
    multiplicative free abelian group of rank $n$. Let $X_l,\cdots
    X_n$ be generators. Let $G$ be the multiplicative subset
    consisting of elements $X_1^{v_1},\cdots,X_n^{v_n}$, where
    $v_i\leq 0$ for all $i$. Then $G$ is a monoid, and it is eas to
    verify that $A[G]$ is just $A[X_1,\cdots, X_n]$.
\end{ex}

Here we take $A$ to be $\mathbb{C}$, then we have clearly
$\mathbb{C}[G]$ a vector space equipped with a bilinear map (the
product), i.e. an algebra over a field. Its dimension is clear equal
to the order of $G$. We can also give it an inner product by:
\begin{equation}
    \braket{v,w} =\sum_{g\in G} v_g^* w_g
\end{equation}
Now we can define the regular representation:

\begin{defi}[Regular Representation]
\nomenclature{Regular Representation}{\nomrefpage.}
    A regular representation of a group $G$ is the following
    endomorphism of $\mathbb{C}[G]$:
    \begin{equation}
        D_\text{reg}: v \mapsto g\cdot v
    \end{equation}
    where $v\in \mathbb{C}[G]$, $g\in G \hookrightarrow \mathbb{C}[G]$.
\end{defi}

\begin{remark}
    This representation can be just seen as a permutation of basis
    vectors, because:
    \begin{equation}
        g\cdot v = \sum_{h\in G} v_h (gh) = \sum_{h'\in G}
        v_{g^{-1}h'}\, h'
    \end{equation}
    Therefore, it is unitary.
\end{remark}

\begin{remark}
    This representation is not 
\end{remark}
