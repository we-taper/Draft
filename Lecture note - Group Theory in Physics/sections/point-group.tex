Here I collect the result related to point group in physical space.
\begin{defi}[Point group]
\nomenclature{Point group}{\nomrefpage.}
    A group is called a point group if it contains of rotations
    that leaves a common point unchanged.
\end{defi}
\begin{defi}[Proper point group]
\nomenclature{Proper point group}{\nomrefpage.}
    A point group is called proper if its elements keep the orientation of the
    volumn element.
\end{defi}
\begin{defi}[Improper point group]
\nomenclature{Improper point group}{\nomrefpage.}
    Improper point group is a point group that contains both improper
    and proper rotations.
\end{defi}
\begin{defi}[Spatial inversion $\sigma$]
\nomenclature{Spatial inversion $\sigma$}{\nomrefpage.}
    
\end{defi}
\begin{remark}
    Let $G$ be a improper point group.
    An improper rotation in it can been seen as the direct product of a
    proper rotation and a spatial inversion $\sigma$. (Note that
    $\sigma$ commutes with any rotation, and $\sigma^2=\id$.)
    Therefore, a improper point group will always contain some proper
    rotation by $\sigma^2=\id$, and of course, the $e$ is always
    proper. Also, the subset of proper rotation will form a subgroup
    $H$. And such division will make $G/H \cong \mathbb{Z}_2$.
\end{remark}
However, a improper point group does not necessarily contains
$\sigma$. For example, assuming a proper point group has a subgroup
$H$ of index $2$, then multiply every elements of the coset by
$\sigma$ will gives us a group denoted $\sigma H$. Then $H\otimes
\sigma H$ gives us a improper point group that does not contain
$\sigma$.

\begin{defi}[P-type and I-type improper point group]
\nomenclature{P-type and I-type improper point group}{\nomrefpage.}
    A P-type improper point group is the improper point group that
    does not contain $\sigma$, spatial inversion. On the other hand,
    if it does contain $\sigma$, it is called an I-type improper point
    group.
\end{defi}

\subsection{Rotation \texorpdfstring{$C_n$}{}}
\label{sec:C_n}
$C_n$ is isomorphic to $\mathbb{Z}_n$, here the $C$ emphasize that it
represent the group generated by the rotation of $2\pi/n$. This group
is abelian.

\paragraph{Conjugacy Classes} Obvious each element form a conjugacy
class on its only. Moreover, when $n$ is even, the conjugacy class of
rotation $\pi$ is self-conjugate.

\paragraph{Invariant Subgroup} When $n$ is prime, there is no
nontrivial invariant subgroup. When $n$ can be factorized as $n_1n_2$,
both not equals $1$, then $C_n$ is the product of $C_m$ and $C_n$,
both are invariant subgroup.

\paragraph{Irreducible representation} (\textbf{pp.64-65 of \cite{book}})

Since it has $n$ elements and $n$ conjugacy classes, with
$\sum_{j=1}^n d_j^2 = n$, each irreducible representation is
$1$-dimensional. Let $R$ denotes the generator of rotation by
$2\pi/n$, then by calculating $D^j(R)^n=D^j(1)$, one can easily find
that for each irreducible representation $j$:
\begin{equation}
    D^j(R) = \exp(-i2\pi j/N)
\end{equation}
where $j=0,1,\cdots, n$. Since $R$ is its generator, the rest of the
representation is determined. Hence we have all the irreducible
representations of $C_n$. The table of characters may be found on page
65 of \cite{book}.

Note that since all representations are $1$-dimensional, their
products will keep being irreducible. So having two set of irreducible
representations of $C_n$ and $C_m$, will give us all the irreducible
representations of $C_{nm}$. For example:
% TODO pdf scan of C_6.

\subsection{Dihedral Group \texorpdfstring{$D_n$}{}}
\label{sec:Dihedral-Group}
\begin{defi}[Dihedral Group $D_n$]
\nomenclature{Dihedral Group $D_n$}{\nomrefpage.}
    Dihedral group $D_n$ is the symmetric group of regular $n$-gons
    for $n\geq 3$. For $n=1/2$, it is usually not defined. I found that
    in book \cite{book}, $D_2$ is isomorphic to Klein Group, $D_1$ is
    of course the trivial group.
\end{defi}

But we can have another more mathematical definition of Dihedral
group:
\marginpar{To be added}
\paragraph{Conjugacy Classes}
There is some important difference between the case when $n$ is odd
and when $n$ is even. When $n$ is odd, the $2$-fold axis are all
linked with the vortices of the $n$-polygon. On the contrary, when $n$
is even, there is two different $2$-fold axis. The first $n/2$ axises
are those linked with the vortices, the second $n/2$ axises are those
cut through the opposite edges. 

Note that by fact \ref{fact:20161010-srs} one can see that such
difference will note make the two classes in even $n$ the same
conjugacy class. On the contrary, by the same fact
\ref{fact:20161010-srs} one sees that those $2$-fold axises are in the
same conjugacy class when $n$ is odd.

Also denotes one $2\pi/n$ rotation around the $N$-fold axis by $T$,
then by this very same fact \ref{fact:20161010-srs}, $T$ and $T^{-1}$
lies in the same conjugacy class, related by a reflection around a
$2$-fold axis. This is obvious a self-conjugate class. (Note that,
when only $2\pi/n$ rotation is considered, they form an abelian group
and each rotation forms a conjugacy class.) Similarly for the
conjugacy class $\{T^m, T^{-m}\}$, where $m$ is an integer.

In summary, this discussion shows that the self-conjugate classes in $D_n$
are:
\begin{fact}
    When $n$ is odd (Let $n=2n'+1$): The identity class, the class of
    all $2$-fold axis. The class of $\{ T^m, T^{-m}\}$, where
    $m=1,2,\cdots, n'$. (Why $m\leq n'$? Think about the case when
    $n=5,n'=2$.). In total $n'+2=\frac{n+3}{2}$ conjugacy
    classes/irreducible representations. All conjugacy classes are
    self-conjugate.
\end{fact}
\begin{fact}
    When $n$ is even (Let $n=2n'$): The identity class, the two
    classes of all $2$-fold axis, as discussed above. The class of
    $\{ T^m, T^{-m}\}$, where $m=1,2,\cdots, n'$. In total
    $n'+3=\frac{n+6}{2}$. All conjugacy classes are self-conjugate.
\end{fact}

\paragraph{Invariant Subgroups} (\textbf{pp.30 of \cite{book}})
By the distinction between edge and vortices type of $2$-fold axises
discussed before, one can see that there is difference in invariant
subgroup when $n$ is odd or even. For example, when $n=5$ is odd,
$D_5$ has only $C_5$ as a notrivial subgroup. But when
$n=6$ is even, it has $C_6$ (then $C_2$ and $C_3$), two copies of
$D_3$ (each formed by edge and vortex type of $2$-fold axis).

More generally, there importnat subgroup of index for $D_n$. When $n$
is odd (let $n'2n'+1$), there are one subgroup of index $2$, i.e.
$C_n'$. When $n$ is even (let $n'=n/2$), there are three invariant
subgroup of index $2$: $C_n'$, $D_n$, and $D_n'$. Denote $n$ $2$-fold
axis as $S_j$, rotation around $n$-fold axis by $2\pi/n$ as $T$,
then:
\begin{align}
    D_n &= \{E, T^2,T^4,\cdots,T^{2n-2},S_0,S_2,\cdots,S_{2n-2}\} \\
    D_n' &= \{E, T^2,T^4,\cdots,T^{2n-2},S_1,S_3,\cdots,S_{2n-1}\}
\end{align}
Note that this list is not complete. For example, $D_6$ contains other
nontrivial invariant subgroups, listed in page 30 of \cite{book}.

\paragraph{Character Table} (\textbf{pp.66 of \cite{book}})

For $D_2$, it is isomorphic to the group
$\{1,\sigma,\tau,\sigma\tau\}$, where $\sigma$ is spatial inversion,
$\tau$ is time-reversal. Notice that this group is abelian, so it
obvious has $4$ irreducible representations. It is easy to get its
character table, which may be found on page 66, table 3.6 of
\cite{book}.

For $D_3$, it is non-abelian. As discussed previously, it has $3$
conjugacy classes, so it has 3 irreducible representations. Notice
that $D_3/C_3 \cong C_2$, so it inherets the two simple irreducible
representations of $C_2$. This gives two lines in the character
table. Also, $6-1^2-1^2=2^2$, so we have one $2$-dimensional
irreducible representation left.  Next, by fact
\ref{fact:character-of-identity}, we can determine one element in the
character for $2$-dimensional irreducible representation. The rest two
empty blanks can be filled using the orthogonal relations for the
character table.
%TODO Add the scanned example.

For general $D_n$, see the part for irreducible representations.

\paragraph{Irreducible representation} (\textbf{pp.71-74 of \cite{book}}) 

The general procedure is to use the invariant subgroup of index $2$
mentioned above to get the $1$-dimensional representations. And use
the relation $\sum_j m_j^2 = |D_n|$ and $\sum_j 1 = $ number of
conjugacy class to "guess" the dimensional of remaining irreducible
representations. Finally use subgroup $C_n$ to construct those
representations (induced representation) and reduce these into
irreducible ones. But how to reduce them into irreducible one is
actually the most difficult part, in my opinion.

For odd $n$, let $n=2n'+1$. We have:
\begin{align}
    \sum_{j=1}{n'+2} m_j^2 = 4n'+2
\end{align}

There is two $1$-dimensional irreducible representations, formed using
the invariant subgroup $C_{2n'+1}$. Since $4n'+2 = 2^2 n'+2$, one
guesses that the remaining ones are $n$ $2$-dimensional irreducible
representation. They are constructed (construction process could be
found in \cite{book}) to be (notice $D_{2n'+1}$ is generated by
$T_{2n'+1}$, the rotation $2\pi/(2n'+1)$, and $C'_2$, any one of the
$2$-fold reflection):
\begin{align}
    \bar{D}^{E_j}(C_{2n'+1}) &=\left( \begin{array}{cc}
    \cos(\frac{2j\pi}{2n'+1}) & -\sin(\frac{2j\pi}{2n'+1}) \\
    \sin(\frac{2j\pi}{2n'+1}) & \cos(\frac{2j\pi}{2n'+1}) \\
        \end{array} \right) \\
    \bar{D}^{E_j}(C'_{2}) &= \left( \begin{array}{cc}
                 1 & 0 \\
                 0 & -1 \\ 
                 \end{array} \right)
\end{align}
where $j=1,2,\cdots,n$.

The case for even $n$ is simialr (let $n'=n/2$). Except that here one
as three invariant subgroup of index $2$ (see dicusson about invariant
subgroup above). So it has three $1$-dimensional irreducible
representations, and $n-1$ $2$-dimensional irreducible representations.

When $n$ is odd, let $n=2n'$. (TBD)

%TODO Finish it.
