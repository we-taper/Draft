\begin{defi}[Point group]
\nomenclature{Point group}{\nomrefpage.}
    A group is called a point group if it contains of rotations
    that leaves a common point unchanged.
\end{defi}
\begin{defi}[Proper point group]
\nomenclature{Proper point group}{\nomrefpage.}
    A point group is called proper if its elements keep the orientation of the
    volumn element.
\end{defi}
\begin{defi}[Improper point group]
\nomenclature{Improper point group}{\nomrefpage.}
    Improper point group is a point group that contains both improper
    and proper rotations.
\end{defi}
\begin{defi}[Spatial inversion $\sigma$]
\nomenclature{Spatial inversion $\sigma$}{\nomrefpage.}
    
\end{defi}
\begin{remark}
    Let $G$ be a improper point group.
    An improper rotation in it can been seen as the direct product of a
    proper rotation and a spatial inversion $\sigma$. (Note that
    $\sigma$ commutes with any rotation, and $\sigma^2=\id$.)
    Therefore, a improper point group will always contain some proper
    rotation by $\sigma^2=\id$, and of course, the $e$ is always
    proper. Also, the subset of proper rotation will form a subgroup
    $H$. And such division will make $G/H \cong \mathbb{Z}_2$.
\end{remark}
However, a improper point group does not necessarily contains
$\sigma$. For example, assuming a proper point group has a subgroup
$H$ of index $2$, then multiply every elements of the coset by
$\sigma$ will gives us a group denoted $\sigma H$. Then $H\otimes
\sigma H$ gives us a improper point group that does not contain
$\sigma$.

\begin{defi}[P-type I-type improper point group]
\nomenclature{P-type I-type improper point group}{\nomrefpage.}
    A P-type improper point group is the improper point group that
    does not contain $\sigma$, spatial inversion. On the other hand,
    if it does contain $\sigma$, it is called an I-type improper point
    group.
\end{defi}

