He first introduces some important matrices:

\subsection{Some Important Matrices}
\label{sec:Some-Important-Matrices}

\paragraph{Unitary matrix} Eigenvalues of Unitary matrices has modulus $1$,
i.e. $|\lambda|=1$. This can be proved directly. Also, Unitary matrices are unitarily diagonalizable. This is a result of the following Spectral Theorem:
\begin{thm}[Spectral Theorem]
    One matrix $A$ is normal (i.e. $A^\dagger A= AA^\dagger$), if and only if it is unitarily diagonalizable.
\end{thm}
\begin{proof}
    If $A$ is normal, then by Schur decomposition, we can write
    $A=UTU^\dagger$, here $U$ is unitary and $T$ is upper-triangular. 
    Using the condition of being normal, one can show directly that $T$ 
    is in fact also normal. Now we show that any triangular matrix that 
    is normal must be diagonal. Observe that we have 
    $\braket{e_i,T^\dagger T e_i}=\braket{e_i,TT^\dagger e_i}$, 
    i.e. $\braket{T^\dagger e_i, T^\dagger e_i}=\braket{T e_i, T e_i}$. 
    This is saying that the norm of the first column of $A^\dagger$ 
    is equal to the norm of the first column of $A$. Obviously 
    $A$ has to be diagonal.

    The converse is obvious.
\end{proof}

Also, unitary matrix's eigenvector corresponding to different eigenvalues 
are orthogonal. This is a direct result of fact mentioned above.

\paragraph{Hermitian matrices} They have real eigenvalues and orthogonal
eigenvectors (proof omitted). Also, if $\text{det}(R^\dagger R)\neq 0$,
then $R^\dagger R >0$, i.e. it is positive-definite. 

\textbf{This is wong:} An example is that
the matrix $\Omega$ introduced in the previous lecture has
$\text{det}(\Omega^\dagger \Omega)=\text{det}(\Omega)$, hence
$\text{det}(\Omega)=1$ (it cannot be $0$), hence it is positive definite.

\textbf{Actually} 
$\text{det}(\Omega^\dagger \Omega)\neq\text{det}(\Omega)$, because
\begin{align}
    \sum_\rho \ket{e_\rho}\bra{e_\rho} \neq 1 \text{(unless the basis is
        orthonormal)}
    % TODO think about this in depth
\end{align}
Therefore we need anthoer argument for $\Omega$ being positive-definite.
It is provided in page 11 of \cite{book}.

\paragraph{Orthogonal matrix} For an orthogonal matrix over $\mathbb{C}$,
it is quite troublesome. For example, if $Ra=\lambda a$ and
$\lambda \neq \pm 1$, then we have $a^T a=0$, which is quite bad because
this force $a$ to have complex components.

\paragraph{Orthogonal matrix over $\mathbb{R}$} In this case, we have
similar result. But it is easy to show that for an orthogonal matrix
$R$ having only real elements, then its eigenvalues $\lambda= \pm 1$.

Then he proceeds to direct product.
\paragraph{Direct product} and also the Kronecker Product of two
matrices. Properties (let $T=R\otimes S$):
\begin{enumerate}
    \item $\mathrm{dim}T = \mathrm{dim}R \times \mathrm{dim}S$
    \item $\mathrm{tr}(T)=\mathrm{tr}(R)\mathrm{tr}(S)$
    \item $\otimes$ commutes with the operation of inverse, transpose,
        and transpose conjugation.
    \item \begin{align}
            \frac{d}{d\alpha} (R(\alpha)\otimes S(\alpha)) =
            R'(\alpha)\otimes S(\alpha) + R(\alpha)\otimes S'(\alpha)
    \end{align}
    \item when the dimentions are the same:
        \begin{enumerate}
            \item 
            $(R_1\otimes S_1)(R_2\otimes S_2) = (R_1R_2)\otimes (S_1S_2)$
        \end{enumerate}
\end{enumerate}

\subsection{Symmetry and Group}
\label{sec:Symmetry-and-Group}

Finally we arrived in the group theory.
\paragraph{Symmetry examples} Dipole transition. 
$\braket{\phi_f|\hat{P}|\phi_i}$, must happen when the parity of $\phi_i$
and $\phi_f$ is of opposite parity. (pp.18 of \cite{book})

\paragraph{Group}
\begin{defi}[Group]
\nomenclature{Group}{\nomrefpage.}
    Omitted.
\end{defi}
Some basic properties (Omitted).
\begin{defi}[Abel Group]
\nomenclature{Abel Group}{\nomrefpage.}
    Omitted.
\end{defi}
\begin{defi}[Cardinality of group $\# A$]
\nomenclature{Cardinality of group $#$}{\nomrefpage.}
    Omitted.
\end{defi}

\paragraph{Multiplication table}

Facts: group of order $1,2,$ and $3$ are unique up to an isomorphism.

\begin{defi}[Cyclic group, generators]
\nomenclature{Cyclic group, generators}{\nomrefpage.}
    Omitted.
\end{defi}

\begin{CJK}{UTF8}{gbsn}
固有转动是指的那些 $det(M)>0$ 的转动. 用$C_n$来表示他们.

Also, 周期 of $R$ is just $\braket{R}$.
\end{CJK}

Let $\sigma$\nomenclature{$\sigma$}{\nomrefpage.} for spatial reflection.
\begin{defi}[$C_N,\bar{C}_N$]
\nomenclature{$C_N,\bar{C}_N$}{\nomrefpage.}
    $\bar{C}_N=C_N*\sigma$
\end{defi}
