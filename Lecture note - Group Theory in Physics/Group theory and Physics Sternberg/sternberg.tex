% The entire content of this work (including the source code
% for TeX files and the generated PDF documents) by 
% Hongxiang Chen (nicknamed we.taper, or just Taper) is
% licensed under a 
% Creative Commons Attribution-NonCommercial-ShareAlike 4.0 
% International License (Link to the complete license text:
% http://creativecommons.org/licenses/by-nc-sa/4.0/).
\documentclass{article}

% My own physics package
% The following line load the package xparse with additional option to
% prevent the annoying warnings, which are caused by the package
% "physics" loaded in package "physicist-taper".
\usepackage[log-declarations=false]{xparse}
\usepackage{physicist-taper}
\makenomenclature % For an index of symbols.

\title{Notes of Group Theory and Physics}
\date{\today}
\author{Taper}


\begin{document}


\maketitle
\abstract{
    Notes of Group Theory and Physics written by Sternberg \cite{Sternberg1994}.
}
\tableofcontents

\section{The Classification of the finite subgroups of
\texorpdfstring{$\mathrm{SO}(3)$}{} (1.8)}
\label{sec:Classification-of-finite-subgroups-of-SO3}

Previous section established one important formula for this section. Suppose a
group $G$ acts on a set $M$. Define the \nomen{fix point set}
$\operatorname{FP}(a)$ of an element $a\in G$ as the set of all $m\in M$ such
that $am=m$, i.e. left invariant under $a$. Denote \nomen{$G_m$} for $m\in M$ as the
isotropy subgroup of $m$. Then we have
\begin{equation}
    \sum_{a\neq e} \#\operatorname{FP}(a) =
    \sum_{\text{orbits}}\frac{\#G}{\#G_m}(\#G_m-1)
    \label{eq:fp-orbit-formula}
\end{equation}
The proof is on section 1.7 of \cite{Sternberg1994}.

This section classifies all possible finite subgroups of $\mathrm{SO}(3)$. The
classification of finite subgroups of $\mathrm{O}(3)$ is on the next section.
We basically study the action of $G=\mathrm{SO}(3)$ on $M=S^2$, the unit sphere.
The first interesting discovery is
\begin{thm}[(Euler)]
    When the dimension $n$ is odd, any $a\in \mathrm{SO}(n)$ leaves at least one
    non-zero vector invariant, i.e. any $a\in \mathrm{SO}(n)$, $\Ker(a-I)\neq
    \emptyset$, or there is always a $\vb{v}\neq 0$, such that $a\vb{v}=\vb{v}$.
\end{thm}
This implies that any rotation in odd dimensional space is a rotation about some
fixed axis (since $a\vb{v}=\vb{v}$ implies
$(a\vb{p})\vdot\vb{v}=(a\vb{v})\vdot(a\vb{p})=\vb{v}\vdot\vb{p}=0$).

Then the book analyses the formula counting fix point sets and
orbits~\ref{eq:fp-orbit-formula}. Use new symbols for three numbers:
\begin{table}[H]
    \centering
    \label{tab:three-symbols-for-3-numbers}
    \begin{tabular}{c l}
        $n$ & $=$ $\# G$ \\
        $r$ & $=$ $\text{number of orbits of $G$}$ \\
        $n_i$ & $=$ $\# G_m \text{, where $m\in i$th orbit.}$
    \end{tabular}
\end{table}

Then we have
\begin{equation}
    2-\frac{2}{n} = r-\sum_{1}^{r} \frac{1}{n_i}
\end{equation}

This equation can be simplified by considering the practical numerical values of
$n$, $r$ and $n_i$, with $n_i \leq n$. By eliminating case by case, he finally arrived
at five sets of possible values:
\begin{table}[H]
\centering
\label{tab:label}
\caption{Finite rotation groups}
\begin{tabular}{l l l l l l}
    \hline
    $r$ & $(n_1,n_2,n_3)$ & $\# G$ & Schoenflies & Hermann-Mauguin & Note
    \\
    \hline
    \hline
    2 & (n,n,0) & n & $C_n$ & $n$ & Cyclic group \\
    3 & $(2,2,k), k\geq 2$ & $2k$ & $D_k$ & $222$ for $D_2$, $k2$ otherwise & Dihedral group \\
    3 & $(2,3,3)$ & 12  & $T$  & 23& of regular tetrahedron \\
    3 & $(2,3,4)$ & 24 & $O$ & 432 & of regular octahedron \\
    3 & $(2,3,5)$ & 60 & $I$ & not mentioned & of icosahedron \\
    \hline
\end{tabular}
\end{table}
For details about derivation please visit pp. 28 to 31 of \cite{Sternberg1994}.

But we need to exclude some subgroups from the list of crystallographic groups.
The result is that, on the first row, $n$ is restricted to be $1,2,3,4$ and $6$;
on the second row, $k$ is restricted to be $2,3,4$ and $6$. And $I$ is excluded
from the list of crystallographic groups. 

The reason is the common one on Solid States classes about possible space-filling
polygons (see pp.31 of \cite{Sternberg1994}). Also, a good proof about possible
rotation angles in three-dimension for a lattice is provided on pp.31 to 32 of
\cite{Sternberg1994}. The key is that the rotation matrix could be
made of integers, and hence its characteristic (trace) should be an integer.

Next the author mentioned the \nomen{atomic hypothesis}, an interesting
historical account of our view on crystals. The essence is that because only
above mentioned angles occurred in rotational symmetries of crystals, we can
presume that a crystal is not a continuum, but is "built up from discrete
subunits in a regular repetitive pattern" (pp.32).

He also mentions the \nomen{law of rational indices} which found a basis for
modern way of labeling different faces of a lattice (the $(100)$ side, etc.).
But this law is too long to be typed here.

In sum, we have only following finite subgroups of $\mathrm{SO}(3)$ that is
interested for crystals:

$$ C_1, C_2, C_3, C_4, C_6, D_2, D_3, D_4, D_6, T, O $$

\bibliography{cite}
\bibliographystyle{alpha}

\printnomenclature
\section{License}
The entire content of this work (including the source code
for TeX files and the generated PDF documents) by 
Hongxiang Chen (nicknamed we.taper, or just Taper) is
licensed under a 
\href{http://creativecommons.org/licenses/by-nc-sa/4.0/}{Creative 
Commons Attribution-NonCommercial-ShareAlike 4.0 International 
License}. Permissions beyond the scope of this 
license may be available at \url{mailto:we.taper[at]gmail[dot]com}.
\end{document}
