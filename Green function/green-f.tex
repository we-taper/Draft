% The entire content of this work (including the source code
% for TeX files and the generated PDF documents) by 
% Hongxiang Chen (nicknamed we.taper, or just Taper) is
% licensed under a 
% Creative Commons Attribution-NonCommercial-ShareAlike 4.0 
% International License (Link to the complete license text:
% http://creativecommons.org/licenses/by-nc-sa/4.0/).
\documentclass{article}

\usepackage{float}  % For H in figures
\usepackage{amsmath} % For math
\usepackage{amssymb}
\usepackage{mathrsfs}
\numberwithin{equation}{subsection} % have the enumeration go to the subsection level.
                                    % See:https://en.wikibooks.org/wiki/LaTeX/Advanced_Mathematics
\usepackage{graphicx}   % need for figures
\usepackage{cite} % need for bibligraphy.
\usepackage[unicode]{hyperref}  % make every cite a link
\usepackage{CJKutf8} % For Chinese characters
\usepackage{fancyref} % For easy adding figure,equation etc in reference. Use \fref or \Fref instead of \ref
\usepackage{braket} %http://tex.stackexchange.com/questions/214728/braket-notation-in-latex

% Following is for the special character: differential "d".
\newcommand*\diff{\mathop{}\!\mathrm{d}}
\newcommand*\Diff[1]{\mathop{}\!\mathrm{d^#1}}
% Following is for theorems etc environments
% http://tex.stackexchange.com/questions/45817/theorem-definition-lemma-problem-numbering && https://en.wikibooks.org/wiki/LaTeX/Theorems
\usepackage{amsthm}
\newtheorem{defi}{Definition}[section]
\newtheorem{thm}{Theorem}[section]
\newtheorem{lemma}{Lemma}[section]
\newtheorem{remark}{Remark}[section]
\newtheorem{prop}{Proposition}[section]
\newtheorem{coro}{Corollary}[section]
\theoremstyle{definition}
\newtheorem{ex}{Example}[section]

\usepackage{xcolor} %For colourful math:http://tex.stackexchange.com/questions/21598/how-to-color-math-symbols

% To include PDF in higher version format.
\pdfoptionpdfminorversion=6

% A list of nomenclatures.
\usepackage{nomencl}
\makenomenclature

\title{(Tentative) Calculating Green Function}
% TODO give it a better name.
\date{\today}
\author{Taper}


\begin{document}


\maketitle
\abstract{
    This is a note for reading the paper \cite{dissertation}, and for 
    understanding the code produced from that file.
}
\tableofcontents

\section{Chapter 1 - Introductions}
\label{sec:Chapter_1_-_Introductions}
This chapter is really a nice introduction to the current fields of mesoscopic physics.
The writing is clear and it traces the develpment of this field. It gives me a lucid
and holistic historical account of both the important discoveries and motives behind
them. I should find those marked regions on pdf inside this part very useful.

\begin{thebibliography}{1}
    \bibitem{dissertation} 
        \href{https://sundoc.bibliothek.uni-halle.de/diss-online/07/07H039/prom.pdf}{Electronic Transport in Mesoscopic Systems}, 
        by von Georgo Metalidis. (Link found via Google)
\end{thebibliography}
\section{Chapter 2 - 2 Landauer-Büttiker formalism}
\label{sec:Chapter_2_-_2_Landauer-Buttiker_formalism}

This chapter introduces the Landauer-Büttiker formalism for calculating the
transport properties. The typical setup is illustrated below:

\begin{figure}[H]
    \centering
    \includegraphics[width=0.8\linewidth]{pics/{ch2.setup_for_L-B_formalism}.png}
    \caption{Setup for the Landauer-Büttiker formalism}
    %\label{fig:}
\end{figure}

In that formalism, the currents following through the leads have the following
expression:
\begin{align}
    I_p = \frac{-e}{h} \sum_q \int T_{qp}(E)\left( f_p(E)-f_q(E) \right)\diff E
\end{align}
where $T_{pq}$ is the transmission coefficients for electrons to go from
lead $q$ to lead $p$. This formula can be simplified/linearized into:
\begin{align}
    I_p = \frac{e^2}{h}\sum_q T_{pq}(E_F)(V_p-V_q)
\end{align}
An obvious advantage of Landauer-Büttiker formalism is that it makes the
dependence of $I_p$ on experimental setup explicit in the formula.

This chapter continue to discuss some time reversal symmetry (TR) properties 
of this formula, centring/centering around the coefficient $T_{pq}$.
But I am perplexed by that he, while discussing TR, mentions the
magnetic field $B$ and formulae like:
\begin{align}
    T_{12}(+B)= T_{12}(-B)
\end{align}
\nomenclature{confusion: TR and $B$}{\nomrefpage}

\section{Chapter 3 Tight-binding model}
\label{sec:Chapter_3_Tight-binding_model}

\printnomenclature
\section{License}
The entire content of this work (including the source code
for TeX files and the generated PDF documents) by 
Hongxiang Chen (nicknamed we.taper, or just Taper) is
licensed under a 
\href{http://creativecommons.org/licenses/by-nc-sa/4.0/}{Creative 
Commons Attribution-NonCommercial-ShareAlike 4.0 International 
License}. Permissions beyond the scope of this 
license may be available at \url{mailto:we.taper[at]gmail[dot]com}.
\end{document}
