% The entire content of this work (including the source code
% for TeX files and the generated PDF documents) by 
% Hongxiang Chen (nicknamed we.taper, or just Taper) is
% licensed under a 
% Creative Commons Attribution-NonCommercial-ShareAlike 4.0 
% International License (Link to the complete license text:
% http://creativecommons.org/licenses/by-nc-sa/4.0/).
\documentclass{article}

% My own physics package
% The following line load the package xparse with additional option to
% prevent the annoying warnings, which are caused by the package
% "physics" loaded in package "physicist-taper".
\usepackage[log-declarations=false]{xparse}
\usepackage{physicist-taper}
\makenomenclature % For an index of symbols.

\title{Grassmann Number notes}
\date{\today}
\author{Taper}


\begin{document}


\maketitle
\abstract{
    From Supermanifolds \cite{DeWitt1992}.
}
\tableofcontents

\section{Inverse of a Supernumber}

Let $z=z_B + z_S$ be a supernumber with $z_B$ its body, $z_S$ its soul.
If $z_B\neq 0$, then it has a inverse. The inverse can be found via the
following formal series:
\begin{equation}
    \frac{1}{a+x} = \frac{1}{a}\frac{1}{1+\frac{x}{a}} =
    \frac{1}{a}\sum_{n=0}^{\infty}(-\frac{x}{a})^n
\end{equation}
Thus,
\begin{equation}
    z^{-1} = z_B^{-1}\sum_{n=0}^{\infty} (-z_B^{-1}z_S)^n
\end{equation}
\bibliography{cite}{}
\bibliographystyle{alphaurl}
\printnomenclature
\section{License}
The entire content of this work (including the source code
for TeX files and the generated PDF documents) by 
Hongxiang Chen (nicknamed we.taper, or just Taper) is
licensed under a 
\href{http://creativecommons.org/licenses/by-nc-sa/4.0/}{Creative 
Commons Attribution-NonCommercial-ShareAlike 4.0 International 
License}. Permissions beyond the scope of this 
license may be available at \url{mailto:we.taper[at]gmail[dot]com}.
\end{document}
