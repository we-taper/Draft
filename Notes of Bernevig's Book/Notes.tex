% The entire content of this work (including the source code
% for TeX files and the generated PDF documents) by 
% Hongxiang Chen (nicknamed we.taper, or just Taper) is
% licensed under a 
% Creative Commons Attribution-NonCommercial-ShareAlike 4.0 
% International License (Link to the complete license text:
% http://creativecommons.org/licenses/by-nc-sa/4.0/).
\documentclass{article}

\usepackage{float}  % For H in figures
\usepackage{amsmath} % For math
\usepackage{amssymb}
\usepackage{bbm} % for numbers within mathbb
\usepackage{mathrsfs} % For \mathscr{ABC}
\numberwithin{equation}{subsection} % have the enumeration go to the subsection level.
                                    % See:https://en.wikibooks.org/wiki/LaTeX/Advanced_Mathematics
\usepackage{graphicx}   % need for figures
\usepackage{cite} % need for bibligraphy.
\usepackage[unicode]{hyperref}  % make every cite a link
\usepackage{CJKutf8} % For Chinese characters
\usepackage{fancyref} % For easy adding figure,equation etc in reference. Use \fref or \Fref instead of \ref

% For highlighting
\usepackage{color,soul}

% Following is for theorems etc environments
% http://tex.stackexchange.com/questions/45817/theorem-definition-lemma-problem-numbering && https://en.wikibooks.org/wiki/LaTeX/Theorems
\usepackage{amsthm}
\newtheorem{defi}{Definition}[section]
\newtheorem{thm}{Theorem}[section]
\newtheorem{lemma}{Lemma}[section]
\newtheorem{remark}{Remark}[section]
\newtheorem{prop}{Proposition}[section]
\newtheorem{coro}{Corollary}[section]
\newtheorem{fact}{Fact}[section]
\theoremstyle{definition}
\newtheorem{ex}{Example}[section]
\newtheorem{argument}{Argument}[section]

% A list of nomenclatures.
\usepackage{nomencl}
\makenomenclature

% For drawing diagrams with arrows
\usepackage[all]{xy}

% -=---------------- My Own New commands ---------------
\usepackage[log-declarations=false]{xparse}
\usepackage{physics-taper}
% Referenced package: \usepackage{physics}

%  Vector notation ---------------------
% \DeclareDocumentCommand\vectorbold{ s m }{
%     \IfBooleanTF{#1}
%     {\boldsymbol{#2}}
%     {\mathbf{#2}}
% } % Vector bold [star for Greek and italic Roman]
% \DeclareDocumentCommand\vb{}{\vectorbold} % Shorthand for \vectorbold
% 
% \DeclareDocumentCommand\vectorunit{ s m }{
%     \IfBooleanTF{#1}
%     {\boldsymbol{\hat{#2}}}
%     {\mathbf{\hat{#2}}}
% } % Unit vector [star for Greek and italic Roman]
% \DeclareDocumentCommand\vu{}{\vectorunit} % Shorthand for \vectorunit
% 
% \DeclareDocumentCommand\vdot{}{\boldsymbol\cdot} % Vector dot product symbol
% \DeclareDocumentCommand\cross{}{\boldsymbol\times} % Vector cross product symbol
% \DeclareDocumentCommand\vnabla{}{\boldsymbol\nabla} % Vector bold \nabla symbol
% \DeclareDocumentCommand\grad{}{\vnabla} 
% \DeclareDocumentCommand\dive{}{\vnabla\vdot} 
% \DeclareDocumentCommand\curl{}{\vnabla\cross}
% \DeclareDocumentCommand\laplacian{}{\nabla^2} 
% 
% % Operators ----------------------------
% \let\imaginary\Im
% \RenewDocumentCommand\Im{g}{
%     \IfNoValueTF{#1}
%     {\operatorname{Im}}
%     {\operatorname{Im}\left\{ #1 \right\}}
% }
% \let\real\Re
% \DeclareDocumentCommand\Re{g}{
%     \IfNoValueTF{#1}
%     {\operatorname{Re}}
%     {\operatorname{Re} \left\{ #1 \right\} }
% }
\title{Notes of Chapter 2 of Bernevig's Book}
\date{\today}
\author{Taper}


\begin{document}


\maketitle
\abstract{
Since I have already made a written one, this note is only an outline
of the written script, in the hope of makeing it eaiser to read.
}
\tableofcontents

\section{Skipped}

\section{Berry Phase}
\input{sections/chapter2-berry-phase}

\section{Hall Conductance and Chern Numbers}

This chapter is written in an unbelievably hard to read style. I
skipped it almost completely. There are a few hand written notes,
which are not worth even re-read.

\section{Time-Reversal Symmetry}

This chapter contains material important for quantum mechanics in
general and I had written another note bearing the same name for this
chapter. Since that note is too big to be contained here, I will left
here empty and advice one search for "TR Symmetry" in this
repository for that note.

\section{Magnetic Field on the Square Lattice}

\section{Anchor}
\label{sec:Anchor}

\begin{thebibliography}{1}
    \bibitem{book} Bernevig's Topological Insulators and
    Superconductors
\end{thebibliography}
\printnomenclature
\section{License}
The entire content of this work (including the source code
for TeX files and the generated PDF documents) by 
Hongxiang Chen (nicknamed we.taper, or just Taper) is
licensed under a 
\href{http://creativecommons.org/licenses/by-nc-sa/4.0/}{Creative 
Commons Attribution-NonCommercial-ShareAlike 4.0 International 
License}. Permissions beyond the scope of this 
license may be available at \url{mailto:we.taper[at]gmail[dot]com}.
\end{document}
