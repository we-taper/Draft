% The entire content of this work (including the source code
% for TeX files and the generated PDF documents) by 
% Hongxiang Chen (nicknamed we.taper, or just Taper) is
% licensed under a 
% Creative Commons Attribution-NonCommercial-ShareAlike 4.0 
% International License (Link to the complete license text:
% http://creativecommons.org/licenses/by-nc-sa/4.0/).
\documentclass{article}

\usepackage{float}  % For H in figures
\usepackage{amsmath} % For math
\usepackage{amssymb}
\usepackage{bbm} % for numbers within mathbb
\usepackage{mathrsfs} % For \mathscr{ABC}
% Followings are for the special character: differential "d".
\newcommand*\diff{\mathop{}\!\mathrm{d}}
\newcommand*\Diff[1]{\mathop{}\!\mathrm{d^#1}}
\numberwithin{equation}{subsection} % have the enumeration go to the subsection level.
                                    % See:https://en.wikibooks.org/wiki/LaTeX/Advanced_Mathematics
\usepackage{graphicx}   % need for figures
\usepackage{cite} % need for bibligraphy.
\usepackage[unicode]{hyperref}  % make every cite a link
\usepackage{CJKutf8} % For Chinese characters
\usepackage{fancyref} % For easy adding figure,equation etc in reference. Use \fref or \Fref instead of \ref
\usepackage{braket} %http://tex.stackexchange.com/questions/214728/braket-notation-in-latex

% For highlighting
\usepackage{color,soul}

% Following is for theorems etc environments
% http://tex.stackexchange.com/questions/45817/theorem-definition-lemma-problem-numbering && https://en.wikibooks.org/wiki/LaTeX/Theorems
\usepackage{amsthm}
\newtheorem{defi}{Definition}[section]
\newtheorem{thm}{Theorem}[section]
\newtheorem{lemma}{Lemma}[section]
\newtheorem{remark}{Remark}[section]
\newtheorem{prop}{Proposition}[section]
\newtheorem{coro}{Corollary}[section]
\newtheorem{fact}{Fact}[section]
\theoremstyle{definition}
\newtheorem{ex}{Example}[section]
\newtheorem{argument}{Argument}[section]

% A list of nomenclatures.
\usepackage{nomencl}
\makenomenclature

% For drawing diagrams with arrows
\usepackage[all]{xy}

\title{Notes of Chapter 2 of Bernevig's Book}
\date{\today}
\author{Taper}


\begin{document}


\maketitle
\abstract{
Since I have already made a written one, this note is only an outline
of the written script, in the hope of makeing it eaiser to read.
}
\tableofcontents

\section{Deriving Berry Phase}
\label{sec:Deriving-Berry-Phase}
From page 1 to page 2 (equation 2.7), one derives the expression for
the phase $e^{i\gamma_m}$ using instaneous energy eigenstate for
$E_m$:
\begin{align}
    & H(\vec{R})\ket{n(\vec{R})} = E_m(\vec{R})\ket{n(\vec{R})} \\
    & \gamma_m = i\int_0^t
    \braket{
        m\left(\vec{R}(t')\right)|\frac{\partial }{\partial t}
        m\left(\vec{R}(t')\right) \diff t' }
\end{align}
Then I writes about different ways to get the $\gamma_m$:
\begin{align}
    \gamma_m = i\int_{\text{curve}}\braket{m|\nabla_{\vec{R}}\, m }
    \diff \vec{R}
\end{align}
(equation 2.8) 
Also, one defines
\begin{defi}[Berry Connection, Berry Vector Potential $\vec{A}$]
\nomenclature{Berry Connection, Berry Vector Potential
$\vec{A}$}{\nomrefpage.}
    \begin{align}
        \vec{A}_n \equiv i \braket{n|\nabla_{\vec{R}}\, n }
    \end{align}
\end{defi}
Then I proves several facts:
\begin{fact}
    \label{fact:gamma_is_real}
    $  $
    \begin{center}
        $\gamma_n$ is real
    \end{center}
\end{fact}
\begin{fact}
    $ $
    \begin{center}
        Berry connection $\vec{A}_n$ is gauge-dependent 
    \end{center}
    The dependence is: If 
    $$\ket{n}\to \ket{n'}= e^{i\xi(\vec{R})}\ket{n}$$
    then
    \begin{align}
        \vec{A}_n \to \vec{A}_n - \nabla_{\vec{R}}\,\xi(\vec{R})
    \end{align}
\end{fact}
Therefore we have
\begin{fact}
    $  $
    \begin{center}
        $\gamma_n$ is gauge-dependent, unless the path tranverses a
        closed loop.
    \end{center}
    Since 
    $$ \gamma_n \to \gamma_n - \left(\xi(\vec{R}(T))-
    \xi(\vec{R}(0)) \right) $$
    It is unchanged unless the integration curve is a closed loop. In
    which case
    $$\xi(\vec{R}(T)) - \xi(\vec{R}(0)) = 2\pi m
    \overset{\text{mod }2\pi}{=} 0 $$
\end{fact}
\begin{ex}
    There is a simple example on page 3 to show that the Berry phase
    can be actually detected.
\end{ex}
By virtue of fact \ref{fact:gamma_is_real}, we have
\section{Anchor}
\label{sec:Anchor}

\begin{thebibliography}{1}
    \bibitem{book} Bernevig's Topological Insulators and
    Superconductors
\end{thebibliography}
\printnomenclature
\section{License}
The entire content of this work (including the source code
for TeX files and the generated PDF documents) by 
Hongxiang Chen (nicknamed we.taper, or just Taper) is
licensed under a 
\href{http://creativecommons.org/licenses/by-nc-sa/4.0/}{Creative 
Commons Attribution-NonCommercial-ShareAlike 4.0 International 
License}. Permissions beyond the scope of this 
license may be available at \url{mailto:we.taper[at]gmail[dot]com}.
\end{document}
