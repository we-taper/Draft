% The entire content of this work (including the source code
% for TeX files and the generated PDF documents) by 
% Hongxiang Chen (nicknamed we.taper, or just Taper) is
% licensed under a 
% Creative Commons Attribution-NonCommercial-ShareAlike 4.0 
% International License (Link to the complete license text:
% http://creativecommons.org/licenses/by-nc-sa/4.0/).
\documentclass{article}

% My own physics package
% The following line load the package xparse with additional option to
% prevent the annoying warnings, which are caused by the package
% "physics" loaded in package "physicist-taper".
\usepackage[log-declarations=false]{xparse}
\usepackage{physicist-taper}
\makenomenclature % For an index of symbols.

\title{Condensed Matter Field Theory notes}
\date{\today}
\author{Taper}


\begin{document}


\maketitle
\abstract{
Notes of book \cite{Altland2010}, and another book \cite{Greiner1996} for
information about path integral.
}
\tableofcontents

\section{Todo}

\begin{enumerate}
    \item Understand in what case can the Gaussian integral formula can be
        applied. In another word, understand the analytical continuation of the
        Gaussian integral. See for example, buttom of pp.343 of
        \cite{Greiner1996}.
        \label{todo:analytical-c}
    \item Understand the Wick Rotation, cf. pp.356(ch11.5) of
        \cite{Greiner1996}. This is related to todo \ref{todo:analytical-c}
    \item Understand how the constant term in the path integral of a Feynman
        Kernal will(or will not) affect the physics. Understand the mathematical
        rationale to support this. (cf. buttom of pp.344 of \cite{Greiner1996}.
    \item I have doubt about the correctness of pp.110 eq 3.28 till pp.111
        (Construction recipe of the path integral), especially about his
        argument, the size of the \textit{Planck cell}.
\end{enumerate}

Bonus objectives:
\begin{enumerate}
    \item Find abou the similarity between Path Integral of a free particle and
        the solution to a classical diffusion equation (cf. pp.112, footnote 1
        of \cite{Altland2010}).
    \item Those marked \textit{todo} in \cite{Altland2010}.
\end{enumerate}
\section{pp.33 eq.1.43}
\label{sec:Table}

In page 33 of \cite{Altland2010}, the author derives a difference of action, when we
have a symmetry transformation paraterized by $\omega_a$:

\begin{align}
    x_\mu \to x'_\mu &= x_\mu + \frac{\partial x_\mu}{\omega_a}|_{\omega=0} \omega_a(x) \\
    \phi^i(x) \to \phi'(x') &= \phi^i(x) + \omega_a(x)F^i_a[\phi]
\end{align}

We have:
\begin{align}
    \mathcal{L} &= \mathcal{L}(\phi^i(x),\partial_{x_\mu}\phi^i(x)) \\
    \mathcal{L}' &= \mathcal{L'}(\phi'^i(x'),\partial_{x'_\mu}\phi'^i(x')) \\
    &= \mathcal{L}\left(\phi^i+F^i_a\omega_a, 
                    \left(\delta_{\mu\nu}-\partial_{x_\mu}(\omega_a
                    \partial_{\omega_a}x_\mu)\right)
                    \partial_{x_\nu}(\phi^i+F^i_a\omega_a)
                  \right)
\end{align}
And 

\begin{align}
    \Delta S &= \int \dd[m]{x'} \mathcal{L}' - \int \dd[m]{x} \mathcal{L} 
    \label{eq:dS-integrand}\\
    &=
    \int \dd[m]{x}
    \left(1+\partial_{x_\mu}\left(\omega_a \partial_{\omega_a}x_\mu\right) \right) 
    \nonumber\\
    &\times \mathcal{L}\left(
                    \phi^i+F^i_a\omega_a, 
                    \left(\delta_{\mu\nu}-\partial_{x_\mu}\left(\omega_a
                        \partial_{\omega_a}x_\mu\right)\right)
                        \partial_{x_\nu}(\phi^i+F^i_a\omega_a)
                    \right) \nonumber\\
    &- \int \dd[m]{x}  \mathcal{L}(\phi^i(x),\partial_{x_\mu}\phi^i(x))
\end{align}

Then he argues that, "for constant parameters $\omega_a$ the action difference
$\Delta a$ vanishes". Therefore "the leading contribution to the action
difference of a symmetry transformation must be linear in the derivative
$\partial_{x_\mu}\omega_a$".

Then he writes that "A straightforward expansion of the formula above
for $\Delta S$ shows that these terms are given by"

\begin{equation}
    \Delta S = -\int \dd[m]{x} j^a_\mu (x) \partial_{x_\mu}\omega_a
\end{equation}
where $j^a_\mu$ is:
\begin{equation}
    j^a_\mu = \left(
        \frac{\partial \mathcal{L}}{\partial(\partial_{x_\mu}\phi^i)}
            \partial_{x_\nu}\phi^i
        -\mathcal{L}\delta_{\mu\nu} \right) \frac{\partial x_\nu}{\partial\omega_a}
    - \frac{\partial\mathcal{L}}{\partial(\partial_{x_\mu}\phi^i)} F^i_a
\end{equation}

I am partically confused about how to do the "straightforward expansion". I
guess I should do $\frac{\partial}{\partial (\partial_{x_\mu}\omega_a)}$ to the
integrand inside expression for $\Delta S$, though I don't really understand the
reason. Even so, the integrand contains terms like
$\partial_{x_\mu}\partial_{\omega_a}x_\mu$, which I don't know how to deal with.

\textbf{Solution}. The reality is a bit more complicated. We first do a first
order expasion to get the infinitesimal difference:

\begin{align}
    &\mathcal{L}'-\mathcal{L} \\
    \approx &
    \frac{\partial \mathcal{L}}{\partial\phi^i} F^i_a\omega_a
    +\frac{\partial \mathcal{L}}{\partial(\partial_{x_\mu}\phi^i)}
        \left[
        \partial_\mu\left(F^i_a\omega_a\right) 
            - \partial_\mu\left(\omega_a\frac{\partial x_\nu}{\partial\omega_a}\right)
                \partial_\nu\left(\phi^i+F^i_a\omega_a\right)
        \right]
    \nonumber\\
    =&\quad
    \omega_a
    \left[
        \frac{\partial \mathcal{L}}{\partial\phi^i}F^i_a 
        +
        \frac{\partial \mathcal{L}}{\partial(\partial_{x_\mu}\phi^i)}
        \left(
            \partial_\mu F^i_a 
            - \partial_\mu(\frac{\partial x_\nu}{\partial \omega_a})
            \partial_\nu(\phi^i+F^i_a\omega_a)
        \right)
    \right]
    \label{eq:l-l-omega}
    \\
    +&
    \partial_\mu\omega_a
    \left[
        \frac{\partial \mathcal{L}}{\partial(\partial_{x_\mu}\phi^i)}
        \left(
            F^i_a-\frac{\partial x_\nu}{\partial \omega_a}
            \partial_\nu(\phi^i+F^i_a\omega_a)
        \right)
    \right]
    \label{eq:l-l-pmu-omega}
\end{align}

We also discover the integrand in Eq.\ref{eq:dS-integrand} to be
\begin{align}
    &\left(1+\partial_\mu(
        \omega_a\frac{\partial x_\mu}{\partial \omega_a})
        \right)\mathcal{L}'-\mathcal{L} 
    \\
    =& 
    \left(
    1+\partial_\mu(\omega_a\frac{\partial x_\mu}{\partial \omega_a})
    \right)
    (\mathcal{L}'-\mathcal{L})
    +
    \left(\partial_\mu(\omega_a\frac{\partial x_\mu}{\partial \omega_a})\right)
    \mathcal{L}
    \label{eq:integrand-l-density}
\end{align}
For the first term 
$\left(
    1+\partial_\mu(\omega_a\frac{\partial x_\mu}{\partial \omega_a})
\right) (\mathcal{L}'-\mathcal{L}) $, the $(\mathcal{L}'-\mathcal{L})$ already
has terms of first order of $\omega_a$ and of first order of
$\partial_\nu\omega_a$. For our purpose, the second order terms
($\partial_\nu(F^i_a\omega_a)$) from item \ref{eq:l-l-omega} and item
\ref{eq:l-l-pmu-omega} can be ignored. Also, the item
$(\partial_\mu(\omega_a\frac{\partial x_\mu}{\partial
\omega_a}))(\mathcal{L}'-\mathcal{L})$ in eq.\ref{eq:integrand-l-density} can
also be ignored.

Therefore the integrand in Eq.$\ref{eq:dS-integrand}$ becomes

\begin{align}
    & (\mathcal{L}'-\mathcal{L})
    +
    \left(\partial_\mu(\omega_a\frac{\partial x_\mu}{\partial \omega_a})\right)
    \mathcal{L} \\
    = & 
    \omega_a
    \left[
        \frac{\partial \mathcal{L}}{\partial\phi^i}F^i_a 
        +
        \frac{\partial \mathcal{L}}{\partial(\partial_{x_\mu}\phi^i)}
        \left(
            \partial_\mu F^i_a 
            - (\partial_\mu\frac{\partial x_\nu}{\partial \omega_a})
            \partial_\nu(\phi^i+F^i_a\omega_a)
        \right)
        +
        (\partial_\nu \frac{\partial x_\mu}{\partial \omega_a})\mathcal{L}
    \right]
    \\
    +&
    \partial_\mu\omega_a
    \left[
        \frac{\partial \mathcal{L}}{\partial(\partial_{x_\mu}\phi^i)}
        \left(
            F^i_a-\frac{\partial x_\nu}{\partial \omega_a}
            \partial_\nu(\phi^i+F^i_a\omega_a)
        \right)
        +
        \frac{\partial x_\mu}{\partial \omega_a}\mathcal{L}
    \right]
\end{align}

Therefore, the term we seek, i.e. the coefficient of $\partial_\mu\omega_a$ is
\begin{align}
  & \frac{\partial \mathcal{L}}{\partial(\partial_{x_\mu}\phi^i)}
    \left(
        F^i_a-\frac{\partial x_\nu}{\partial \omega_a}
        \partial_\nu(\phi^i+F^i_a\omega_a)
    \right)
    +
    \frac{\partial x_\mu}{\partial \omega_a}\mathcal{L} \\
    =&
    \left(
        \mathcal{L}\delta_{\mu\nu}
        -\frac{\partial \mathcal{L}}{\partial(\partial_{x_\mu}\phi^i)}
            \partial_\nu \phi^i
    \right) \frac{\partial x_\nu}{\partial \omega_a}
    +
    \frac{\partial \mathcal{L}}{\partial(\partial_{x_\mu}\phi^i)}F^i_a
\end{align}
which is what we expect in equation 1.43 of \cite{Altland2010}.

\textbf{Question}: as for why we should ignore the term with $\omega_a$, there
are two posts (
    \href{http://physics.stackexchange.com/questions/122965/derivation-of-the-noether-current}{[1]},
    \href{http://physics.stackexchange.com/questions/99853/on-a-trick-to-derive-the-noether-current}{[2]} )
might be useful for a thought.

\todo{confusion}

I had great doubt about this problem. Though I have posted an answer on
\href{http://physics.stackexchange.com/questions/122965/derivation-of-the-noether-current}{[1]},
I don't think that answer is satisfactory.

\section{About \texorpdfstring{$\dot{q}$}{} and imaginary time}

The $\dot{q}$ in all the path integrals, especially eq.3.6 and eq.3.8, is in
fact a shorthand for the divided difference $\frac{q_{n+1}-q_n}{\Delta t}$ as in
pp.99 (the buttom). It is not exactly the same as $\dot{q}$. pp.343 of
\cite{Greiner1996} also mentioned that in this sense, the Lagrangian in all
path integrals is not identical with the ordinary Lagrange function. Though, I
still do not know if this matters at all.

In the most common imaginary time transformation, such as those mentioned in
pp.106 of \cite{Altland2010}, and pp.358 of \cite{Greiner1996}, we have the
transformation $t\to -i\tau$. This in effect change all the $\Delta t$ in, e.g.
eq 3.5 (pp.99) of \cite{Altland2010}, to $-i\Delta\tau$. Therefore, the divided
difference
$$ \frac{q_{n+1}-q_n}{\Delta t} \to \frac{q_{n+1}-q_n}{-i \Delta\tau}$$
Therefore,
$$ \dot{q} \to i\partial_\tau q $$
$$ \dot{q}^2 \to -\partial^2_\tau q $$
\section{Eq. 3.5}

It is not so obvious to get Eq.3.5 in pp.99 of \cite{Altland2010}. Here is my
notes.

According to the book, Eq.3.3 is turned into (I set $\hbar=1$ occasionally,
though sometimes I forgot that I have set $\hbar=1$, orz):
\begin{align}
    \bra{q_f} 
    &\int \dd{q_N}\dd{p_N}\ket{q_N}\braket{q_N|p_N}\bra{p_N}
        e^{-i\hat{T}\Delta t}e^{-i\hat{V}\Delta t}\times \nonumber\\
    &   \int \dd{q_{N-1}}\dd{p_{N-1}}\ket{q_{N-1}}\braket{q_{N-1}|p_{N-1}}\bra{p_{N-1}}
        e^{-i\hat{T}\Delta t}e^{-i\hat{V}\Delta t} \times
        \dots \nonumber\\
    &   \int \dd{q_1}\dd{p_1}\ket{q_1}\braket{q_1|p_1}\bra{p_1}
        e^{-i\hat{T}\Delta t}e^{-i\hat{V}\Delta t} \ket{q_i}
\end{align}
Notice that
\todo{T has only p, V has only q}
\begin{align}
    \braket{q|p} &= \frac{\exp(iqp/\hbar)}{\sqrt{2\pi\hbar}} \\
    \bra{p_N}e^{-i\hat{T}\Delta t} &= \bra{p_N}e^{-iT(p_N)\Delta t} \\
    e^{-i\hat{V}\Delta t}\ket{q_{N-1}} &= e^{-iV(q_{N-1})\Delta t}\ket{q_{N-1}} \\
\end{align}
Also, 
\begin{align}
    &\braket{q_N|p_N}\bra{p_N}e^{-i\hat{T}\Delta t}e^{-i\hat{V}\Delta t}\ket{q_{N-1}} 
    = \frac{e^{iq_N p_N/\hbar}}{\sqrt{2\pi\hbar}}
        \bra{p_N}e^{-iT(p_N)\Delta t} e^{-iV(q_{N-1})\Delta t}\ket{q_{N-1}}
    \nonumber\\
    =& \frac{e^{iq_N p_N/\hbar}}{\sqrt{2\pi\hbar}}
        \braket{p_N | q_{N-1}}e^{-iT(p_N)\Delta t} e^{-iV(q_{N-1})\Delta t}
    = \frac{e^{ip_N (q_N-q_{N-1})/\hbar}}{2\pi\hbar}
        e^{-i[T(p_N)+V(q_{N-1})]\Delta t} 
\end{align}
etc. Now we have to pay special attentiont to the start and end. For the start,
we have a
$$ \int \dd{q_N} \braket{q_f | q_N} = \int\dd{q_N}\delta(q_N-q_f) $$
So every $q_N$ is replaced by $q_f$. For the end, we have
$$ \braket{q_1|p_1}\bra{p_1}e^{-i\hat{T}\Delta t}e^{-i\hat{V}\Delta t} \ket{q_i}
= e^{-i[T(p_1)+V(q_i)]}\frac{e^{ip_1(q_1-q_i)}}{2\pi\hbar}
$$

Together we have the whole thing into:

\begin{align}
    \int & \dd{q_1}\cdots\dd{q_{N-1}}\dd{p_1}\dd{p_N} \frac{1}{(2\pi\hbar)^N}
    \times \nonumber\\
  & e^{i\left[p_1(q_1-q_i)+\cdots p_N(q_N-q_{N-1})\right]} \times\nonumber\\
  & e^{-i\left[T(p_1)+\cdots+T(p_N)+V(q_i)+V(q_1)+\cdots+V(q_{N-1})\right]}
\end{align}
which is exactly eq.(3.5) in book.
\section{Eq 9.4}

The Hamiltonian for particle on a ring is claimed to be (Eq. 9.1 of
\cite{Altland2010}, pp. 498):
\begin{equation}
    H = \frac{1}{2}(-i\partial_\phi -A)^2 = \frac{1}{2}(p-A)^2
\end{equation}

The book \cite{Altland2010} claims that 
\begin{equation}
    L = \frac{1}{2}\dot{\phi}^2 - iA \dot{\phi}
\end{equation}
I am quite confused, especially about the appearance of $\dot{\phi}$. Can any explain
a bit?

How I tried: Since the inverse of a Legendre transformation is Legendre
transformation itself, 
\begin{align}
    \text{Denote }x &\equiv \frac{\partial H}{\partial p} = p-A,\text{ so,} \\
    p &= x + A,\quad H = \frac{1}{2}x^2 ,\text{ so,}\\
    L = x p - H &= x(x+A) - \frac{1}{2}x^2 = \frac{1}{2}x^2 + x A 
\end{align}
So my calculation found that the Lagrangian of above Hamiltonian is:
\begin{equation}
    L = \frac{1}{2}x^2 + x A
\end{equation}
where
\begin{equation}
    x = \frac{\partial H}{\partial p}
\end{equation}


\bibliography{cite}{}
\bibliographystyle{alphaurl}

\printnomenclature
\section{A brief Summary of Quantum double well}

The Quantum double well:
\begin{figure}[H]
    \centering
    \includegraphics[width=0.6\linewidth]{pics/quantum-double-well.png}
    \caption{Quantum double well}
    The solid line is the double well potential. The dotted line in
    the second quarter is to remind us of the single well potential and its sets
    of eigenvalues. The inverted potential in Euclidean time is shown below the
    $q$-axis.
\end{figure}
The Euclidean path integral is
\begin{equation}
    G_E(a,\pm a:\tau)=\int_{q(0)=\pm a,q(\tau)=a} Dq\,
    \exp(-\frac{1}{\hbar}\int_0^\tau\dd{\tau'}
        \left(\frac{m}{2}\dot{q}^2+V(q)\right)
    )
\end{equation}
with the imaginary time saddle point equation
\begin{equation}
    -m\ddot{q}+V'(q) = 0
\end{equation}

It is turned into several contributions:
\begin{equation}
    A = A_{cl}\times A_{qu}
\end{equation}
where $A$ denotes transition amplitude, "cl" for classical, "qu" for quantum.

The classical contribution includes a stationary not moving part $A_{cl,st}$, a
moving but bouncing back and forth in the inverted potential part $A_{cl,inst}$.
The moving and bouncing back and forth motion is termed \nomen{instanton}, see
p.117(\cite{Altland2010}) for details.

The $A_{cl,st}$ is $e^{0}=1$ since the stationary path $q\equiv0$. The
$A_{cl,inst}$ is calculated in the book (pp.116-117), and the result is:
\begin{equation}
    A_{cl,inst,one-trip} = \exp(-\frac{1}{\hbar}S_{inst})
\end{equation}
for one bouncing trip. Here $S_{inst}$ is given by eq.3.36 in p.117. Adding them
together gives:
\begin{equation}
    A_{cl,inst} =
    \sum_{n\text{ even/odd}}^\infty \frac{1}{n!}
    \left(\tau K e^{-S_{inst}/\hbar}\right)^n
    =\cosh(\tau Ke^{-S_{inst}/\hbar})\text{ or }\sinh(\tau Ke^{-S_{inst}/\hbar})
\end{equation}

where $K$ is just some prefactor explained in p.118 and calculated in
pp.122-123.

The quantum contribution is due to the fluctuation of path around the classical
path. This part is also divided into two categories. The $A_{qu,st}$ for
fluctuation around the stationary path is calculated in the section about single
well green function, spcifically eq.3.31 in p.114. It is approximated in p.119
to be
\begin{equation}
    A_{qu,st,n} = e^{-\omega(\tau_{i+1}-\tau_i)/2}
\end{equation}
and adding together gives
\begin{equation}
    A_{qu,st}= \prod_i e^{-\omega(\tau_{i+1}-\tau_i)/2} = e^{-\tau\omega}
\end{equation}
The rest $A_{qu,inst}$ from fluctuation around instanton is assumed to be
negligible (p.119).

In summary:
\begin{table}[H]
    \centering
    \begin{tabular}{l c c}
        Path       & Classical ($0$th order) & Fluctuation ($2$nd order) \\
        Stationary & $1$                     & $\approx e^{-\omega\tau}$  \\
        Instanton  & $e^{-\frac{1}{\hbar}S_{\text{inst}}}$, or $\cosh(\tau Ke^{-S_{\text{inst}}/\hbar}) \sinh(\tau Ke^{-S_{\text{inst}}/\hbar})$ & $\approx 1$\\
    \end{tabular}
\end{table}

\section{Tunneling of Quantum field}

This part is particular hard for me. There are majorly three problems. The first
and most obvious one, is why do we require a "thin wall" approximation? Why is
this approximation physical?

The next problem is how is expansion of first order in $f$ will make the
following equation:
\begin{equation}
    m\partial^2_r \phi = -\frac{m}{r}\partial_r \phi +
    \partial_\phi\left(V(\phi)-f\phi\right)
\end{equation}
into a simplified form:
\begin{equation}
    m\partial^2_r \phi = \partial_\phi(\phi)
\end{equation}

The last problem is that, the action in the new coordinate is:
\begin{equation}
    \int_0^{r_0}\dd{r} \int_0^{2\pi}\dd{\theta} \left[
        m(\partial_r\phi)^2+V(\phi)-f\phi\right] r
\end{equation}
where I have ignored several constants, and have used the formula:
\begin{equation}
    \dd{\tau}\dd{x} = r\dd{r}\dd{\theta}
\end{equation}
For the moment ignores $-f\phi r$, the first two terms:
\begin{equation}
    \int_0^{2\pi}\dd{\theta} \int_0^{r_0}\dd{r} \left[
        m(\partial_r\phi)^2 + V(\phi)
    \right] r
\end{equation}
is still one step away from producing a $S_{\text{inst}}$ term:
\begin{equation}
    S_{\text{inst}} := \int_0^T \dd{\tau} 
    \left(\frac{m}{2}(\partial_\tau\phi)^2 + V(\phi)\right)
\end{equation}
\section{Eq. 3.50}

This equation is actually a lemma in combinatorics, see

\href{https://en.wikipedia.org/wiki/Baker%E2%80%93Campbell%E2%80%93Hausdorff_formula#An_important_lemma}{Wikipedia
Baker–Campbell–Hausdorff formula}.

It says:
\begin{align}
    \mathrm{Ad}_{e^X}Y &\equiv e^X Y e^{-X} = e^{\mathrm{ad}_X}Y  \nonumber\\
    &\equiv Y + [X,Y] + \frac{1}{2!}[X,[X,Y]] + \frac{1}{3!}[X,[X,[X,Y]]] + \cdots
\end{align}
or just
\begin{equation}
    \mathrm{Ad}_{e^X} = e^{\mathrm{ad}_X}
\end{equation}

Here the author uses the notation $[X,\,] Y \equiv \mathrm{ad}_X Y \equiv
[X,Y]$.
\section{p.138 Exercise}
Using eq.3.50 on p.138, it is straightforward to obtain:
\begin{align*}
    n_1 &= \cos{\phi}\cos{\theta}\braket{S_1} +
    \cos{\phi}\sin{\theta}\braket{S_3} -\sin{\phi}\braket{S_2} \\
    n_2 &= \cos{\phi}\braket{S_2} + \sin{\phi}\cos{\theta}\braket{S_1} +
    \sin{\theta}\sin{\phi}\braket{S_3} \\
    n_3 &= \cos{\theta}\braket{S_3} + \sin{\theta}\braket{S_1}
\end{align*}
And we have
\begin{align*}
    \braket{S_1}&=\braket{\uparrow| S_1 |\uparrow} = 0 \\
    \braket{S_2}&=\braket{\uparrow| S_2 |\uparrow} = 0 \\
    \braket{S_3}&=\braket{\uparrow| S_3 |\uparrow} = S
\end{align*}
by the representation thoery of $\mathrm{SU}(2)$.
\section{Eq. 3.52}

\begin{align}
    \braket{\partial_\tau g|g} 
    &= \braket{\uparrow|
        \partial_\tau e^{i\phi\hat{S}_3+i\theta\hat{S}_2}
        e^{-i\phi\hat{S}_3-i\theta\hat{S}_2}
        |\uparrow}
    \nonumber\\
    &= \braket{\uparrow|
        e^{i\phi\hat{S}_3+i\theta\hat{S}_2}
        i (\hat{S}_3\partial_\tau\phi + \hat{S}_2\partial_\tau\theta)
        e^{-i\phi\hat{S}_3-i\theta\hat{S}_2}
        |\uparrow }
    \nonumber\\
    &= i\left(\partial_\tau\phi\braket{g|\hat{S}_3|g} +
        \partial_\tau\theta\braket{g|\hat{S}_2|g} \right)
    \nonumber\\
    &= i(n_3\partial_\tau\phi + n_2 \partial_\tau\theta)
    \nonumber\\
    &= i S(\cos{\theta}\partial_\tau\phi + \sin{\theta}\sin{\phi}\partial_\tau\theta)
\end{align}
Then how is $\partial_\tau\theta$ missing in (3.52).

\section{Boson coherent state (p.159)}

The aim is to show:
\begin{equation}
    a \exp(\phi a^\dagger) \ket{0} = \phi\exp(\phi a^\dagger)\ket{0}
\end{equation}

Use $aa^\dagger = 1+ a^\dagger a$. And observe:
\begin{align*}
    a(a^\dagger)^n &= (a^\dagger)^{n-1}+ a^\dagger a (a^\dagger)^{n-1} \\
    &= (a^\dagger)^{n-1}+ (a^\dagger)^{n-1}+ (a^\dagger)^2 a (a^\dagger)^{n-2}
    \\
    &= 3(a^\dagger)^{n-1} + (a^\dagger)^3 a (a^\dagger)^{n-3} \\
    &\cdots \\
    &= n (a^\dagger)^{n-1} + (a^\dagger)^{n} a
\end{align*}
Then (notice $a\ket{0}=0$):
\begin{align*}
    \sum_n a \frac{\phi^n (a^\dagger)^n}{n!}\ket{0} 
    &= \sum_n \phi\, \phi^{n-1} \frac{n(a^\dagger)^{n-1}+(a^\dagger)^n a}{n!}\ket{0}\\
    &= \sum_n \phi\, \phi^{n-1} \frac{n(a^\dagger)^{n-1}}{n!}\ket{0} \\
    &= \phi\sum_n \phi^{n} \frac{(a^\dagger)^{n}}{n!}\ket{0} \\
    &= \phi\exp{\phi a^\dagger}\ket{0}
\end{align*}
\section{Something about Grassmann Algebra, Grassmann Calculus}

I believe that there are several important points missing in the book, when he
talks about Grassmann.

\paragraph{The first} thing is the derivative rule. The differentiation with respect to
Grassmann numbers should follow some version of \nomen{graded Leibniz rule}. But
I am not sure about the details for the following reason. Consider the
traditional graded Leibniz rule:
\begin{equation}
    \dd (fg) = (\dd f) g + (-1)^{[f]} f\dd g
\end{equation}
where $[f]$ is something defined as the parity, or the degree of f.

Now comes the problem: how to define a grade for something like:
\begin{align}
    f &= \eta_1 +\eta_2 + \eta_3 + \eta_1\eta_2 +\eta_1\eta_2\eta_3 \\
    g &= \eta_1 -\eta_2 -\eta_3 + \eta_1\eta_3 - \eta_1\eta_2\eta_3
\end{align}

\todo{understand graded Leibniz rule}

By calculation, I found:
\begin{align}
    \frac{\partial }{\partial \eta_1}(fg) &= -2\eta_2-2\eta_3-2\eta_2\eta_3 \\
    (\frac{\partial }{\partial \eta_1}f)g &=
        \eta_1-\eta_2-\eta_3+\eta_1\eta_3-\eta_1\eta_2-\eta_2\eta_3-\eta_1\eta_2\eta_3
    \\
    f( \frac{\partial }{\partial \eta_1}g) &=
        \eta_1+\eta_2+\eta_3+ \eta_1\eta_3 +\eta_1\eta_2+\eta_2\eta_3
        +\eta_1\eta_2\eta_3
\end{align}
It is easy to see that:
\begin{equation}
    \pdv{\eta_1}(fg) \neq (\pdv{\eta_1}f)g \pm f(\pdv{\eta_1} g)
\end{equation}

A more general calculation gives:
\begin{align}
    \pdv{\eta}\left[ (a+\eta b)(c+\eta d) \right] &= bc + (-1)^{[a]} ad \\
    \left(\pdv{\eta}(a+\eta b)\right) (c+\eta d) &= bc + b\eta d \\
    (a+\eta b)\left(\pdv{\eta}(c+\eta d)\right) &= ad + \eta bd
\end{align}
where $(-1)^{[a]}$ is the number such that $a\eta = (-1)^{[a]}\eta a$. So, there
is no way to reconcile the above calculations into something like:
\begin{align*}
    \pdv{\eta}\left[ (a+\eta b)(c+\eta d) \right] = 
    \left(\pdv{\eta}(a+\eta b)\right) (c+\eta d)  
    + (-1)^?  (a+\eta b)\left(\pdv{\eta}(c+\eta d)\right)
\end{align*}

\paragraph{The second} point is very important. It is about the postulations of
Grassmann variables. I postulate that (in addition to the postulations in p.162
of \cite{Altland2010}):
\begin{enumerate}
    \item All $\eta_i,\bar{\eta}_i,a_i,a^\dagger_i$ anti-commute with each
        other. From this we also have $\eta_i^2=0$, etc.
    \item $\{\dd \eta_i,\dd\eta_j\}=0$ for ($i\neq j$).
    \item $\int \dd{\eta} (af(\eta) + bg(\eta) )
        = a\int\dd{\eta}f(\eta) + b\int\dd{\eta}g(\eta) $
    \item $\{\eta_i,\dd{\eta_j}\}=0$, for $i\neq j$.
    \item $\{\bar{\eta}_i,\dd{\eta_j}\}=0$, for $\forall i,j$.
    \item $\pdv{\eta} (af(\eta)) = a\pdv{\eta}f(\eta)$, for $a\in\C$.
\end{enumerate}
It is not so simple to tell whether we should have 
$$[\eta_i,\dd\eta_i]=0 \,\text{?}$$ or
$$\{\eta_i,\dd\eta_i\}=0\,\text{?}$$
But we should have 
$$\{\eta_i,\dd{\eta_j}\}=0 \quad\text{for $i\neq j$} $$
For if $[\eta_i,\dd{\eta_j}]=0$, then we have a contradiction:
\begin{align}
    \int\dd{\eta_j} \eta_i\eta_j &= \eta_i \int\dd{\eta_j}\eta_j = \eta_i
    \nonumber\\
    &= -\int\dd{\eta_j}\eta_j \eta_i = -\eta_i\quad \text{contradiction!}
\end{align}
And similar argument applies for $\{\bar{\eta}_i,\dd{\eta_j}\}=0$, $\forall
i,j$.

There is also another point missing. The relations between $\eta$ and
$\ket{0}\bra{0}$ is ambiguous. It seems that there is no harm in assuming:
\begin{equation}
    \eta \ket{0}\bra{0} = \ket{0}\bra{0}\eta
\end{equation}
Because, if so, we have:
\begin{equation}
    a\ket{\eta}\bra{\eta} = \eta\ket{\eta}\bra{\eta} = \ket{\eta}\bra{\eta}\eta
\end{equation}
without conflict with:
\begin{equation}
    \bar{\eta}\ket{\eta}\bra{\eta} = \ket{\eta}\bra{\eta}\bar{\eta} =
    \ket{\eta}\bra{\eta} a^\dagger
\end{equation}

With above assumptions, I have:
\begin{thm}
    Assuming that $a,b,c,d$ either commute or anticommute with $\eta$.
    \begin{equation}
        \int\dd{\eta} \left(\pdv{\eta}(a+\eta b) \right)(c+\eta d)
        = (-1)^s\int\dd{\eta} (a+\eta b) \left(\pdv{\eta}(c+\eta d)\right)
        \label{eq:inteByPart-grass}
    \end{equation}
    where $s$ is such that $b\eta = (-1)^s \eta b$.
\end{thm}
This can also be generalized into
\begin{equation}
    \int\dd{\eta} \left(\pdv{\eta}f(\eta) \right) g(\eta)
    = (-1)^s\int\dd{\eta} f(\eta) \left(\pdv{\eta}g(\eta)\right)
\end{equation}
where $s$ is some appropriate constant as in eq~\ref{eq:inteByPart-grass}. Here
we have assumed that any $f(\eta)$, we can decompose it into $f=a+\eta b$, where
$a$ and $b$ do not contain $\eta$. We also assume $a,b$ either commute or
anticommute with $\eta$.
\begin{proof}
    Direct calculation.
\end{proof}

\paragraph{Change of variables in integration}
First we consider a simple example.
\begin{equation}
    \int \dd{\phi} a+b\phi = b
\end{equation}
Let $v = \lambda \phi$, since
\begin{equation}
    \int\dd{v} a+ b\frac{v}{\lambda} = \frac{b}{\lambda}
\end{equation}
We naturally require
\begin{equation}
    \dd{\phi} = \lambda \dd{v}
\end{equation}
For multivariable case, it is easy to guess that:
\begin{align}
    \vb{v} &= M\vb{\phi} \\
    \dd{\phi_1}\dd{\phi_2}\cdots \dd{\phi_N} &= \det(M)\dd{v_1}\dd{v_2}\cdots\dd{v_N}
\end{align}
(Took from
\href{https://en.wikipedia.org/wiki/Berezin_integral#Change_of_Grassmann_variables}{Wikipedia
Berezin Integral}.)

\section{License}
The entire content of this work (including the source code
for TeX files and the generated PDF documents) by 
Hongxiang Chen (nicknamed we.taper, or just Taper) is
licensed under a 
\href{http://creativecommons.org/licenses/by-nc-sa/4.0/}{Creative 
Commons Attribution-NonCommercial-ShareAlike 4.0 International 
License}. Permissions beyond the scope of this 
license may be available at \url{mailto:we.taper[at]gmail[dot]com}.
\end{document}
