% The entire content of this work (including the source code
% for TeX files and the generated PDF documents) by 
% Hongxiang Chen (nicknamed we.taper, or just Taper) is
% licensed under a 
% Creative Commons Attribution-NonCommercial-ShareAlike 4.0 
% International License (Link to the complete license text:
% http://creativecommons.org/licenses/by-nc-sa/4.0/).
\documentclass{article}

% My own physics package
% The following line load the package xparse with additional option to
% prevent the annoying warnings, which are caused by the package
% "physics" loaded in package "physicist-taper".
\usepackage[log-declarations=false]{xparse}
\usepackage{physicist-taper}
\makenomenclature % For an index of symbols.

% \newcommand\keydash[]{

\title{Classification lecture by A. Ludwig}
\date{\today}
\author{Taper}


\begin{document}


\maketitle
\abstract{
This presents notes for Ludwig's lecture in BSS-2016. It took a lot of
content directly from Ludwig's handwritten notes. Those notes are
available from
\href{http://boulderschool.yale.edu/2016/boulder-school-2016-lecture-notes}{BSS's
website}.

This not serves as majorly an introduction to the classification
framework. It introduces the reason why the classification is done
with anti-unitary symmetries and explains how the three symmetries
(T,C,S) are related to the Ten-fold classes of Hamiltonians.
}

\tableofcontents

\section{Reference suggested}
\label{sec:Reference suggested}

This work is done with A. Ludwig, and his collaborators: Shinsei Ryu,
Andreas Schnyder, Akira Furusaki, Joel Moore. 

The most recent review (at that time) is: \cite{LudwigNobelSym}.
Besides, the author also suggested many useful references:
\cite{Schnyder2009},\cite{Schnyder2008},\cite{Ryu2010},
\cite{Ryu2012}, \cite{Kitaev2009a}, \cite{Chiu2013},
\cite{Zirnbauer1996}, \cite{Altland1997}, \cite{Heinzner2005},
\cite{Ryu2015}, \cite{Witten2016}.


\section{General Introduction}
\label{sec:General Introduction}

Before the classification, we should first clarify the objects that we
are going to classify. There are \textit{roughly} two types of
materials that we called topological, according to Ludwig. They are:
\begin{enumerate} % Two class of Topological M
    \item Topological phase with intrinsic topological order.
        
        For example, the Fractional Quantum Hall state. Typically they
        have:
        \begin{itemize}
            \item Ground state degeneracy on topologically non-trivial
                position space.
            \item Anyonic excitations which may have fractional
                quantum numbers and non-trivial Braiding properties.
        \end{itemize}

    \item Symmetry Protected Topological (SPT) Phases

        (example: The $(2+1)$d $(p+ip)$ superconductor has a small
        symmetry group $\Z_2$ (Fermion parity) protecting its SPT
        phase.)
        \begin{itemize}
            \item which have \textbf{none} of the above properties,
            \item but whose ground state cannot be continuously
                deformed into a direct product state (without crossing
                a quantum phase transition at which the gap closes),
                \todo{quantum phase trans?}
                as long as the symmetry protecting the SPT is
                protected.
        \end{itemize}
\end{enumerate}

The aim of this classification is to classify the $2^{\text{nd}}$
group, that of those SPT phases.

\textbf{About this classification}
\begin{itemize}
    \renewcommand{\labelitemi}{$\rightarrow$}
    \item Topological insulators and topological superconductors
        of non-interacting fermions provide the simplest and first
        examples SPT phases.

        They can be completely classified in any dimension of space.
        There are several approaches to this classification.

    \item More general, interacting fermionic SPT phases is not
        yet fully understood. However, their classification should be
        built upon these "templates" of non-interacting cases.
    \item This simplest and most general classification applies to
        the case when only anti-unitarily realized symmetries are
        required to protect the SPT phase. 
    \item (\textbf{Important Note}: all that we talked about being
        \textit{realized}, are refer to the realization in
        $1$st-quantized Hamiltonian, not in the $2$nd-quantized
        Hamiltonian. For example, as we shall see, charge conjugation
        ($C$) is unitary in the $2$nd-quantized Hamiltonian, but
        anti-unitary in the $1$st-quantized Hamiltonian.
    \item This Ten-fold way, is a framework to characterize all possible
        Hamiltonians (single-particle Hamiltonians). (Ref.
        \cite{Zirnbauer1996},\cite{Altland1997},\cite{Schnyder2009}.)
    \item The unitarily realized symmetries are not required to
        protect our SPT, in the sense which will be explained
        immediately. The reason for not considering those unitarily
        realized symmetry is that their classification will depends on
        the specific symmetry group that implements the symmetry.
        Hence its classification will not be as universal as in the
        case of anti-unitarily realized symmetries.

\end{itemize}

\textbf{Looking further}: For Bosonic SPT, the group cohomology
approach aims to address this issue. For fermionic SPT, there are
generalizations of this approach, named group super-cohomology, which
is currently under exploration.

\begin{key}[Why No Unitarily realized symmetry]
    If the Hamiltonian has a symmetry that is unitarily realized, then
    by Quantum Mechanics this Hamiltonian can be block diagonalized,
    and its blocks will not possess memory of the unitary symmetry,
    i.e. they do not commute with the unitary matrix representation of
    that symmetry. Also, those blocks are responsible for the
    topological properties of the system. \todo{really? How they are
    responsible?}

    Therefore, our classification of symmetry is in fact a
    classification of these block Hamiltonians. And it turns out that
    the classification in terms of anti-unitarily realized symmetry
    does not depend on the specific matrix that represents the
    symmetry. In contrast, the classification in terms of unitarily
    realized symmetry depends on the specific group that implements the
    symmetry.
    \todo{What contrast? Specific matrix v.s. Specific group?}

    It may happen that the Hamiltonian possesses some unitarily
    realized symmetries. But these symmetries do not affect our
    classification since when the Hamiltonian becomes a block-diagonal
    matrix, the blocks has no memory of the unitarily realized
    symmetries, i.e. \textit{unitary symmetries do not give more
    constraints to any of the blocks}.
\end{key}

\paragraph{Specification of above statements}

Suppose we have a non-interacting Hamiltonian in the $2$nd-quantized
Fock space:

\begin{equation}
    \label{eq:H-2nd}
    \hat{H} = \sum_{A,B} \hat\psi^\dagger_A H_{AB} \hat\psi_{B}
\end{equation}

Where $A,B$ label lattice sites (goes from $1$ to $N$), and possibly
spin indices (then goes from $1$ to $2N$). $H_{AB}$
is the 1st quantized Hamiltonian.\todo{Is it single-particle? Make it
precise!}.

Suppose the Hamiltonian is invariant under a symmetry transformation
$\hat{U}$:

\begin{equation}
    \label{eq:sym-in-2nd-1}
    \hat{U}\hat{H}\hat{U}^{-1} = \hat{H}
\end{equation}

And it acts on creation and annihilation operators by a matrix $u$:
\begin{align}
    \label{eq:sym-in-2nd-2}
    \hat{\psi}'_A = 
    \hat{U} \hat\psi_A \hat{U}^{-1} &=
    \sum_B (u^\dagger)_{AB} \hat\psi_B \\
    \label{eq:sym-in-2nd-3}
    \hat\psi'^\dagger_A =
    \hat{U} \hat\psi^\dagger_A \hat{U}^{-1} &= 
    \sum_B  \hat\psi^\dagger_B u_{AB}
\end{align}

(we could think about $ \hat\psi^\dagger_A$ as a row vector, and
$\hat\psi_B$ as a column vector. This helps us remember this rules.)

It is easy to find that preserving of canonical commutation relation
($\{\hat\psi'^\dagger_A,\hat\psi'_B\}=\delta_{A,B}$), implies that the
matrix $u$ is unitary.

Next, when we want to pass from the $2$nd quantized picture to the
first quantized picture, we immediately realized one thing.

\paragraph{The linear problem} Passing from $2$nd quantized to the
1st quantized (i.e. plugging Eq.\ref{eq:sym-in-2nd-1},
Eq.\ref{eq:sym-in-2nd-2}, Eq.\ref{eq:sym-in-2nd-3} into
Eq.\ref{eq:H-2nd}), one is faced with $\hat{U} H_{AB} \hat{U}^{-1}$.
If $\hat{U}$ is unitary, then $H_{AB}$ is unchanged. Otherwise,
$H_{AB}$ becomes $H^*_{AB}$.

\paragraph{The unitary (linear) case and the block form of
Hamiltonian}
Suppose we choose $\hat{U}$ to be unitary, then on the 1st quantized
Hamiltonian will satisfy the usual condition:
\begin{equation}
    u H u^{-1} = H
\end{equation}

In this situation, the Hamiltonian will have a block form. More
precisely, suppose all our symmetry transformations form a group
$G_0$. Denote $\mathcal{V}$ as the $N$-dimensional single-particle
Hilbert space spanned by the single-particle states:
\begin{equation}
    \ket{A} = \hat\psi^\dagger_A \ket{0}, \quad A = 1,\cdots,N
\end{equation}
($\ket{0}$ is the Fock vacuum.)

We have
\begin{thm}[Diagonalization of Hamiltonian in unitary representation]
This space $\mathcal{V}$ decomposes into a direct sum of vector spaces
$\mathcal{V}_\lambda$ associated with the irrep (irreducible
representations, labeled by $\lambda$) of $G_0$.
\begin{equation}
    \mathcal{V} = \oplus_\lambda m_\lambda \mathcal{V}_\lambda
\end{equation}
where $m_\lambda$ denotes the multiplicity of $\lambda$th irrep.
Denote the dimension of each irrep as $d_\lambda$.

In each vector space $\mathcal{V}_\lambda$, one can choose a
(orthogonal) basis of the form:
\begin{equation}
    \ket{v^{(\lambda)}_\alpha} \otimes \ket{w^{(\lambda)}_k}
\end{equation}
where
\begin{itemize}
    \item $G_0$ acts only only $\ket{w^{(\lambda)}}_k$,
        $k=1,\cdots,d_\lambda$,
    \item $H$ acts only on $\ket{v^{(\lambda)}}_\alpha$,
        $\alpha=1,\cdots,m_\lambda$.
\end{itemize}
\end{thm}

Thus, each irrep $\lambda$ defines a block-Hamiltonian matrix of size
$m_\lambda\times m_\lambda$:
\begin{equation}
    H^{(\lambda)}_{\alpha\beta} \equiv 
    \braket{v^{(\lambda)}_\alpha | H | v^{(\lambda)}_\beta}
\end{equation}

\paragraph{The exact aim of this classification}
This classification aims to classify sets of all block Hamiltonians
$H^{(\lambda)}$ when we fix a symmetry group $G_0$ and let the
Hamiltonian runs through all possible single-particle Hamiltonians
that commute with operations (this operation should be unitarily
realized on the single-particle Hamiltonian) of $G_0$.

The result turns out to be independent of the group $G_0$, and of
course independent of its irrep $\lambda$.

\section{The Anti-Unitary Symmetries}
\label{sec:The Anti-Unitary Symmetries}

It turns out that under reasonable assumptions, there are only a
finite number of anti-unitary symmetries. Let us revisit our
definition of symmetry transformation:

\begin{equation*}
    \tag{\ref{eq:sym-in-2nd-1}, revisit}
    \hat{U}\hat{H}\hat{U}^{-1} = \hat{H}
\end{equation*}

If it acts on creation/annihilation operators by:
\begin{align*}
    \tag{\ref{eq:sym-in-2nd-2}, revisit}
    \hat{\psi}'_A = 
    \hat{U} \hat\psi_A \hat{U}^{-1} &=
    \sum_B (u^\dagger)_{AB} \hat\psi_B \\
    \tag{\ref{eq:sym-in-2nd-3}, revisit}
    \hat\psi'^\dagger_A =
    \hat{U} \hat\psi^\dagger_A \hat{U}^{-1} &= 
    \sum_B  \hat\psi^\dagger_B u_{BA}
\end{align*}

\paragraph{Case 1} Instead of requiring $\hat{U}$ to be linear (which
would give a unitarily realized symmetry on single-particle
Hamiltonian), we require it to be anti-linear:
\begin{equation}
    \hat{U} i \hat{U}^{-1} = -i
\end{equation}
Then, calculation shows:
\begin{equation}
    u H^* u^\dagger = H
\end{equation}
where $H$ is 1st quantized single-particle Hamiltonian. This symmetry
is called time-reversal symmetry, denoted $\hat{\tau}$, its action on
1st-quantized single-particle Hamiltonian $H$ can be denoted using
\begin{equation}
    T \equiv \eval{\hat{\tau}}_{\text{1st quantized Hilbert space}}
\end{equation}
and by above formula, we have:
\begin{equation}
    T = uK
\end{equation}
where $K$ is complex conjugation. And we have the usual property
$THT^{-1}=H$, where $T^{-1}=u^t K$. ($u^t$ is the transpose of $u$).

\paragraph{Case 2} If instead, we change eq.\ref{eq:sym-in-2nd-2} and
eq.\ref{eq:sym-in-2nd-3} to such that

\begin{align}
    \label{eq:sym-cc-1}
    \hat{\psi}'_A = 
    \hat{U} \hat\psi_A \hat{U}^{-1} &=
    \sum_B (u^*)^\dagger_{AB} \hat\psi^\dagger_B \\
    \label{eq:sym-cc-2}
    \hat\psi'^\dagger_A =
    \hat{U} \hat\psi^\dagger_A \hat{U}^{-1} &= 
    \sum_B  \hat\psi_B u^*_{BA}
\end{align}

(Again, a tip is to remember that $\hat\psi^\dagger_A$ is a row
vector, and $\hat\psi_A$ is a column vector)

In words, we require creation operators transform into annihilation
operators, and annihilation operators transform into creation
operators.

Now proceed again in the same way, we have the choice of linear or
antilinear $\hat{U}$. If $\hat{U}$ is linear, then the condition
$\hat{U}\hat{H}\hat{U}^{-1} = \hat{H}$ leads to the equation:
\begin{equation}
    \label{eq:sym-C-cond}
    u (H-\frac{1}{2}\tr(H))^t u^\dagger = - (H-\frac{1}{2}\tr(H))
\end{equation}
(notice in calculation that $\sum_i\hat\psi^\dagger_i \hat\psi_i =
\id$ on 1st-quantized single-particle Hilbert space.)

Taking the trace of above equality will give $2\tr(H)=N\tr(H)$, since
in solids $N>>2$, we must have $\tr(H)=0$. Then the above equality
simplifies into (note that $H^t = H^*$ for Hermitian $H$):
\begin{equation}
    u H^* u^\dagger = -H
\end{equation}

This type of symmetry is called charge-conjugation symmetry. It is
also called particle-hole symmetry in condensed matter physics. Notice
that if we denote this symmetry in 2nd-quantized Hilbert space as
$\hat{C}$, then in 1st-quantized single-particle Hilbert space, we
have
\begin{equation}
    C \equiv \eval{\hat{C}}_{\text{1st-quantized}}, \quad
    \text{and}\, C = uK
\end{equation}
with
\begin{equation}
    C H C^{-1} = -H
\end{equation}
for 1st-quantized single-particle Hamiltonian.

\paragraph{Case 3} If we now require again $\hat{U}$ being
anti-linear, and proceed with similar arguments, we would have, very
similar to eq.\ref{eq:sym-C-cond}, the equality: 
\begin{equation}
    \label{eq:sym-S-cond}
    u (H-\frac{1}{2}\tr(H)) u^\dagger = - (H-\frac{1}{2}\tr(H))
\end{equation}
and since $N>>2$, we have $\tr(H)=0$. Then
\begin{equation}
    u H u^\dagger = -H
\end{equation}
This symmetry will be called the chiral symmetry, denoted $\hat{S}$.
And we have a simple relation for its action on 2nd-quantized
Hamiltonian and the 1st-quantized Hamiltonian:
\begin{equation}
    S \equiv \eval{\hat{S}}_{\text{1st-quantized}},\quad
    \text{and}\,\, S = u
\end{equation}
with
\begin{equation}
    S H S^{-1} = -H
\end{equation}
It is easy to see that $S$ is just a combination of $T$ and $C$
symmetry. This will be noted in the coming section.

\subsection{Summary}
\label{sec:Summary}

There are 3 different symmetries that do not lie in our usual sense of
symmetry, which is unitarily realized and commutes with $H$. They are
summarized here. Also, there is a some other property listed below.
They are proved in this lecture, but are either commonly available in
most textbooks, or easy to prove. So I did not give details of proof
below.
\paragraph{Time-reversal}:
\begin{itemize}
    \item 2nd-quantized: 
        $\hat{\tau}\hat{H}\hat{\tau}^{-1}=\hat{H}$ 
    \item Relation:
    \begin{align}
        \hat{\tau} \hat\psi_A \hat{\tau}^{-1} &=
        \sum_B (u_T^\dagger)_{AB} \hat\psi_B \\
        \hat{\tau} \hat\psi^\dagger_A \hat{\tau}^{-1} &= 
        \sum_B  \hat\psi^\dagger_B (u_T)_{BA} \\
        \hat{\tau}i\hat{\tau}^{-1} &= -i
    \end{align}
    \item 1st-quantized:
        $T=u_T K$, $THT^{-1}=H$
    \item Squared term
        
        If $T$ commutes with $H$, then $T^2$ also
        commutes with $H$. Calculation shows $T^2= u_Tu_T^* = \pm 1$.
        For fermion system, $T^2=(-1)$. Hence in 2nd-quantized Hilbert
        Space, $\hat{T}^2=(-1)^{\hat{Q}}$, where $\hat{Q}=\sum_A
        \hat\psi^\dagger_A \hat\psi_A$. This is exactly the Fermion
        number parity operator.
\end{itemize}
\paragraph{Charge-conjugation}:
\begin{itemize}
    \item 2nd-quantized: 
        $\hat{C}\hat{H}\hat{C}^{-1}=\hat{H}$ 
    \item Relation:
    \begin{align}
        \hat{C} \hat\psi_A \hat{C}^{-1} &=
        \sum_B (u_C^*)^\dagger_{AB} \hat\psi^\dagger_B \\
        \hat{C} \hat\psi^\dagger_A \hat{C}^{-1} &= 
        \sum_B  \hat\psi_B (u_C^*)_{BA} \\
        \hat{C}i\hat{C}^{-1} &= i
    \end{align}
    \item 1st-quantized:
        $C=u_C K$, $CHC^{-1}=-H$, and $\tr(H)=0$.
    \item Squared term
        
        If $C$ commutes with $H$, then $C^2$ also
        commutes with $H$. Calculation shows $C^2= u_Cu_C^* = \pm 1$.
        For fermion system, $C^2=(-1)$. Hence in 2nd-quantized Hilbert
        Space, $\hat{C}^2=(-1)^{\hat{Q}}$, where $\hat{Q}=\sum_A
        \hat\psi^\dagger_A \hat\psi_A$. This is exactly the Fermion
        number parity operator.

    \item Action on $\mathcal{F}_q$:

        Consider the $\mathcal{F}_q$, the eigenspace of
        $\hat{Q}=\sum_{A=1}^N \hat\psi^\dagger_A \hat\psi_A$, with
        eigenvalue $q$. Then, direct calculation shows:
        \begin{equation}
            \hat{C}\hat{Q}\hat{C}^{-1} = N -\hat{Q}
        \end{equation}
        Therefore, $\hat{C}$ links $\mathcal{F}_q$ and
        $\mathcal{F}_{N-q}$.
\end{itemize}
\paragraph{Chiral/sub-lattice}:
\begin{itemize}
    \item 2nd-quantized: 
        $\hat{S}\hat{H}\hat{S}^{-1}=\hat{H}$ 
    \item Relation:
    \begin{align}
        \hat{S} \hat\psi_A \hat{S}^{-1} &=
        \sum_B (u_S^*)^\dagger_{AB} \hat\psi^\dagger_B \\
        \hat{S} \hat\psi^\dagger_A \hat{S}^{-1} &= 
        \sum_B  \hat\psi_B (u_S^*)_{BA} \\
        \hat{S}i\hat{S}^{-1} &= -i
    \end{align}
    \item 1st-quantized:
        $S=u_S$, $SHS^{-1}=-H$, and $\tr(H)=0$.
    \item \textbf{Relation between Chiral, Time-reversal, and
        Charge-conjugation}

        We have $\hat{S}=\hat{\tau}\hat{C}$, $u_S = u_T u_C^*$. 

    \item Squared term
        
        If $S$ commutes with $H$, then $S^2$ also
        commutes with $H$. Calculation shows $S^2= u^2_S$. But since
        $u_S=u_Tu_C^*$, we can always make a phase choice of $u_T$ and
        $u_C$ such that $S^2=u^2_S = 1$.

    \item Let $S'\equiv CT$. If $T^2=\epsilon_T\id$,
        $C^2=\epsilon_C\id$, and we adopt the phase choice such that
        $S^2=1$, then it can be shown that $S=\epsilon_C\epsilon_T
        S'$, or
        \begin{equation}
            TC = \epsilon_C\epsilon_T CT
        \end{equation}
\end{itemize}



\subsection{How many anti-unitary symmetries are there?}
\label{sec:How many anti-unitary symmetries are there?}

As we have seen, there are three classes of anti-unitary symmetry
transformations.
Within each class, there is actually one unique anti-unitary symmetry
transformation, module those of unitary ones. To be more specific,
suppose we have two realizations of charge-conjugation symmetry:
\begin{equation}
    C_1 = u_{C,1} K,\quad C_2=u_{C,2}K
\end{equation}
They are clearly related by a unitary matrix
$u_{12}=u_{C,1}u_{C,2}^\dagger$, and ${C_1=u_{12}C_2}$.
Then, if we blindly assume the 1st-quantized Hamiltonian has symmetry
$C_1$:
\begin{equation}
    C_1 H C_1^{-1} = -H
\end{equation}
then, plugging $C_2$ inside gives
\begin{equation}
    C_2 H C_2^{-1} = - u_{12}^\dagger H u_{12}
\end{equation}
Therefore, if we further enlarge the group $G_0$ to include
$u_{12}^\dagger$ (i.e. by requiring that $H$ commutes with
$u_{12}^\dagger$), then with $G'_0=G_0\cup\{u_{12}^\dagger\}$, the
classification will be done and nothing new is actually introduced.
Therefore, there is only one $C$ symmetry to be considered, module the
unitary matrices.

Similar argument applies for $T$. But for $S$, Ludwig specifically
point out that $S$'s expression should always be kept explicit. I am
confused about the reason. \todo{Why $S$ so special?}

\section{Incorporating Superconductors}
\label{sec:Incorporating Superconductors}

We can consider the Bogoliubov-de Gennes (BdG) Hamiltonians of fermionic
excitations inside superconductors in the same way. Except that
\begin{itemize}
    \item The particle-number $\hat{Q}$ is not conserved
    \item There is a built-in charge-conjugation symmetry, to be
        explained later.
\end{itemize}

The BdG Hamiltonian uses Nambu Spinor:
\begin{equation}
    \hat\chi \equiv \begin{pmatrix}
        \hat\chi_1 \\ \cdots \\ \hat\chi_N \\ 
        \hat\chi_{N+1} \\ \cdots \\ \hat\chi_{2N}
    \end{pmatrix}
    =
    \begin{pmatrix}
        \hat\psi_1 \\ \cdots \\ \hat\psi_N \\
        \hat\psi^\dagger_1 \\ \cdots \\ \hat\psi^\dagger_N
    \end{pmatrix}
    =
    \begin{pmatrix}
        \hat\psi \\ (\hat\psi^\dagger)^t
    \end{pmatrix}
\end{equation}

(It seems that the author is ignoring spin here.)
Here, we regard $\hat\psi^\dagger$ as a row vector, and $\hat\psi$ as
a column vector. The BdG Hamiltonian is written as
\todo{Check the spin indices later.}
\begin{equation}
    \label{eq:super-H}
    \hat{H} =
    \frac{1}{2} \hat{\chi^\dagger} H \hat{\chi} 
    =
    \frac{1}{2} \sum_{A,B=1}^{2N} \hat{\chi^\dagger}_A H_{AB} \hat\chi_B
    = \frac{1}{2} \begin{pmatrix}
        \hat\psi^\dagger & \hat\psi^t
    \end{pmatrix} H 
    \begin{pmatrix}
        \hat\psi \\ (\hat\psi^\dagger)^t
    \end{pmatrix}
\end{equation}

The 1st-quantized Hamiltonian $H$ has the of 4 $N\times N$ blocks:
\begin{equation}
    \label{eq:super-H-def}
    H =
    \begin{bmatrix}
        \Xi & \Delta \\
        \Delta^* & -\Xi^t
    \end{bmatrix}
\end{equation}
where $\Xi$ is Hermitian, and $\Delta=-\Delta^t$ due to fermion
statistics. \todo{Why fermion statistics leads to this?}

Writing explicitly, we have:
\begin{equation}
    \hat{H} = \sum_{a,b=1}^N \hat\psi^\dagger_a \Xi_{ab} \hat\psi_b
    + \frac{1}{2} \sum_{a,b=1}^N \left(
        \hat\psi^\dagger_a \Delta_{ab} \hat\psi^\dagger_b
        + \hat\psi_a \Delta^*_{ab} \hat\psi_b
    \right)
\end{equation}

A striking difference is that $\hat\chi$ and $\hat\chi^\dagger$ are
related by a linear transformation, whereas in non-superconducting
case, $\hat\psi^\dagger$ and $\hat\psi$ are linear-independent.
Explicitly, since $\left(\hat\chi^\dagger \right)^t=
\begin{bmatrix}
    \left(\hat\psi^\dagger\right)^t \\ \hat\psi
\end{bmatrix}$, then clearly:
\begin{equation}
    \left(\hat\chi^\dagger \right)^t = \tau_1 \hat\chi
\end{equation}
where $\tau_1$ interchange the components:
\begin{equation}
    \tau_1 = \begin{bmatrix}
        0               & \id_{N\times N} \\
        \id_{N\times N} & 0
    \end{bmatrix}
\end{equation}

Taking the transpose again gives
\begin{equation}
    \label{eq:super-relat-1}
    \hat\chi^\dagger = \hat\chi^t \tau_1
\end{equation}
and then,
\begin{equation}
    \label{eq:super-relat-2}
    \hat\chi = \tau_1 (\hat\chi^\dagger)^t
\end{equation}
\paragraph{The Charge-conjugation Symmetry of Superconductors} There
is one charge-conjugation symmetry associated with the matrix
$\tau_1$. If one plugs eq.\ref{eq:super-relat-1} and
eq.\ref{eq:super-relat-2} into the Hamiltonian (eq.\ref{eq:super-H}),
one will find (note that $\tr(H)=0$ by eq.\ref{eq:super-H-def}, and be
careful about summation index):
\begin{equation}
    \frac{1}{2} \hat\chi^\dagger H \hat\chi
    = \frac{1}{2} \hat\chi^\dagger (-\tau_1 H \tau_1)^t \hat\chi
\end{equation}
Therefore (note that $H^t=H^*,\tau_1^t=\tau_1,\tau_1^2=\id$):
\begin{equation}
    \tau_1 H^* \tau_1 = -H
\end{equation}
Therefore, the BdG Hamiltonian has built-in charge-conjugation
symmetry, realized by $u_C = \tau_1$.

\paragraph{Encompass non-interacting problem and superconducting
problem} The way to put the two different kind of problems together,
is to work with operator $\hat\Psi$/$\hat\Psi^\dagger$ which has first
few components for ordinary annihilation/creation operators, and the
last few components for superconductor Nambu spinors, and with
$H$ consisting of two diagonal blocks for each respective. 
\todo{This sounds silly, perhaps I do not get some more deep reason?}

\section{TO BE CONTINUED}

\section{License}
The entire content of this work (including the source code
for TeX files and the generated PDF documents) by 
Hongxiang Chen (nicknamed we.taper, or just Taper) is
licensed under a 
\href{http://creativecommons.org/licenses/by-nc-sa/4.0/}{Creative 
Commons Attribution-NonCommercial-ShareAlike 4.0 International 
License}. Permissions beyond the scope of this 
license may be available at 
\href{http://www.google.com/recaptcha/mailhide/d?k=015LguzBJigi0rpyuJRqLoig==\&c=p1c-M-mm7ZcjUCkTuZZa9eEPHRVk6paN0694iazlQy8=}
{[My Email Address(Click)]}.
\bibliography{../../library}{}
\bibliographystyle{alphaurl}
\printnomenclature
\end{document}
