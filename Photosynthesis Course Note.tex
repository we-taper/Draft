% The entire content of this work (including the source code
% for TeX files and the generated PDF documents) by 
% Hongxiang Chen (nicknamed we.taper, or just Taper) is
% licensed under a 
% Creative Commons Attribution-NonCommercial-ShareAlike 4.0 
% International License (Link to the complete license text:
% http://creativecommons.org/licenses/by-nc-sa/4.0/).
\documentclass{article}

\usepackage{float}  % For H in figures
\usepackage{amsmath, amssymb} % For math
\usepackage{mathtools} % dcases*, see https://en.wikibooks.org/wiki/LaTeX/Advanced_Mathematics#The_cases_environment
\numberwithin{equation}{subsection} % have the enumeration go to the subsection level.
									% See:https://en.wikibooks.org/wiki/LaTeX/Advanced_Mathematics
\usepackage{graphicx}   % need for figures
\usepackage{cite} % For bibligraphy
\usepackage{fancyref} % For lazy reference \fref
\usepackage{hyperref} % For hyperlink everything.
\usepackage{CJKutf8} % For Chinese characters
%\usepackage{ dsfont } % For double struck fonts
\usepackage{braket} 
\usepackage[T1]{fontenc}
\usepackage{listings}

\usepackage{amsthm}
\newtheorem{defi}{Definition}[section]
\newtheorem{thm}{Theorem}[section]
\newtheorem{lemma}{Lemma}[section]
\newtheorem{remark}{Remark}[section]
\newtheorem{prop}{Proposition}[section]
\newtheorem{coro}{Corollary}[section]
\theoremstyle{definition}
\newtheorem{ex}{Example}[section]

\title{Photosynthesis Course Note}
\date{\today}
\author{we.taper}
\begin{document}
\maketitle

\tableofcontents

\section{8th, July}
\label{sec:8th_July}
\paragraph{pp.2 of PPT}
In the last lecture, the \textit{master equation} for density matrix
\footnote{Reference not found.}
could lead to problematic result. As the density matrix propagates
as described by the master equation, the eigenvalues for it might
involved into negative values. This is because the added terms
are not rigorously derived, but just an approximation.

\textbf{Digression}:  PRE 65 056120, something about entanglement between
oscillators.

\paragraph{pp.3 of PPT}
\textbf{Lindblad equations} This equation ensures that the
density matrix $\rho \geq 0$ (i.e. having all positive
eigenvalues). However, this operator sometimes breaks down
the symmetries of the system. That is, a system started with
translation symmetry at $t_0$ might not have translation
symmetry at $t > t_0$.

An Example for Lindblad equation:

Considering the case of a harmonic oscillator.
$$ H = w_0 a^\dagger a $$
If we add interaction of oscillator and other excitons:
$$ A^\dagger A (a^\dagger + a) = \text{extra} + \text{bath}$$
Then $A^\dagger A$ is $V_m$ in ppt.


\paragraph{Digression} Two Spin system. One way to judge
whether the two particles are entangled or not, is to do partial
transpose: transposing only part of the density
matrix corresponding to only one particle of the system.

\paragraph{Energy Transfer}

\textbf{Note}: This section's materail could be found in
\textit{Excitonic energy transfer in light-harvesting complexes 
in purple bacteria. JChemPhys\_136\_245104.pdf}

% TODO what is exciton?
\paragraph{pp.5 of PPT}
Exciton transport process. External
influence to the system includes trapping, decaying and disspation.

The $k_t$ characterizes the trapping rate. And $k_d$ characterizes
the decay rate.

\paragraph{pp.6 of PPT}
Life time $\tau_n = \int_0^\infty \rho(t)_n dt$, where
$\rho_n \equiv \rho_{nn}$. We hope the $\braket{t} = \sum \tau_n$
to be small, because the shorter an exciton is fixed on a state,
the more random the system is and the more likely that the exciton
is trapped.

Quantum yeild $q= \text{On PPT} 
            = \frac{\text{trapped}}{\text{trapp} + \text{died}}$.

\paragraph{pp.8 of PPT}
Showcase of calculating the $\rho(t)$ using the Haken-Strobl model
A simple Ring.
% TODO try to calculate the example.

\paragraph{pp.9 of PPT}
Complex models with more and more elements in the ring. 

\paragraph{pp.10 of PPT}
\footnote{Could be found on \textit{Efficient energy transfer in 
light-harvesting systems. Cao and Silbey et al New J Phys (2010)}}
The 16 sites ring shows that there is always 
a peek in the opposite site(See upper-left plot). This is
because there is always
two channels of equal length for it to get to the opposite side.

\paragraph{pp.11 of PPT}
A two ring case, with 8 sites on each ring. The initial configuration
is evenly excited in the left ring. The right ring comprised of
trapping sites.

\paragraph{pp.12 of PPT}
Another two ring case, with a change in the right ring's number
of trapping sites. It shows that a asymmetric design would somehow
imporve the efficiency, since the right side plot has a maximum region.

\paragraph{pp.15 of PPT}

Start with a reduced single-electron density matrix. Consider an
external interruption caused by Laser, and apply the 
\textit{Principle of the nearsightedness of equilibrium systems} to
reduce the computation difficulties. This principle is related 
R.P.A. in condensed matter physics.

\textbf{Caution:} I am quite confused by the following content,
from pp.16 to the end of PPT. Erogo the
following notes are note well organized. It is advised that one should
look at the original paper
\paragraph{pp.16 of PPT}
\footnote{
\url{http://journals.aps.org/pre/pdf/10.1103/PhysRevE.69.032902}} 
instead.

% TODO unclear picture about this slide.
Calculating the absorption spectrum, with the result that only one level
is acceptable.
$A\propto amplitude$ characterize the behavior of laser input. 
$M_{mn}:$ the angle between oscillation angles.

\paragraph{pp.17 of PPT}
Changing the bahavior of lasers could result (in generally) more
different acceptable levels. However, the laser configuration
is not practical.

\paragraph{pp.19 of PPT}
Using Frankel-Exciton Model and find very good fit, balabala.
% TODO transition dipole, dimerization.

\end{document}
