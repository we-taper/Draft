\documentclass{article}

% My own physics package
% The following line load the package xparse with additional option to
% prevent the annoying warnings, which are caused by the package
% "physics" loaded in package "physicist-taper".
\usepackage[log-declarations=false]{xparse}
\usepackage{physicist-taper}
\makenomenclature % For an index of symbols.

% For drawing quantum circuits
\usepackage{qcircuit}

\title{Quantum Machine Learning}
\date{\today}
\author{Taper}


\begin{document}


\maketitle
\abstract{
Hello world! 
}
\tableofcontents

\section{General Measurement}
\label{sec:General Measurement}
\begin{itemize}
  \item General Measurement in Density Matrices
    \begin{itemize}
      \item Defined by $M_1,\cdots, M_m$, with $\sum_i M^\dagger_i M_i = \id$
        (Kraus representation?)
      \item The transformed state is: $\sum_i M_i \rho M_i^\dagger$.
      \item The probability of seeing the item labeled $j$ is,
        \begin{equation}
          p_j = \Tr(M_j \rho M_j^\dagger) = \Tr(M_j^\dagger M_j \rho)
        \end{equation}
      \item The state collapsed into
        \begin{equation}
          \frac{M_j \rho M_j^\dagger}{p_j} = 
          \frac{M_j \rho M_j^\dagger}{\Tr(M_j \rho M_j^\dagger)}
        \end{equation}
    \end{itemize}
  \item Why it is so similar to the statistical description of ensembles, in
    which:
    \begin{equation}
      Z = \Tr(M_j \rho M_j^\dagger)
    \end{equation}
   \item POVMs in Density Matrices:
     \begin{itemize}
       \item We let $E_i = M_i^\dagger M_i$, then
       \item $\sum_i E_i = \id$, $p_j = \Tr(E_j \rho)$,
       \item We don't have an expression for the collapsed state.
     \end{itemize}
\end{itemize}

\section{Discriminating Two Pure States}
\label{sec:Discriminating Two Pure States}
\begin{itemize}
  \item Given $d$-dim state $\psi$ known to be either $\psi_1$ or $\psi_2$.
    Tell which state $\psi$ is in.
  \item Best strategy:
    \begin{itemize}
      \item Do the POVMs with outcome $\{\ket{v_1},\ket{v_2}\}$ such that
      \item \begin{figure}[H]
            \centering
            \includegraphics[width=0.6\linewidth]{fig/dis-two-pure.pdf}
        \end{figure}
      \item The angle $\theta$ is angle between $\psi_i$. The $v_i$ are
        symmetric about the angle bisector of $\psi_i$.
      \item The best probability of success is
        \begin{equation}
          |\braket{\psi_1|v_i}|^2 
          = \cos^2(\frac{\pi/2-\theta}{2}) = \frac{1}{2} + \frac{1}{2}\sin(\theta)
        \end{equation}
    \end{itemize}
  \item ?: Why is this optimal?
  \item ?: How is the probability of success defined?
\end{itemize}

\section{Discriminating Two Mixed States}
\label{sec:Discriminating Two Mixed States}
Given a mixed state $\rho$ in $\rho_1$ or $\rho_2$.

\textbf{Easy case}: 
\begin{itemize}
  \item Assuming equal \textit{a priori} probability.
  \item Assuming $\rho_1=$ ensemble $\{p_1,\psi_i\}$, and $\rho_2=$
    ensemble $\{q_i,\psi_i\}$ (i.e. produced in the same pool of
    $\psi_i$s.).
  \item Optimal:
    \begin{itemize}
      \item Measure in the same basis $\{\psi_i\}$.
      \item If got $\psi_i$, we believe it is $\rho_1$ or $\rho_2$ according to
        $\text{max}(p_i,q_i)$.
      \item Then the overall success is
        \begin{align*}
          P_\text{success} &=
          \frac{1}{2} \sum_i \text{max}(p_i,q_i) \\
          &= \frac{1}{2} \sum_i \left( \frac{p_i+q_i}{2} + \frac{|p_i-q_i|}{2} \right) \\
          &= \frac{1}{2} + \frac{1}{2} \sum_i \frac{|p_i-q_i|}{2} \\
          &= \frac{1}{2} + \frac{1}{2} d_\text{TV}(\{p_i\},\{q_i\})
        \end{align*}
      \item statistical distance/total variation: $d_\text{TV}(\{p_i\},\{q_i\})
        = \sum_i \frac{|p_i-q_i|}{2}$.
    \end{itemize}
  \item Still the optimal is unproven!
\end{itemize}


% \bibliography{library.bib}{}
% \bibliographystyle{alphaurl}
% \printnomenclature

\end{document}
