
\chapter{Local Theory}
\label{chap:Local Theory}

\section{1.1 Holomorphic Functions of Several Variables}
\label{sec:book_1.1}

\begin{prop}
    The local ring $\mathcal{O}_{\mathbb{C}^n,0}$ is a UFD.
\end{prop}
(\textbf{pp.14 of \cite{book}})

\begin{prop}{Weierstrass division theorem}
    Let $f\in \mathcal{O}_{\mathbb{C}^n,0}$ and $g\in
    \mathcal{O}_{\mathbb{C}^{n-1},0}[z_1]$ be a Weierstrass
    polynomial of degree $d$. Then there exist $r\in
    \mathcal{O}_{\mathbb{C}^{n-1},0}[z_1]$ of degree $<d$
    and $h\in \mathcal{O}_{\mathbb{C}^n,0}$ such that $f=g\cdot h+r$.
    The functions $h$ and $r$ are uniquely determined.
\end{prop}
(\textbf{pp.15 of \cite{book}})

\begin{prop}
    The local UFT $\mathcal{O}_{\mathbb{C}^n,0}$ is Noetherian.
\end{prop}
(\textbf{pp.16 of \cite{book}})

\begin{coro}
    Let $g\in \mathcal{O}_{\mathbb{C}^n,0}$ be an irreducible function.
    If $f\in\mathcal{O}_{\mathbb{C}^n,0}$ vanishes on $Z(g)$, then
    $g$ divides $f$.
\end{coro}
(\textbf{pp.16 of \cite{book}})

\begin{lemma}
    For any germ $X\subset \mathbb{C}^n$ the set $I(X)\subset
    \mathcal{O}_{\mathbb{C}^n,0}$ is an ideal.
    If $(A)\subset \mathcal{O}_{\mathbb{C}^n,0}$ denotes the ideal
    generated by the subset $A\subset \mathcal{O}_{\mathbb{C}^n,0}$,
    then $Z(A)=Z((A))$ and $Z(A)$ is analytic.
\end{lemma}
(\textbf{pp.18 of \cite{book}})

\begin{lemma}
    If $X_1\subset X_2$, then $I(X_2)\subset I(X_1)$. If $I_1\subset I_2$,
    then $Z(I_2)\subset Z(I_1)4$. For any analytic germ $X$ one has
    $Z(I(X))=X$. For any ideal $I\subset \mathcal{O}_{\mathbb{C}^n,0}$,
    one has $I\subset I(Z(I))$.
\end{lemma}
(\textbf{pp.18 of \cite{book}})
% TODO bookmark this chapter stops at page 18.

\section{1.2 Complex and Hermitian Structures}
\begin{lemma}
    If $I$ is an almost complex structure on a real vector space $V$,
    then $V$ admits in a natural way the structure of a complex vector
    space
\end{lemma}
(\textbf{pp.25 of \cite{book}})

\begin{remark}
    An almost complex structure can only exist on an even dimensional
    real vector space.
\end{remark}

\begin{coro}
    Any almost complex structure on $V$ induces a natural orientation
    on $V$.
\end{coro}
(\textbf{pp.25 of \cite{book}})

\begin{lemma}
    Let $V$ be a real vector space endowed with an almost complex
    structure $I$. Then
    $$ V_{\mathbb{C}} = V^{1,0}\oplus V^{0,1}$$
    Complex conjugation on $V_{\mathbb{C}}$ induces an $\mathbb{R}$-linear
    isomorphism $V^{1,0}\cong V^{0,1}$.
\end{lemma}
(\textbf{pp.26 of \cite{book}})

\begin{remark}
    Two almost complex structures on $V_{\mathbb{C}}$: $I$ and $i$,
    coincide on the subspace $V^{1,0}$ but differ by a sign on
    $V^{0,1}$.
\end{remark}

\begin{lemma}
    Let $V$ be a real vector space endowed with an almost complex
    structure $I$. Then the dual space 
    $V^*=\text{Hom}_{\mathbb{R}}(V,\mathbb{R})$ has a natural almost
    complex structure given by $I(f)(v)=f(I(v))$. The induced 
    decomposition on 
    $(V^*)_{\mathbb{C}}=\text{Hom}_{\mathbb{R}}(V,\mathbb{C})
    = (V_{\mathbb{C}})^*$ is given by
    $$ (V^*)^{1,0} = \{f\in \text{Hom}_{\mathbb{R}}(V,\mathbb{C}) |
        f(I(v)) = i f(v) \} = (V^{1,0})^*$$
    $$ (V^*)^{0,1} = \{f\in \text{Hom}_{\mathbb{R}}(V,\mathbb{C}) |
        f(I(v)) = -i f(v) \} = (V^{0,1})^*$$
    Also note that $(V^*)^{1,0} = 
        \text{Hom}_{\mathbb{C}}((V,I),\mathbb{C})$.
\end{lemma}

\begin{prop}
    For a real vector space $V$ endowed with an almost complex
    structure $I$, one has:
    \begin{enumerate}
        \item $\bigwedge^{p,q}V$ is in a canonical way a subsapce
            of $\bigwedge^{p+q} V_{\mathbb{C}}$.
        \item $\bigwedge^k V_{\mathbb{C}}
            = \bigoplus_{p+q=k}\bigwedge^{p,q}V$.
        \item Complex conjugation on $\bigwedge^* V_{\mathbb{C}}$ defines
            a ($\mathbb{C}$-linear) isomorphism
            $\bigwedge^{p,q}V\cong \bigwedge^{q,p}V$, i.e.
            $\bar{\bigwedge^{p,q}V}=\bigwedge^{q,p}V$.
        \item The exterior prodoct is of bidegree $(0,0)$.
    \end{enumerate}
\end{prop}
(\textbf{pp.27 of \cite{book}})
\begin{remark}{Local calculation of $V^{1,0}$,$(V^*)^{1,0}$}
    $$z_i=\frac{1}{2}(x_i-y_i)\text{, }\bar{z}_i=\frac{1}{2}(x_i+iy_i)$$
    $$z^i=x^i+iy^i\text{, }\bar{z}^i=x^i-iy^i$$
    $$I(z_i)=i z_i\text{, }I(z^i)=i z^i$$
\end{remark}
(\textbf{pp.27 to 28 of \cite{book}})

\begin{lemma}
    For any $m\leq \text{dim}_{\mathbb{C}}V^{1,0}$, one has
    $$(-2i)^m (z_1\wedge\bar{z}_1)\wedge\cdots\wedge(z_m\wedge\bar{z}_m)=
        (x_1\wedge y_1)\wedge\cdots\wedge(x_m\wedge y_m).$$
    For $m=\text{dim}_\mathbb{C} V^{1,0}$, this defines a positive
    oriented volume form for the natural orientation of $V$.

    Also
    $$\left(\frac{i}{2}\right)^m
      (z^1\wedge\bar{z}^1)\wedge\cdots\wedge(z^m\wedge\bar{z}^m)=
      (x^1\wedge y^1)\wedge\cdots\wedge(x^m\wedge y^m).$$
\end{lemma}

\begin{prop}[Lefschetz decomposition]
    There exists a direct sum decomposition of the form:
    \begin{align}
        \bigwedge^kV^* = \bigoplus_{i\geq 0} L^i(P^{k-2i})
    \end{align}
    Also, $P^k={\alpha\in \bigwedge^k V^*|L^{n-k+1}\alpha=0}$, for
    $k\leq n$. Naturally $P^k=0$ for $k>0$.

    We also have several morphisms induced by $L$, which is illustrated
    in the following graph adapted from the book:

    \begin{figure}[H]
        \centering
        \includegraphics[width=0.9\linewidth]{pics/{prop-1.2.30-pp.36}.pdf}
        \caption{Morphisms}
    \end{figure}
\end{prop}
(\textbf{pp.36 of \cite{book}})

As shown in the theorem, the map $\Lambda^{n-k}$ is produce a mirror
effect in $\bigwedge^*V^*$, very similar to the Hodge $*$.
The next proposition relates the two:

\begin{prop}
    For all $\alpha\in P^k$, we have:
    \begin{align}
        *L^j \alpha = (-1)^{\frac{k(k+2)}{2}}
            \frac{j!}{(n-k-j)!}\cdot L^{n-k-j}I(\alpha).
    \end{align}
\end{prop}
Particularly, when $j=k=0$, we have $*1=\text{vol} = \frac{\omega^n}{n!}$,
or,
\begin{align}
    n!\text{vol} = \omega^n
\end{align}
(\textbf{pp.37 of \cite{book}})

\begin{coro}[Hodge—Riemann bilinear relation]
    \begin{align}
        Q(\bigwedge^{p,q}V^*, \bigwedge^{p',q'}V^*)=0
    \end{align}
    for $(p,q)\neq (p',q')$, and
    \begin{align}
        i^{p-q}Q(\alpha,\bar\alpha)=(n-(p+q))! \cdot
        \langle\alpha,\alpha\rangle_{\mathbb{C}}>0
    \end{align}
    for $0\neq\alpha\in P^{p,q}$, with $p+q\leq n$.
\end{coro}
(\textbf{pp.39 of \cite{book}})
