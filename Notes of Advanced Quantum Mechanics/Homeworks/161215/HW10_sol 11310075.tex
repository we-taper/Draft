% The entire content of this work (including the source code
% for TeX files and the generated PDF documents) by 
% Hongxiang Chen (nicknamed we.taper, or just Taper) is
% licensed under a 
% Creative Commons Attribution-NonCommercial-ShareAlike 4.0 
% International License (Link to the complete license text:
% http://creativecommons.org/licenses/by-nc-sa/4.0/).
\documentclass{article}

% My own physics package
% The following line load the package xparse with additional option to
% prevent the annoying warnings, which are caused by the package
% "physics" loaded in package "physicist-taper".
\usepackage[log-declarations=false]{xparse}
\usepackage{physicist-taper}

\title{Solution for HW10}
\date{\today}
\author{Taper}

\begin{document}
\maketitle
\abstract{
\begin{CJK}{UTF8}{gbsn}陈鸿翔(11310075)\end{CJK}
}
%\tableofcontents
\section*{Problem 1}
\begin{tabular}{l|r}
    Classcal & $2^2=4$ \\
    Boson    & $3$ \\
    Fermion & $1$
\end{tabular}

\section*{Problem 2}
Notice two facts:
\begin{equation}
    (\vec{\sigma}\vdot\vec{a})(\vec{\sigma}\vdot\vec{b})
    = \vec{a}\vdot\vec{b}+i\vec{\sigma}\vdot(\vec{a}\cross\vec{b})
\end{equation}
which is Eq.(3.2.39) on Sakurai's Modern Quantum Mechanics. Also
\begin{equation}
    \Tr(\vec{a}\vdot\vec{\sigma})=0
\end{equation}
This is because the trace function is "almost" linear and all pauli
matrices have zero trace.

Then:
\begin{align*}
    \Tr(\rho_A\rho_B)
    &=
    \frac{1}{4}\Tr(
        1+(\vec{n}_A+\vec{n}_B)\vdot\vec{\sigma}
        + (\vec{n}_A\vdot\vec{\sigma})(\vec{n}_B\vdot\vec{\sigma})
    )\\
    &=
    \frac{1}{4}\left\{
        \Tr(1+
            \vec{n}_A\vdot\vec{n}_B +
            i\vec{\sigma}\vdot(\vec{n}_A\cross\vec{n}_B)
    )\right\}\\
    &=\frac{1}{4}\Tr(1+\vec{n}_A\vdot\vec{n}_B)
    \\
    &=\frac{1}{2}(1+\vec{n}_A\vdot\vec{n}_B)
\end{align*}

\section*{Problem 3}
Denote $\ket{a'}$ as $ \begin{pmatrix}
    1\\0
\end{pmatrix}$, and $\ket{a''}$ as $ \begin{pmatrix}
    0\\1
\end{pmatrix}$. Then
\begin{equation}
    H= \begin{pmatrix}
        0 & \delta \\
        \delta & 0
    \end{pmatrix}
\end{equation}
The eigenvectors and eigenvalues can be easily guessed:
\begin{align}
    \ket{+}&=\frac{1}{\sqrt{2}}\begin{pmatrix}
        1\\1
    \end{pmatrix}, \lambda_+ = \delta
    \\
    \ket{-}&=\frac{1}{\sqrt{2}}\begin{pmatrix}
        1\\-1
    \end{pmatrix}, \lambda_- = -\delta
\end{align}

For time evolution:

Since $\ket{a'}=\frac{\sqrt{2}}{2} (\ket{+}+\ket{-})$, under time
evolution it will be
\begin{equation}
    \ket{a'(t)} = \frac{\sqrt{2}}{2} \left(
        e^{-i\delta t/\hbar}\ket{+}
        + e^{i\delta t/\hbar}\ket{-}
    \right)
    = \frac{1}{2}
    \begin{pmatrix}
        e^{it\delta/\hbar}+e^{-it\delta/\hbar} \\
        e^{-it\delta/\hbar}-e^{it\delta/\hbar} \\
    \end{pmatrix}
    = \begin{pmatrix}
        \cos(t\delta/\hbar) \\ -i\sin(t\delta/\hbar)
    \end{pmatrix}
\end{equation}

For probability:

The probability is clearly: $\sin^2(t\delta/\hbar)$.

\begin{remark}
    This is actually the famous Rabi frequency, see more at:
    \url{https://en.wikipedia.org/wiki/Rabi_cycle}.
\end{remark}
\end{document}
