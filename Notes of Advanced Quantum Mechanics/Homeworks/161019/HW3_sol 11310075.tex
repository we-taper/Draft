% The entire content of this work (including the source code
% for TeX files and the generated PDF documents) by 
% Hongxiang Chen (nicknamed we.taper, or just Taper) is
% licensed under a 
% Creative Commons Attribution-NonCommercial-ShareAlike 4.0 
% International License (Link to the complete license text:
% http://creativecommons.org/licenses/by-nc-sa/4.0/).
\documentclass{article}

\usepackage{float}  % For H in figures
\usepackage{amsmath} % For math
\usepackage{amssymb}
\usepackage{bbm} % for numbers within mathbb
\usepackage{mathrsfs} % For \mathscr{ABC}
% Followings are for the special character: differential "d".
\newcommand*\diff{\mathop{}\!\mathrm{d}}
\newcommand*\Diff[1]{\mathop{}\!\mathrm{d^#1}}
\numberwithin{equation}{subsection} % have the enumeration go to the subsection level.
                                    % See:https://en.wikibooks.org/wiki/LaTeX/Advanced_Mathematics
\usepackage{graphicx}   % need for figures
\usepackage{cite} % need for bibligraphy.
\usepackage[unicode]{hyperref}  % make every cite a link
\usepackage{CJKutf8} % For Chinese characters
\usepackage{fancyref} % For easy adding figure,equation etc in reference. Use \fref or \Fref instead of \ref
\usepackage{braket} %http://tex.stackexchange.com/questions/214728/braket-notation-in-latex

% Following is for theorems etc environments
% http://tex.stackexchange.com/questions/45817/theorem-definition-lemma-problem-numbering && https://en.wikibooks.org/wiki/LaTeX/Theorems
\usepackage{amsthm}
\newtheorem{defi}{Definition}[section]
\newtheorem{thm}{Theorem}[section]
\newtheorem{lemma}{Lemma}[section]
\newtheorem{remark}{Remark}[section]
\newtheorem{prop}{Proposition}[section]
\newtheorem{coro}{Corollary}[section]
\theoremstyle{definition}
\newtheorem{ex}{Example}[section]


\usepackage{CJKutf8}

% A list of nomenclatures.
\usepackage{nomencl}
\makenomenclature

% For drawing diagrams with arrows
\usepackage[all]{xy}

\title{Solution for HW3 20161019}
\date{\today}
\author{Taper}

\begin{document}

\maketitle
\abstract{
\begin{CJK}{UTF8}{gbsn}陈鸿翔(11310075)\end{CJK}
}
%\tableofcontents
\section{Give the brief description of quantum state and operator in
Hilbert space}

\paragraph{Quantum state in Hilbert space}: each quantum state
correspondes to a vector of unit length in Hilbert space.

\paragraph{Operator in Hilbert space}: each operator coorespondes to
an invertible linear transformation (or a transformation of basis) in
Hilbert space. In rare case, an operator may coorespondes to an
invertible anti-linear transformation in Hilbert space, such as the
time reversal opertor.

\section{Proof}

$(1)$:
\begin{proof}
\begin{align*}
    \text{LHS} &= \left( \begin{array}{c}
                     a_1 + b_1\\
                     a_2 + b_2\\
                \end{array} \right)^\dagger * 
                \left( \begin{array}{c}
                     c_1 \\
                     c_2 \\
                \end{array} \right) \\
                &= (a_1^*+b_1^*)c_1 + (a_2^*+b_2^*)c_2
\end{align*}
\begin{align*}
    \text{RHS} &= (a_1^*c_1+a_2^* c_2) + (b_1^*c_1+b_2^* c_2)  \\
               &= (a_1^*+b_1^*)c_1 + (a_2^*+b_2^*)c_2 \\
               &= \text{LHS}
\end{align*}
\end{proof}

$(2)$:
\begin{proof}
\begin{align*}
    \text{LHS} &= \left( \begin{array}{c}
                     a_1 + b_1\\
                     a_2 + b_2\\
                \end{array} \right) * \left( \begin{array}{cc}
                     c_1^* & c_2^* \\
                \end{array} \right) \\
               &= \left( \begin{array}{cc}
                    (a_1+b_1)c_1^* & (a_1+b_1)c_2^* \\
                    (a_2+b_2)c_1^* & (a_2+b_2)c_2^* \\
                \end{array} \right)
\end{align*}
\begin{align*}
    \text{RHS} &= \left( \begin{array}{cc}
                    a_1c_1^* & a_1c_2^* \\
                    a_2c_1^* & a_2c_2^* \\
                \end{array} \right) + \left( \begin{array}{cc}
                    b_1c_1^* & b_1c_2^* \\
                    b_2c_1^* & b_2c_2^* \\
                \end{array} \right) \\
               &= \left( \begin{array}{cc}
                    (a_1+b_1)c_1^* & (a_1+b_1)c_2^* \\
                    (a_2+b_2)c_1^* & (a_2+b_2)c_2^* \\
                \end{array} \right) \\
               &= \text{LHS}
\end{align*}
\end{proof}

\section{Check}
\paragraph{$A^\dagger=$}
\begin{align*}
    \left( \begin{array}{ccc}
     0 & 0 & i \\
     0 & 1 & 0 \\
     -i & 0 & 0 \\
    \end{array} \right)^\dagger = 
    \overline{\left( \begin{array}{ccc}
     0 & 0 & -i \\
     0 & 1 & 0 \\
     i & 0 & 0 \\
    \end{array} \right)} = 
    \left( \begin{array}{ccc}
     0 & 0 & i \\
     0 & 1 & 0 \\
     -i & 0 & 0 \\
    \end{array} \right) = A
\end{align*}
\paragraph{$B^\dagger=$}
\begin{align*}
    \left( \begin{array}{ccc}
     3 & i & 0 \\
     3 & 1 & 5 \\
     0 & -i & 2 \\
    \end{array} \right)^\dagger = 
    \overline{\left( \begin{array}{ccc}
     3 & 3 & 0 \\
     i & 1 & -i \\
     0 & 5 & 2 \\
    \end{array} \right)} = 
    \left( \begin{array}{ccc}
     3 & 3 & 0 \\
     -i & 1 & i \\
     0 & 5 & 2 \\
    \end{array} \right) \neq B
\end{align*}

So $A$ is Hermitian whereas $B$ is not.
\end{document}
