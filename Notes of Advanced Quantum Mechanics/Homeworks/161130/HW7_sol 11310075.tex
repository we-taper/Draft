% The entire content of this work (including the source code
% for TeX files and the generated PDF documents) by 
% Hongxiang Chen (nicknamed we.taper, or just Taper) is
% licensed under a 
% Creative Commons Attribution-NonCommercial-ShareAlike 4.0 
% International License (Link to the complete license text:
% http://creativecommons.org/licenses/by-nc-sa/4.0/).
\documentclass{article}

% My own physics package
% The following line load the package xparse with additional option to
% prevent the annoying warnings, which are caused by the package
% "physics" loaded in package "physicist-taper".
\usepackage[log-declarations=false]{xparse}
\usepackage{physicist-taper}

\title{Solution for HW7}
\date{\today}
\author{Taper}

\begin{document}

\maketitle
\abstract{
\begin{CJK}{UTF8}{gbsn}陈鸿翔(11310075)\end{CJK}
}
%\tableofcontents
\section{Prove}
\label{sec:Prove}
$$  [x,p^n] = in\hbar p^{n-1} $$

\begin{proof}
    \begin{align*}
        [x,p^n] &= p[x,p^{n-1}] + [x,p]p^{n-1} 
            = p[x,p^{n-1}]+i\hbar p^{n-1} \\
        &= p[x,p^{n-2}]+ i\hbar p^{n-1} + i\hbar p^{n-1} \\
        &\cdots \\
        &= ni\hbar p^{n-1}
    \end{align*}
\end{proof}
$$ [p,x^n] = -in\hbar x^{n-1} $$
\begin{proof}
\begin{align*}
    [p,x^n] &= x[p,x^{n-1}] + [p,x]x^{n-1} = x[p,x^{n-1}]  -i\hbar x^{n-1}
    \\
    &= x[p,x^{n-1}] -i\hbar x^{n-1} - i\hbar x^{n-1} \\
    &\cdots \\
    &= -in\hbar x^{n-1}
\end{align*}
\end{proof}

$$ [x,f(p)] = i\hbar \frac{\partial f(p)}{\partial p} $$
\begin{proof}
    Any smooth function of $p$ can be taylor-expanded into a series of
    the form $\sum_n c_n p^n$, since the commutator is linear and 
    $[x,p^n]=i\hbar np^{n-1}$, one see the result will be of the form
    $i\hbar \sum_n c_{n} n p^{n-1}$, which is exactly the expansion of
    the taylor expansion of $\frac{\partial f(p)}{\partial p}$. This
    observation concludes the proof.
\end{proof}

$$ [p,g(x)] = -i\hbar \frac{\partial f(x)}{\partial x} $$

\begin{proof}
    The proof is exactly the same of the one above, if one do a
    transformation $x\to p$, $p\to x$, and $i\to -i$.
\end{proof}

\section*{Problem 2}
\label{sec:Problem 2}
\begin{align}
    A=\left( \begin{array}{ccc}
         a & 0 & 0 \\
         0 & -a & 0 \\
         0 & 0 & -a \\
    \end{array} \right), B=\left( \begin{array}{ccc}
                             b & 0 & 0 \\
                             0 & 0 & \text{ib} \\
                             0 & -\text{ib} & 0 \\
                        \end{array} \right)
\end{align}
The eigenvalues of $A$ are obviously $a,-a,-a$. The eigenvalues of $B$
is found by characteristic equations $\mathrm{Det}(B-\lambda \id)=0$ to be
$\{-b,b,b\}$. So $B$'s spectrum is degenerate too.

It can be found that
\begin{equation*}
    AB = \left( \begin{array}{ccc}
         a b & 0 & 0 \\
         0 & 0 & -i a b \\
         0 & i a b & 0 \\
        \end{array} \right), \\
    BA = \left( \begin{array}{ccc}
         a b & 0 & 0 \\
         0 & 0 & -i a b \\
         0 & i a b & 0 \\
        \end{array} \right), \\
\end{equation*}

The eigenvectors of $A$ are easy to guess, they are
just the three natural basis $\{e_1,e_2,e_3\}$. 

It is easy to find (by solving $BX=\lambda_i X$) $B$'s eigenvectors:
$$
\begin{pmatrix}
    1\\ 0\\ 0
\end{pmatrix}
\begin{pmatrix}
    0\\ -i\\ 1
\end{pmatrix}
\begin{pmatrix}
    0\\ i\\ 1
\end{pmatrix}
$$
Since both are degenerate, we can linearly combine the degenerate
eigenvectors of $A$ to form that of $B$, but since $A$'s eigenvectors
are natural basis, and in the only non-degenerate case, the two matrix
share the same eigenvector (notice also that those degenerate
eigenvectors of $B$ does not "mess up" with the non-degenerate
eigenvector). So we do not have to do anything and those
$B$'s eigenvectors will also diagolize $A$. They are the required
eigenvectors.
\end{document}
