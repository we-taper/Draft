% The entire content of this work (including the source code
% for TeX files and the generated PDF documents) by 
% Hongxiang Chen (nicknamed we.taper, or just Taper) is
% licensed under a 
% Creative Commons Attribution-NonCommercial-ShareAlike 4.0 
% International License (Link to the complete license text:
% http://creativecommons.org/licenses/by-nc-sa/4.0/).
\documentclass{article}

\usepackage{float}  % For H in figures
\usepackage{amsmath} % For math
\usepackage{amssymb}
\usepackage{bbm} % for numbers within mathbb
\usepackage{mathrsfs} % For \mathscr{ABC}
% Followings are for the special character: differential "d".
\newcommand*\diff{\mathop{}\!\mathrm{d}}
\newcommand*\Diff[1]{\mathop{}\!\mathrm{d^#1}}
\numberwithin{equation}{subsection} % have the enumeration go to the subsection level.
                                    % See:https://en.wikibooks.org/wiki/LaTeX/Advanced_Mathematics
\usepackage{graphicx}   % need for figures
\usepackage{cite} % need for bibligraphy.
\usepackage[unicode]{hyperref}  % make every cite a link
\usepackage{CJKutf8} % For Chinese characters
\usepackage{fancyref} % For easy adding figure,equation etc in reference. Use \fref or \Fref instead of \ref
\usepackage{braket} %http://tex.stackexchange.com/questions/214728/braket-notation-in-latex

% Following is for theorems etc environments
% http://tex.stackexchange.com/questions/45817/theorem-definition-lemma-problem-numbering && https://en.wikibooks.org/wiki/LaTeX/Theorems
\usepackage{amsthm}
\newtheorem{defi}{Definition}[section]
\newtheorem{thm}{Theorem}[section]
\newtheorem{lemma}{Lemma}[section]
\newtheorem{remark}{Remark}[section]
\newtheorem{prop}{Proposition}[section]
\newtheorem{coro}{Corollary}[section]
\theoremstyle{definition}
\newtheorem{ex}{Example}[section]


\usepackage{CJKutf8}

% A list of nomenclatures.
\usepackage{nomencl}
\makenomenclature

% For drawing diagrams with arrows
\usepackage[all]{xy}

\title{Solution for HW2 20161013}
\date{\today}
\author{Taper}

\begin{document}

\maketitle
\abstract{
\begin{CJK}{UTF8}{gbsn}陈鸿翔(11310075)\end{CJK}
}
%\tableofcontents
\section{Describe wave function}
Wave functgion is a function that is assumed to be describing the
physical system. It is assumed that all physical information can be
extracted from it, which is saying that a wave function is the
complete description of the physical system it is complete. And we
calculate any observable $A$ by calculating its expectation value
using the wave function.

As for the probability wave, I thick this is kind of a misnomer. It is
the wave function that propagates in the form of waves, as can be seen
by the following formal solution to the schrodinger equation:

\begin{align}
    i\hbar \frac{\partial}{\partial t} \Psi = H \Psi \nonumber\\
    \Rightarrow \Psi(t) = e^{-i\hbar H} \Psi(t=t_0)
\end{align}

However, the probability is determined by the absolute value of
wavefunction, so it does not exactly propagate like a wave. 

\section{Eigensystems}

$\bullet e^x$: $\frac{\diff^2}{\diff x^2} e^x = e^x$, so it IS an
eigenfunction, with eigenvalue $1$.

$\bullet \sin(x)$: $\frac{\diff^2}{\diff x^2} \sin(x) = -\sin(x)$, so
it IS an eigenfunction, with eigenvalue $-1$.

$\bullet 2\cos(x)$: $\frac{\diff^2}{\diff x^2} 2\cos(x) = 2
(-\cos(x)) = - (2\cos(x))$, so it IS an eigenfunction, with eigenvalue
$-1$.

$\bullet \sin^2(x)$: $\frac{\diff^2}{\diff x^2} \sin^2(x) =
\frac{\diff}{\diff x}2 \cos(x) = -2 \sin(x)$, so it is NOT an
eigenfunction.

$\bullet x^3$: $\frac{\diff^2}{\diff x^2} x^3 = 6x$, so it is NOT an
eigenfunction.

$\bullet \sin(x)+\cos(x)$: the combination of two eigenfunction with
the same eigenvalue is of course an eigenfunction, with eigenvalue
$-1$.

\section{1D infinite potential well}
The time-independent Schrodinger equation inside the wall is:
$$ -\frac{\hbar^2}{2m} \frac{\diff^2}{\diff x^2} \psi = E\psi$$,
so clearly
$$\psi = A \sin(kx) + B\cos(kx)$$
inside the wall. Here $k=\sqrt{\frac{2mE}{\hbar^2}}$.

Outside the wall the potential is infinite, not suitable for
wavefunction to live. So $\psi=0$ outside. Connecting the wave
function in the boundary will clearly leads to dicretized wavelength.
But the give coordinate is not convenient to determine the parameters
$A,B$. So I shift the origin leftwards for $\frac{a}{2}$. Then the
potential becomes:
$$
U = \begin{cases}
    0& 0\leq x\leq a \\
    \infty& \text{otherwise}
\end{cases}
$$
Then we have $\psi(0)=0$, hence $B=0$. And $\psi(a)=0$, hence
$$ \sqrt{\frac{2mE}{\hbar^2}} a = n\pi \Rightarrow$$
$$ E = \frac{n^2\pi^2\hbar^2}{2ma^2}$$
where $n=1,2,\cdots$. $A$ is determined by normalization, but I don't
have to calculate because the next exercise has already given the
answer: $A = \sqrt{\frac{2}{a}}$. Now shift the coordinate back, I
have
\begin{align}
    \psi(x) &= \sqrt{\frac{2}{a}}\sin\left(k(x-\frac{a}{2})\right)
            =\sqrt{\frac{2}{a}}
                \sin\left(\frac{n\pi}{a}(x-\frac{a}{2})\right)\\
    k       &= \sqrt{\frac{2mE}{\hbar^2}}\\
    E       &= \frac{n^2\pi^2\hbar^2}{2ma^2}
\end{align}

\section{Prove the orthogonality}
This is simple:
\begin{align*}
    &\int_{0}^{l} \sin(\frac{n\pi x}{l}) \sin(\frac{m\pi x}{l})\\
    =& \int_{0}^{l} \frac{ 
        \cos(\frac{(n-m)\pi x}{l}) - \cos(\frac{(n-m)\pi x}{l})
        } {2}
\end{align*}
Notice that when $n\neq m$, both $\cos(\frac{(n-m)\pi x}{l})$ and
$\cos(\frac{(n-m)\pi x}{l})$ oscillate with the period of $l$. That
means the integration is taken over a whole peroid. So the result is
$0$, i.e.:
\begin{align}
    \int_{0}^{l} \sin(\frac{n\pi x}{l}) \sin(\frac{m\pi x}{l})=0
\end{align}
for $m\neq n$.
\end{document}
