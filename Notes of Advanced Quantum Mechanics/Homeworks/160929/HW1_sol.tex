% The entire content of this work (including the source code
% for TeX files and the generated PDF documents) by 
% Hongxiang Chen (nicknamed we.taper, or just Taper) is
% licensed under a 
% Creative Commons Attribution-NonCommercial-ShareAlike 4.0 
% International License (Link to the complete license text:
% http://creativecommons.org/licenses/by-nc-sa/4.0/).
\documentclass{article}

\usepackage{float}  % For H in figures
\usepackage{amsmath} % For math
\usepackage{amssymb}
\usepackage{bbm} % for numbers within mathbb
\usepackage{mathrsfs} % For \mathscr{ABC}
% Followings are for the special character: differential "d".
\newcommand*\diff{\mathop{}\!\mathrm{d}}
\newcommand*\Diff[1]{\mathop{}\!\mathrm{d^#1}}
\numberwithin{equation}{subsection} % have the enumeration go to the subsection level.
                                    % See:https://en.wikibooks.org/wiki/LaTeX/Advanced_Mathematics
\usepackage{graphicx}   % need for figures
\usepackage{cite} % need for bibligraphy.
\usepackage[unicode]{hyperref}  % make every cite a link
\usepackage{CJKutf8} % For Chinese characters
\usepackage{fancyref} % For easy adding figure,equation etc in reference. Use \fref or \Fref instead of \ref
\usepackage{braket} %http://tex.stackexchange.com/questions/214728/braket-notation-in-latex

% Following is for theorems etc environments
% http://tex.stackexchange.com/questions/45817/theorem-definition-lemma-problem-numbering && https://en.wikibooks.org/wiki/LaTeX/Theorems
\usepackage{amsthm}
\newtheorem{defi}{Definition}[section]
\newtheorem{thm}{Theorem}[section]
\newtheorem{lemma}{Lemma}[section]
\newtheorem{remark}{Remark}[section]
\newtheorem{prop}{Proposition}[section]
\newtheorem{coro}{Corollary}[section]
\theoremstyle{definition}
\newtheorem{ex}{Example}[section]


\usepackage{CJKutf8}

% A list of nomenclatures.
\usepackage{nomencl}
\makenomenclature

% For drawing diagrams with arrows
\usepackage[all]{xy}

\title{Solution for HW1 20160929}
\date{\today}
\author{Taper}

\begin{document}

\maketitle
\abstract{
\begin{CJK}{UTF8}{gbsn}陈鸿翔(11310075)\end{CJK}
}
%\tableofcontents
\section{1.Describe and explain (or derive) the following concepts}

\paragraph{Photoelectric effect} is the effect when light is shone on
metals, there will be currents induced. Also, there exists an lower limits
of light frequency, which is independent of light intensity, for this to
happen. This is because the energy of a single photon is determined by its
frequency.

\paragraph{Compton effect} says that the wavelength of light (especially
the x-ray), after a collision and scattering with electron, changes. This
shows that light also behaves like an ordinary particle when coliding with
other particles.

\section{Problem 2}
\paragraph{Normalize}
\begin{align}
    \int_{-\infty }^{\infty }
      &e^{-\lambda *|x|}e^{-i\omega t}*e^{-\lambda *|x|}e^{i\omega t}dx
      =2 \int_0^{\infty } e^{-2 \lambda x} \, dx
      =\frac{2}{2 \lambda }=\frac{1}{\lambda }
\end{align}
So $A=\sqrt{\lambda}$ (up to an irrelavent phase factor).
\paragraph{$\braket{x}$}=
\begin{align}
    \int _{-\infty }^{\infty }\lambda *e^{-\lambda *|x|}
        e^{-i\omega t}*x*e^{-\lambda *|x|}e^{i\omega t}dx
    =\lambda \int _{-\infty }^{\infty }e^{-2 \lambda |x|}x\, dx=0
\end{align}
\paragraph{$\braket{x^2}$}=
\begin{align}
    \int _{-\infty }^{\infty }
      \lambda*e^{-\lambda *|x|}e^{-\text{i$\omega $t}}
      *x^2*e^{-\lambda *|x|}e^{\text{i$\omega $t}}\, dx
    =2 \lambda  \int_0^{\infty } e^{-2 \text{$\lambda $x}} x^2 \, dx
    =2\lambda *\frac{1}{4}
      \frac{\partial ^2}{\partial \lambda ^2}\frac{1}{2 \lambda }
    =\frac{1}{2 \lambda ^2}
\end{align}
\paragraph{$\sigma $}=
\begin{align}
    \sqrt{\braket{x^2}-\braket{x}^2}=\frac{\sqrt{2}}{2}\frac{1}{\lambda}
\end{align}
Plot when $\lambda=1$, $|\Psi|^2=\exp(-2|x|)$:
\begin{figure}[H]
    \centering
    \includegraphics[width=0.8\linewidth]{{plot-1}.eps}
    %\caption{{plot-1}.eps}
\end{figure}

5. the probability that the particle would be found outside this range
$\braket{ \braket{x}-\sigma,\braket{x}+\sigma }$:

\begin{align}
    &\int_{\braket{x}-\sigma}^{\braket{x}+\sigma} |\Psi(x)|^2 \diff x
    = \int_{-\sigma}^{\sigma}e^{-2\lambda |x|} \lambda \diff x
    = 2\lambda \int_0^{\sigma} e^{-2\lambda x} \diff x \nonumber\\
    &= 2\lambda \frac{1}{2\lambda} (1-e^{-2\lambda\sigma})
    = 1-e^{-2\lambda\sigma}
\end{align}
So the probability that the particle fall out of one sigma away from the
center is: $e^{-2\lambda\sigma}$.

\section{Problem 3}
Derive the Compton shift formula:
$\Delta\lambda=\frac{h}{mc}(1-\cos{\phi})$:
\begin{proof}

    Let $e$ for electron, $h$ for photon. Primed for  Conservation of energy: 
    \begin{align}
        \label{eq:p.1}
        hv+ m_ec^2 = hv' + \sqrt{(p_e'c)^2+m_e^2c^4}
    \end{align}
    Conservation of momentum:
    \begin{align}
        \label{eq:p.2}
        p_h = p_e' + p_h'
    \end{align}
    so:
    \begin{align}
        \label{eq:p.3}
        \vec{p}_e'^2 = 
        (\vec{p}_h - \vec{p}_h')^2 = p_h^2+p_h'^2-2p_h p_h'\cos{\theta}
    \end{align}
    According to Einstein's relationship:
    \begin{align}
        \label{eq:p.4}
        p_h = \frac{h}{\lambda},\, p_h'=\frac{h}{\lambda'}
    \end{align}
    Then \ref{eq:p.4} into \ref{eq:p.3}:
    \begin{align}
        \label{eq:p.5}
        p_e'^2= 
          h^2 \left(\frac{1}{\lambda^2}+\frac{1}{\lambda'^2}-
          \frac{2}{\lambda\lambda'}\cos{\theta}\right)
    \end{align}
    \ref{eq:p.5} into \ref{eq:p.1} $\Rightarrow$
    \begin{align}
        \label{eq:p.6}
        h(v-v') =
        \sqrt{h^2\left(
            \frac{1}{\lambda^2}+\frac{1}{\lambda'^2}
            -\frac{2\cos{\theta}}{\lambda\lambda'}\right) c^2
            +E_0^2}
        -E_0^2
    \end{align}
    where we write $E_0\equiv m_e c^2$. Squaring both sides (notice that
    $\lambda=\frac{c}{v}$):
    \begin{align}
        \label{eq:p.7}
        h^2(v-v')^2+E_0^2 +2h(v-v')E_0 = 
        h^2\left(v^2+v'^2-2\cos{\theta}vv'\right)+E_0^2
    \end{align}
    After simplification:
    \begin{align}
        \label{eq:p.8}
        (v-v')E_0 = h(1-\cos{\theta})vv'
    \end{align}
    Back to $\lambda=\frac{c}{v}$:
    \begin{align}
        \label{eq:p.9}
        \frac{\frac{1}{\lambda}-\frac{1}{\lambda'}}
            {\frac{1}{\lambda\lambda'}c}
            =\frac{h}{E_0} (1-\cos{\theta})
    \end{align}
    Or:
    \begin{align}
        \label{eq:p.10}
        \lambda'-\lambda = \Delta\lambda = \frac{h}{m_e c}(1-\cos{\theta})
    \end{align}
\end{proof}

\end{document}
