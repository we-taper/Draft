% The entire content of this work (including the source code
% for TeX files and the generated PDF documents) by 
% Hongxiang Chen (nicknamed we.taper, or just Taper) is
% licensed under a 
% Creative Commons Attribution-NonCommercial-ShareAlike 4.0 
% International License (Link to the complete license text:
% http://creativecommons.org/licenses/by-nc-sa/4.0/).
\documentclass{article}

\usepackage{float}  % For H in figures
\usepackage{amsmath} % For math
\usepackage{amssymb}
\usepackage{bbm} % for numbers within mathbb
\usepackage{mathrsfs} % For \mathscr{ABC}
% Followings are for the special character: differential "d".
\newcommand*\diff{\mathop{}\!\mathrm{d}}
\newcommand*\Diff[1]{\mathop{}\!\mathrm{d^#1}}
\numberwithin{equation}{subsection} % have the enumeration go to the subsection level.
                                    % See:https://en.wikibooks.org/wiki/LaTeX/Advanced_Mathematics
\usepackage{graphicx}   % need for figures
\usepackage{cite} % need for bibligraphy.
\usepackage[unicode]{hyperref}  % make every cite a link
\usepackage{CJKutf8} % For Chinese characters
\usepackage{fancyref} % For easy adding figure,equation etc in reference. Use \fref or \Fref instead of \ref
\usepackage{braket} %http://tex.stackexchange.com/questions/214728/braket-notation-in-latex

% Following is for theorems etc environments
% http://tex.stackexchange.com/questions/45817/theorem-definition-lemma-problem-numbering && https://en.wikibooks.org/wiki/LaTeX/Theorems
\usepackage{amsthm}
\newtheorem{defi}{Definition}[section]
\newtheorem{thm}{Theorem}[section]
\newtheorem{lemma}{Lemma}[section]
\newtheorem{remark}{Remark}[section]
\newtheorem{prop}{Proposition}[section]
\newtheorem{coro}{Corollary}[section]
\theoremstyle{definition}
\newtheorem{ex}{Example}[section]

% A list of nomenclatures.
\usepackage{nomencl}
\makenomenclature

\title{Notes of Advanced Quantum Mechanics}
\date{\today}
\author{Taper}


\begin{document}


\maketitle
\abstract{
Notes of the course Advanced Quantum Mechanics, told by professor
MH. Wang.
}
\tableofcontents

Evaluation
\begin{enumerate}
    \item Homework 20\%

        5\% bonus for solution based on \LaTeX, other bonus available.
    \item Mid-term 30\%
    \item Final 50 \%
\end{enumerate}

Textbook: Griffiths or Sakurai. As a reference, this course will cover
chapter 1, 2, 4, and 7 of Sakurai's book. And if time permits, chapter 8
about the Relativistic Quantum Mechanics in chapter 8. Since the average
level of the course is not high, He seems to hold negative view towards
this goal.

First two lessons are for the layman, which are not included in
this note.

\section{Anchor}
\label{sec:Anchor}

\begin{thebibliography}{1}
    %\bibitem{book} 
\end{thebibliography}
\printnomenclature
\section{License}
The entire content of this work (including the source code
for TeX files and the generated PDF documents) by 
Hongxiang Chen (nicknamed we.taper, or just Taper) is
licensed under a 
\href{http://creativecommons.org/licenses/by-nc-sa/4.0/}{Creative 
Commons Attribution-NonCommercial-ShareAlike 4.0 International 
License}. Permissions beyond the scope of this 
license may be available at \url{mailto:we.taper[at]gmail[dot]com}.
\end{document}
