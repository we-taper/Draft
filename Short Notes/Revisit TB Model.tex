% The entire content of this work (including the source code
% for TeX files and the generated PDF documents) by 
% Hongxiang Chen (nicknamed we.taper, or just Taper) is
% licensed under a 
% Creative Commons Attribution-NonCommercial-ShareAlike 4.0 
% International License (Link to the complete license text:
% http://creativecommons.org/licenses/by-nc-sa/4.0/).
\documentclass{article}

% My own physics package
% The following line load the package xparse with additional option to
% prevent the annoying warnings, which are caused by the package
% "physics" loaded in package "physicist-taper".
\usepackage[log-declarations=false]{xparse}
\usepackage{physicist-taper}
\makenomenclature % For an index of symbols.
\title{A Short Revisit to Tight-binding Model}
\date{\today}
\author{Taper}


\begin{document}


\maketitle
\tableofcontents

\textbf{Note}: After reading chapter 7.9 of \cite{Parker2009}, I think
that book provides a much clearer understanding of what tight-binding
model is, and therefore I recommend a first reading of that book,
before go into \cite{Ashcroft1976}.


To me, Tight-binding model is almost synonym of the following
2nd-quantized Hamiltonian:
\begin{equation}
    H = \sum_{i,\sigma} c^\dagger_{i,\sigma} c_{i,\sigma}
    - t \sum_{\braket{i,j},\sigma} 
    \left(c^\dagger_{i,\sigma} c_{j,\sigma} + \text{h.c.}\right)
\end{equation}

But actually there is much more that just this. I here give a short
note about some key points in chapter 10, The Tight-Binding Method, in
\cite{Ashcroft1976}.

The aim of Tight-binding model, is to deal 

\begin{myquote} \enquote{
with the case in which the overlap of atomic wave functions is enough
to require corrections to the picture of isolated atoms, but not so
much as to render the atomic description completely irrelevant
} \end{myquote}

For this purpose, we start with a set of states:
\begin{equation}
    \psi_{n}(\vb{r})
\end{equation}
that satisfy two conditions:
% \footnote{
% The section General Formulation in chapter 10 of \cite{Ashcroft1976}
% provides an argument for the first condition. It actually starts with
% assuming $\phi_{n}(\vb{r})$ are eigenstates of atomic Hamiltonian as
% an introduction to the Tight-binding mode.}
\begin{enumerate}
    \item They model a wave function localized at each lattice points.
        Here, they are taken to be the bound levels of atomic
        Hamiltonian which are well localized:
        \begin{equation}
            H_{\text{at}} \psi_n(\vb{r})  = E_n \psi_n(\vb{r})
        \end{equation}
    \item They are orthonormal with respect to each other (which is
        quite reasonable a requirement).
\end{enumerate}

With this states, we test if the following wave function is a good
wavefunction of the system:
\begin{equation}
    \psi_{k} = \sum_R e^{ikR} \phi(r-R)
\end{equation}
where $k$ ranges through the $N$ values in the first Brillouin zone
consistent with the Born-von Karman periodic boundary condition. And
\begin{equation}
    \phi(r) = \sum_n b_n \psi_n(\vb{r})
\end{equation}
Exactly how many $n$ should be summed in the above expression will
depend on the situation (explained below).

It is easy to see that $\psi_{nk}$ satisfy the Bloch condition. So we
could assume that:
\begin{equation}
    H \psi(\vb{r}) = (H_\text{at}+\Delta U(\vb{r}))\psi(\vb{r})
    = \varepsilon(\vb{r})\psi(\vb{r})
\end{equation}

Using the orthonormal condition and some techniques, we have (see
pp.179 to 180 of \cite{Ashcroft1976}):
\begin{align}
    \label{eq:tb-model-eq}
    (\varepsilon(k) - E_m) b_m = & 
    -(\varepsilon(k)-E_m)\sum_n\left(
        \sum_{R\neq 0}\int\psi^*_m(r)\psi_n(r-R)e^{ikR}\dd{r}\right)b_n
    \nonumber\\
    &+ \sum_n \left(\int\psi^*_m(r)\Delta U(r)\psi_n(r)\dd{r}\right) b_n
    \nonumber\\
    &+ \sum_n\left(\sum_{R\neq 0}
        \int\psi^*_m(r)\Delta U(r)\psi_n(r-R)e^{ikR}\dd{r}\right)b_n
\end{align}

Now we argue that the right hand side of this equation should be small
because there are small overlap between those localized $\psi_n$.
Therefore, the left hand side should also be small. Therefore:
\begin{equation}
    \varepsilon(k) \approx E_0,\, b_m\approx 0\text{ unless } E_m\approx E_0
\end{equation}
where $E_0$ is energy of one atomic level. However, the above is only
an approximate analysis. To get more details, we would let $n$ only
summed over those levels with energies either degenerate with or very
close to $E_0$, our anticipated atomic energy level. This would make
\ref{eq:tb-model-eq} for different $m$s into a matrix equation. (see
more at p.181 of \cite{Ashcroft1976}). The solution to this equation
gives us coefficients $b_m$, the wavefunction $\psi_k$, and the energy
$\varepsilon(k)$.

\paragraph{Wannier function} The Tight-binding model is related to
Wannier function, since any Bloch function $\psi_{nk}(r)$ can be
expended
\begin{equation}
    \psi_{nk} = \sum_r f_n(r-R) e^{iRk}
\end{equation}
where
\begin{equation}
    f_n(R,r) = \frac{1}{v_0} \int\dd{k}e^{-iRk}\psi_{nk}(r)
\end{equation}
and $v_0$ is the volume in $k$-space of the first Brillouin zone, and
the integral is taken over the 1st BZ. $f_n$ are the so called Wannier
functions and they could be well localized (see p.188 of
\cite{Ashcroft1976}) and is orthonormal (see problem 3 of chapter 10,
\cite{Ashcroft1976}). Therefore they may be substituted as the atomic
eigenstates $\psi_n(r)$ in the above steps.  \todo{How to proceed
them?} % we don't know the energy, the Bloch wavefunctions. How to do it them?

\bibliography{../Library}{}
\bibliographystyle{alphaurl}

% \begin{thebibliography}{1}
% 	\bibitem{book} 
% \end{thebibliography}
\printnomenclature
\section{License}
The entire content of this work (including the source code
for TeX files and the generated PDF documents) by 
Hongxiang Chen (nicknamed we.taper, or just Taper) is
licensed under a 
\href{http://creativecommons.org/licenses/by-nc-sa/4.0/}{Creative 
Commons Attribution-NonCommercial-ShareAlike 4.0 International 
License}. Permissions beyond the scope of this 
license may be available at 
\href{http://www.google.com/recaptcha/mailhide/d?k=015LguzBJigi0rpyuJRqLoig==\&c=p1c-M-mm7ZcjUCkTuZZa9eEPHRVk6paN0694iazlQy8=}
{[My Email Address(Click)]}.
\end{document}
