% The entire content of this work (including the source code
% for TeX files and the generated PDF documents) by 
% Hongxiang Chen (nicknamed we.taper, or just Taper) is
% licensed under a 
% Creative Commons Attribution-NonCommercial-ShareAlike 4.0 
% International License (Link to the complete license text:
% http://creativecommons.org/licenses/by-nc-sa/4.0/).
\documentclass{article}

% My own physics package
% The following line load the package xparse with additional option to
% prevent the annoying warnings, which are caused by the package
% "physics" loaded in package "physicist-taper".
\usepackage[log-declarations=false]{xparse}
\usepackage{physicist-taper}
\makenomenclature % For an index of symbols.
\title{Shen's View on Topological Quantum Numbers}
\date{\today}
\author{Taper}


\begin{document}


\maketitle
\abstract{
    A short summary of chapter 4, Topological Invariants, of \cite{Shen2012}.
}
\tableofcontents

The local gauge freedom of phase gives us Berry phase, and the Hall conductance
$\sigma_{xy}$ can be related to this Berry phase as:
\todo{Which bands should be included in calculating the Berry phase.}
\begin{equation}
    \sigma_{xy} = \int_{\mathrm{BZ}} \frac{\dd{\vb k}}{(2\pi)^2}
    \Omega_{k_x,k_y}
\end{equation}
where $\Omega_{k_x,k_y}$ is the Berry curvature. The Berry curvature for each
band is defined as
\begin{equation}
    \Omega^n(\vb k) = \grad_{\vb k} \times
        \braket{ u_n(\vb k) | i \grad_{\vb{k}} | u_n(\vb{k})}
\end{equation}

This Berry phase can also be related to the electric polarization (more exactly,
the $\vb P$ in $D=\epsilon_0 \vb{E} + \vb{P}$), in a solid in which there is no
electric field ($\vb{E}=0$). The connection is:
\begin{equation}
    \Delta P_{\alpha} = e \sum_n \int_0^T\dd{t} 
    \int_{\mathrm{BZ}} \frac{\dd{\vb q}}{(2\pi)^d}\Omega^n_{q_\alpha,t}
\end{equation}
(where $d$ should be the dimension of the solid, though this is not mentioned
in the book.)

The quantization of Hall conductance can be viewed from two perspectives. On one
hand, the Berry phase is proportional to the first Chern number, which is proved
mathematically to be quantized. On the other hand, there is the Laughlin's
argument. The Laughlin's argument does two things. First, it relates the change
of flux $\Delta \phi$ to the change of charge $\Delta Q$ by simple
electrodynamics:
\begin{equation}
    \Delta Q = \sigma_{xy} \Delta \phi
\end{equation}
Second, it argues that the threading of one flux quantum $\hbar/e$ does not change the
\todo{Why do we thread this flux of quantum?}
Hamiltonian, but instead pushes integers numbers of charge transport (just like
a Thouless charge pump):
\begin{equation}
    \Delta Q = n e
\end{equation}
Therefore, $ne = \sigma_{xy} \hbar/e$ leads to $\delta_{xy} = n e^2/\hbar$.

However the $\Z_2$ invariant is a complicated stuff which I do not understand.

Also, the generalization of topological insulator from two dimension to three
dimension is done in section 4.9, which I do not understand. But he defines
clearly the concept of strong and weak topological insulator, and mentions that
a weak topological insulator is topologically equivalent to a two-dimensional
topological insulator, and is not robust against disorder.

\bibliography{../Library}{}
\bibliographystyle{alphaurl}

% \begin{thebibliography}{1}
% 	\bibitem{book} 
% \end{thebibliography}
\printnomenclature
\section{License}
The entire content of this work (including the source code
for TeX files and the generated PDF documents) by 
Hongxiang Chen (nicknamed we.taper, or just Taper) is
licensed under a 
\href{http://creativecommons.org/licenses/by-nc-sa/4.0/}{Creative 
Commons Attribution-NonCommercial-ShareAlike 4.0 International 
License}. Permissions beyond the scope of this 
license may be available at 
\href{http://www.google.com/recaptcha/mailhide/d?k=015LguzBJigi0rpyuJRqLoig==\&c=p1c-M-mm7ZcjUCkTuZZa9eEPHRVk6paN0694iazlQy8=}
{[My Email Address(Click)]}.
\end{document}
