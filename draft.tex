% The entire content of this work (including the source code
% for TeX files and the generated PDF documents) by 
% Hongxiang Chen (nicknamed we.taper, or just Taper) is
% licensed under a 
% Creative Commons Attribution-NonCommercial-ShareAlike 4.0 
% International License (Link to the complete license text:
% http://creativecommons.org/licenses/by-nc-sa/4.0/).
\documentclass{article}

\usepackage{float}  % For H in figures
\usepackage{amsmath, amssymb} % For math
\usepackage{mathtools} % dcases*, see https://en.wikibooks.org/wiki/LaTeX/Advanced_Mathematics#The_cases_environment
\numberwithin{equation}{subsection} % have the enumeration go to the subsection level.
									% See:https://en.wikibooks.org/wiki/LaTeX/Advanced_Mathematics
\usepackage{graphicx}   % need for figures
\usepackage{cite} % For bibligraphy
\usepackage{fancyref} % For lazy reference \fref
\usepackage{hyperref} % For hyperlink everything.
\usepackage{CJKutf8} % For Chinese characters
%\usepackage{ dsfont } % For double struck fonts
\usepackage{braket} 
\usepackage[T1]{fontenc}
\usepackage{listings}

\usepackage{amsthm}
\newtheorem{defi}{Definition}[section]
\newtheorem{thm}{Theorem}[section]
\newtheorem{lemma}{Lemma}[section]
\newtheorem{remark}{Remark}[section]
\newtheorem{prop}{Proposition}[section]
\newtheorem{coro}{Corollary}[section]
\theoremstyle{definition}
\newtheorem{ex}{Example}[section]


\title{Topological Insulators and Topological Superconductors}
\date{\today}
\author{we.taper}
\begin{document}
\maketitle
\abstract{
    This is a draft.
}
\tableofcontents

\section{Lectures on the Frontiers of Physics}
\textbf{Given by professor of physics in SUSTC}
    \subsection{By JQ. He.}
    \textbf{Thermal electrics}

    \subsection{By Lang. Chen}
    \label{sec:c.l}
    Grow thin films.
    \begin{itemize}
	\item Rheed-Assited PLD/MBE. (Ray as an exmination).
	\item 
	orbital contral of electrons -> 
	orbitronics -> Control of Spin orbital coupling.
	\item
	Multiferroics -> multiple order parameters, and the interaction
	between them. E.g. $BiFeO_3$.
	\item
	Ferrotorodicity: Spontaneous Toroidal Momennt. Time and spacial
	symmetries simultaneous broken.
	\item
	What is a iridates $Ir_2(X)O_4$, (e.g. $Sr_2RuO_4$) 
	exactly in theoretical physics?
	\item
	$H_2S$: 200K superconductor?
	\item
	The Double Exchange effect of oxygem -> Half-metal, phase transition.
    \end{itemize}

    \subsection{By Alan}
    \textbf{Photocatalysis}: $TiO_2$. Hongkong has $TiO_2$ spurred on
    the keys.

    \subsection{By Li, Huang}
    \begin{itemize}
            \item Computational Physics
            \item Surface Dynamics
            \item Structural factor from 2D to 3D.
            \item Finding Order Amid the Chaos. amorphous -> spatially
                    resolved distributed function.
            \item ?: What is genetically algorithm.
    \end{itemize}
    \textbf{Computational and theoretical studies of Surface dynamics}
    
    \begin{itemize}
            \item Surface atoms is immersed in a very different environment
                    compared with the bulk atoms.
            \item First-principle calculations
            \item DFT + LDA -> Conser equation
            \item Plane wave basis + Ultrasoft pseudopotentials to solve the
                    Conser equation
            \item Continumm method ?
    \end{itemize}

    \subsection{By Junfen, Liu}
    \begin{itemize}
            \item electronic transport in mesoscopic systems:
            \item Spintronics
            \item Graphene eletronics
            \item Superconductors etc.
    \end{itemize}
    \textbf{Quantum wire conductiing}
    The conduction channel in quantum wire is quantized, with discrete value
    of conductance.
    \begin{itemize}
            \item $\lambda_F$ Fermi wavelength
            \item $L_m$ Momuntum relaxation length <-- impurities.
            \item $L_\phi$ Phase relaxation length <-- memory of phase,
                    related to energy $\omega = E/\hbar$.
            \item $L$ Sample length
    \end{itemize}
    \begin{itemize}
            \item Ballistic transport: $L << L_m$ No scattering.
            \item Diffusive $L>L_m$, scattering, reduced transmission.
            \item Localization $L_m<<L<<L_\phi$ -> Prof. Haizhou Lu.
            \item Classical. (Omitted)
    \end{itemize}
    \textbf{Conductance}
    No back-scattering $$ G = \frac{I}{V} = \frac{2e^2}{h}$$
    Landauer formula $G = \frac{2e^2}{h}\cdot T$, $T$ is some coefficient
    accounting for the back scattering, perhaps the transmission
    probability.
    In reality, $G=\sum_{\text{Different channels}}G_i$, 

    We can turn the $G$ into resisivity: 
    $$\text{Resistance}= \frac{h}{2e^2} + \frac{h}{2*e^2}\frac{R}{T}$$
    $R+T=1$

    \paragraph{Resonate Transmission} (Omitted)

    \paragraph{Spintronics} Use the extra freedom of Spin. 
    
    Spin field eletroncis: Datta and Das, Appl. Phys. Lett. 56, 665(1990)

    GMR: 2007 Nobel prize in Physics.
    
    Hall Effect (Omitted)
    Spin Hall Effect: S. murakami, et.al. Science 301 1348(2004);
    

    J. Sinova et.al. Phys.Rev.Lett. 92, 126603 (2004). (Omitted)
    
    \paragraph{Graphene} Carrier -> Relativistic Dirac fermions.
    \textbf{Klein Paradox} 

    \paragraph{Josephson Junction} A phase difference could conduct
    electricity in Superconductors.
    


    \subsection{By Haizhou, Lu}
    \paragraph{Quantum Anomalous Hall Effect}
    Requires strong magnetic field: $\approx 10$ Tesla.

    Anomalous Hall Effect: Without magnetic field. $R_H = R_0 B+ R_A M$
    where $M$ is the magnetic susceptibility.
    Two-factors: SO coupling. Spin-dependent Hall Effect.
    
    An excellent illustrations is found in \cite{Arxiv-Xiao}:
    \begin{figure}[H]
        \centering
        \includegraphics[width=0.8\linewidth]{pics/1}
        \caption{Illustration}
        \label{fig:Xiao Illustration}
    \end{figure}

    \subsection{By Kedong Wang}
    Tunneling current $I\propto V e^{-2kz}$, where 
    $k = \frac{\sqrt{2m\phi}}{\hbar}$, $\phi$ is the Work function.
    $I$ is very sensitive to the distance $z$.
    
    \paragraph{Work function} $\phi$ characterize the obstruction
    that prevents electron from escaping the sample.

\section{Problems in Bernevig's Topological ...}

    \subsection{Chapter 2}
    \paragraph{Non-Abelian Berry Transport}
    Derive Berry curvature to the adiabatic transport of a degenerate
    multiplet of states separated by a gap from the excited states.
    (Cautious about rotation within degenerate states).

    Answer: $\gamma_{mn}(t) = i \int_0^t \braket{m(R(t'))|\frac{d}{dt'}|n(R(t'))}dt'$

    Approach:
    Assuming that the those degenerate states are labeled by $1\cdots N$. 
    Thus we have naturally:
	\begin{align}
            H\phi &=i\hbar\frac{\partial}{\partial t}\phi\text{,  }
        \phi = \sum_n A_n \psi_n
    \end{align}
    
    Then we have:
    \begin{align}
    H \sum_n A_n \psi_n &=i\hbar\frac{\partial }{\partial t}
	  	\sum_n A_n(t) \psi_n(R(t)) \nonumber\\
	\sum_n E A_n \psi_n&=i\hbar\sum_n
		\left(
		\frac{\partial A_n(t)}{\partial t}\psi_n(R(t))+
		A_n(t)\frac{\partial \psi_n(R(t))}{\partial t}
		\right) \nonumber\\
	E A_m &=i\hbar
		\frac{\partial A_m(t)}{\partial t}
		+\sum_n A_n(t)\braket{m|\frac{\partial }{\partial t}|n}
    \end{align}
    Put in another form:
    $$
    \sum_n \left(\delta^n_m E
        - \braket{m| \frac{\partial}{\partial t}|n} \right) A_n
        =
    i\hbar \frac{\partial A_m(t)}{\partial t}
    $$
    In matrix form:
    \begin{align}
            (E - P) \mathbf{A} = i\hbar \dot{\mathbf{A}}
    \end{align}
    where:
    \begin{align}
        E &= \left(
        \begin{array}{ccc}
         \cdots   &     &   \\
              & E &   \\
             &    & \text{...} \\
        \end{array}
        \right) \\
        P &= (P^m_n) = \left(
        \braket{m| \frac{\partial}{\partial t}|n}\right)\\
        A &= \left(\begin{array}{c}
            A_1(t) \\
            A_2(t) \\
            \cdots \\
        \end{array}\right)
    \end{align}
    
    Note that $\braket{n| \frac{\partial}{\partial t}|m}^* \neq
    \braket{m| \frac{\partial}{\partial t} |n}$, thus $P$ may not be
    Hermitian.Ergo $E-P$ is Hermitian. So it is diagonalizable.

    Notice that
    \begin{align}
            0= \frac{\partial}{\partial t} \braket{m|n} &=
        \braket{\frac{\partial}{\partial t}m|n} +
        \braket{m|\frac{\partial}{\partial t}n} &&(\text{any }m,n)
    \end{align}
    temporary mathematica code:

% In[8]:=DSolve[{y1'[x]+y2'[x]==x,y1'[x]-y2'[x]==0},
%         {y1[x], y2[x]}, x]
% In[9]:= DSolve[{e1*A1[t]==A1'[t]+A1[t]*m1+A2[t]*m2, 
% e1*A2[t]==A2'[t]+A1[t]*m1+A2[t]*m2},{A1[t],A2[t]},t]
% In[10]:=DSolve[{e1*A1[t]==A1'[t]+A1[t]*m1},{A1[t]},t]
    

\section{Miscellnaneous Notes}
    \subsection{Super conductor}
    Mean-field approach to deal with a four operator diagonalization.
    
    Suppose we have: $D^*C^* CD$, then let $\delta = CD - \braket{CD} =
    CD - avg$. Then if we assume $\braket{CD}\neq 0$, and $\delta \approx 0$. Then we have:
    
    	$$ \delta^2 \approx 0 $$
    i.e.:
    
    \begin{align}
    	( (CD)^* - avg ) ( CD - avg ) = 0\\
    	D^*C^*CD = avg*(CD+D^*C^*) - avg^2
    \end{align}
    Hence a four operator is reduced into a few of two operators.
    Such method could be naturally extended to treat the operator
    $\sum_{i,j} D^*_i C^*_i C_j D_j$.
    
    A copper pair has the energy of:
    $$\Delta = \braket{C_{k \uparrow}C_{-k \downarrow} }$$
    
    To resist the flow of current carried by Copper Pair, is equivlant to destroying a pair of Copper Pair:
    $$ \braket{C_{k \uparrow}C_{-k \downarrow} } \longrightarrow C_{k \uparrow}C_{-k \downarrow}$$
    
    This will require an additional enegy of $2\Delta$.
    
    The exact meaning of "equivalent to" is as follows:
    \begin{align}
        & \text{break a copper pair} \longrightarrow 
        \text{scatter two electrons consecutively} 
        \nonumber\\ & \longrightarrow 
        \text{create two electron-hole mixed type quasi-particle} \longrightarrow 2\Delta \nonumber
    \end{align}

    \subsection{Why 0/0 is undefined?}
    If we suppose
    $$ \frac{0}{0}= \triangle $$
    Consider the following derivation:
    \begin{align}
	    \frac{0}{0} \cdot 1 &= \triangle \cdot 1 = \triangle \\
	    0 \cdot \frac{1}{0} &= \triangle\\
	    \Rightarrow \triangle &= 0
    \end{align}
    This is already bad enough. And we are forced to define $\frac{1}{0}$.
    Let $\frac{1}{0} = \square$, which literally means $1=0\cdot \square = 0$.
    This is disastrous.

    Alternatively, we could let
    \begin{enumerate}
	    \item Let $\frac{1}{0}$ be undefined.
	    \item Or let $\frac{1}{0} =
		    \infty$.
	    \item Or, let $\frac{a}{b}\cdot c= a\cdot \frac{c}{b}$ be not 
		    true when $b=0$.
    \end{enumerate}
    The third idea is disastrous for algebraic manipulation. \footnote{
    Or more speicifically, it is a disaster for field theory.
    }
    The first idea is not good. Since defining $\frac{1}{0}=\infty$ turns
    out to be very useful in both mathematics and physics. Actully, in
    physics it is common practice to set $\frac{a}{0}=\pm\infty$ for
    any nonzero number $a$, where the sign of $\infty$ is determined by 
    the sign of $a$.
    The second idea is okey. But then we are faced with a serious problem.
    We have to define $\triangle \equiv 0 \cdot \infty$

    $\triangle \cdot 2 = \triangle$, What will be of $\triangle + 1$?

    \subsection{Preface of BSCS}
        \label{sec:Preface_of_Bosonization_and_Strongly 
        Correlated_Systems}
    BSCS: see \cite{BSCS}.
    Parallism between theories in condensed matter physics and those in
    particle physics.
    \begin{itemize}
            \item Anderson-Higgs Phenomenon (Paritcle), Meissner effect
                    (C.M.P.)
            \item 'inflation' in Cosmology, first order phase transition
            \item 'cosmic strings', magnetic field vortex lines in type
                    II superconductors
            \item Hadron-meson interaction, Ginzburg-Landau theory of
                    superfluid $He^3$.
    \end{itemize}
    Same ideas on different space-time scales, different hierachical
    'layers'.
    Strong parallism: \textbf{strongly correlated low dimensional system}

    E.g.:

    The problem of formation and structure of heavy particles - hadrons and mesons. The corresponding fine structure constant $\alpha_G\approx 1$.

    Approaches:
    \begin{enumerate}
            \item Exact solutions
            \item Reformulate complicated interacting models in such a way
                    that they become weekly interacting. -> Bosonization.
                    
                    Spin $1/2$ anisotropic Hisenberg chain $\approx$
                    Model of interacting

                    fermions.
                    (Jordan and Wigner, 1928)
    \end{enumerate}
    Bosonization: transformation from fermions to a scalar massless bosonic
    field.


    \subsection{System of Differential Equations}
    This is a small note of \cite{DETA}.

    pp. 266.

    \begin{defi}
        $\mathbf{x(t)}$ is a vector whose elements are $x_i(t)$.
        $ \frac{d}{d t}$ acts on vector $\mathbf{x}$ element-wise.
        $\dot{\mathbf{x}}$ is abbrevation for $\frac{d}{d t}\mathbf{x}$
    \end{defi}
    
    pp. 291.

    \begin{thm}[Existence-uniqueness theorem]
        There exists one, and only one, solution of the initial-value
        problem

        \begin{align}
            \dot{\mathbf{x}}=\mathbf{A}\mathbf{x}\text{, }&
                \mathbf{x}(t_0) = \mathbf{x}^0 = 
                \left(
                \begin{array}{c}
		            x^0_1\\
                    x^0_2\\
                    \cdots
                \end{array} 
                    \right)
        \end{align}
        
        Moreover, this solution exists for $-\infty\langle t\langle \infty$.
    \end{thm}
    \begin{remark}
        By this, any non-trivial solution $\mathbf{x}(t)\neq 0$ at any
        time $t$. Also notice that the elements of $\mathbf{A}$ are just
        numbers.
    \end{remark}
    
    \begin{thm}
        The dimension of the space $\mathbf{V}$ of all solutions of the
        homogeneous linear system of differential equations:
        \begin{align}
            \frac{d\mathbf{x}}{dt}=\mathbf{Ax}
        \end{align}
        is $n$, i.e. the dimension of vector $\mathbf{x}$.
    \end{thm} 
    \subsection{ODE by Arnold}
    sec. 14
    \begin{defi}
        \begin{align}
            \label{eq:e^A}
            e^A &= I + A + \frac{A^2}{2!} + \frac{A^3}{3!}\\
            \text{or}& \nonumber \\
            e^A &= \lim_{n\to \infty}(I+\frac{A}{n})^n
        \end{align}
        where $I$ is the identity matrix.
    \end{defi}
    Equivalance of the two definition will be addressed in the Theorem on
    pp. 165.

    Important theorems:
    \begin{thm}[pp. 158]
        The series $e^A$ converges for any $A$ uniformly on each set
        $X=\{A:||A||\leq a\}$, $a\in \mathbb{R}$.
    \end{thm}
    \begin{thm}[pp. 160]
        $$e^{At} = H^t$$
        where $H^t$ is the translation operator which sends every polynomial
        $p(x)$ into $p(x+t)$.
    \end{thm}
    \begin{thm}[pp. 163]
        $$\frac{d}{dt} e^{tA} = Ae^{tA}$$
    \end{thm}
    \begin{thm}[Fundamental Theorem of the Theory of Linear Equations with
        Constant Coefficients]
        The solution of:
        \begin{align}
            \label{eq:fund_thm_of_linear_eqs_const_coef}
            \dot{\mathbf{x}} = A\mathbf{x}
        \end{align}
        with initial condition $\phi(0) = \mathbf{x}_0$ is
        \begin{align}
            \mathbf{\phi}(t) = e^{tA}\mathbf{x}_0
        \end{align}
    \end{thm}
    
    Practically solution to
    $$ \dot{\mathbf{x}} = A\mathbf{x}$$
    (pp. 173, Sec 17)
    (Assuming $A$ is diagonalizable.)
    \begin{itemize}
        \item Find the eigenvectors $\xi_1,\cdots ,\xi_n$ and eigenvalues
            $\lambda_1,\cdots ,\lambda_n$. Use them as basis.
        \item Expand the initial condition in the new basis.
            \begin{align}
                \mathbf{x}_0=\sum_{k=1}^{n} C_k\xi_k
            \end{align}
        \item Then $\phi(t) = \sum_{k=1}^n C_k e^{\lambda_k t}\xi_k$
    \end{itemize}
    \subsection{Appearance of Gauge Structure in Simple Dynamical Systems}
    
    \begin{align}
        0=(\eta_b,\dot{\eta_a}) = (\eta_b,\dot{U}_{ac}\psi) +
            (\eta_b,U_{ac}\dot{\psi}_c
    \end{align}
\section{Anchor}
% This is just an anchor for seperating thebibliography from above contents.
\begin{thebibliography}{1}
	\bibitem{Sakurai} Sakurai, J. J. Modern Quantum Mechanics, Addison Wesley.
	\bibitem{BSCS} Bosonization and Strongly Correlated Systems. Cambridge.

        \href{http://www.cambridge.org/us/academic/subjects/physics/condensed-matter-physics-nanoscience-and-mesoscopic-physics/bosonization-and-strongly-correlated-systems}{Cambridge Press Link}

    \bibitem{DETA} Martin Braun. Differential Equations and Their
    Applications. 4ed. Springer.
    \bibitem{Arxiv-Xiao} \url{http://arxiv.org/abs/1508.07106v1}
\end{thebibliography}
\section{License}
The entire content of this work (including the source code
for TeX files and the generated PDF documents) by 
Hongxiang Chen (nicknamed we.taper, or just Taper) is
licensed under a 
\href{http://creativecommons.org/licenses/by-nc-sa/4.0/}{Creative 
Commons Attribution-NonCommercial-ShareAlike 4.0 International 
License}. Permissions beyond the scope of this 
license may be available at \url{mailto:we.taper[at]gmail[dot]com}.

\end{document}
