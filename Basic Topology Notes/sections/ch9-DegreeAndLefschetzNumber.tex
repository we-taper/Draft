
Professor Li Qin skipped many parts before. In this chapter, he also
mentioned only important theorems in the first two sections.

The language in this note will be quite loose. Proofs will be omitted,
so will some theorems, as this was how my teacher taught me. I am not
sure if there are more valuable techniques in these sections. Hence
this can be only regarded as a guideline/introduction to the book.
% todo: ask why he mentioned only this few parts

\subsection{Section 9.1 Maps of spheres}
\label{sec:Maps-of-spheres}

In this section, we restrict our attention only on spheres $S^n$. The
goal is investigate the homotopy classes of continuous functions
between spheres. We will see, in the end, that we can obtain some
information by examining the how these continuous functions acting on
a simple triangulation of sphere.

We first define the degree of a continuous function $f:S^n\to S^n$.
Since $H_n(S^n)=\Z$, we define \nomen{$\operatorname{deg}f$} to be the number
$f_*(1)$, in $f_*:H_n(S^n)\to H_n(S^n)$. Or, equivalently, the number
$\lambda$ such that $f([z])=\lambda[z]$, where $[z]$ is the generator
of $H_n(S^n)=\Z$.

Then we discover several facts:
\begin{fact}
    The degree does not depend on the choice of triangulation of
    sphere.
\end{fact}
\begin{fact}
    Homotopic continuous functions have the same degree.
\end{fact}
\begin{fact}
    $\operatorname{deg}(f\circ g)=\operatorname{deg}f\times
    \operatorname{deg}g$
\end{fact}
\begin{fact}
    By above fact, if $f$ is a homeomorphism,
    $\operatorname{deg}f=\pm 1$.
\end{fact}
\begin{fact}
    Obvious, the degree of identity map of $S^n$ is $1$, that of
    constant map is $0$. Hence the identity map is not homotopic to
    the constant map.
\end{fact}

The above facts can be found in page 195 of \cite{book}.

Since degree is triangulation independent, we will now stick to a very
convenient triangulation constructed in page 196 of \cite{book}. It is
the most straightforward one, and is denoted by \nomen{$\Sigma$}
(remember, we have constructed similar things in chapter 8, denoted
$\Sigma^n$). The vertices are $v_i=(0,\cdots,0,1,0,\cdots,0)$, for
$i=1,2,\cdots,n+1$, and the only $1$ is on the $i$-th coordinate.
Similary we have $v_{-i}=0,\cdots,0,-1,0,\cdots,0)$. Whenever
$\abs{i_1}<\abs{i_2}<\cdots<|i_s|$, with $1\leq s \leq n+1$, the
vertices $v_{i_1},v_{i_2},\cdots,v_{i_s}$ spans a simplex in $\Sigma$.
The collection of all such simplexes constitutes the simplicial
complex $\Sigma$. A graphical illustration can be found on page 196 of
\cite{book}.

The degree of a continuous function $f$ can than be easily determined
on this triangulation. But I will only mention that this is
demonstrated in theorem (9.1) on page 196 of \cite{book}.

Then we discover a sequel of facts, each depending on the previous
one:
\begin{fact}
    The antipodal map of $S^n$ has degree $(-1)^{n+1}$
\end{fact}
\begin{fact}
    A continuous function $f:S^n\to S^n$ which has no fixed points
    must have degree $(-1)^{n+1}$.
\end{fact}
\begin{fact}
    If $n$ is even, and if $f:S^n\to S^n$ is homotopic to the
    identity, then $f$ has a fixed point.
\end{fact}

Using this we can derive an interesting theorem:
\begin{thm}[Hairy ball theorem]
    $S^n$ admits a continuous nonvanishing vector field if and only if
    $n$ is odd.
\end{thm}
\begin{proof}
    Let $v$ be a vector field on $S^n$, for each $v(\vb{x})$, it can
    be regarded just as a normal vector in $\E^{n+1}$ since each
    tangent space is locally a $\R^{n+1}$.

    The continuous function:
    $$f(\vb{x}) = \frac{\vb{x}+v(\vb{x})}{||\vb{x}+v(\vb{x})||}$$
    is homotopic (hint, add a parameter $t$ in front of $v(\vb{x})$), 
    to identity map. Hence $f(\vb{x})$ admits a fix point, i.e. there
    is a point $\vb{x}$ such that $v(\vb{x})=0$.
\end{proof}

Notice that in proof above, the function $f(\vb{x})$ actually help us
regard a vector filed as a speed field, a field that moves each point
on the sphere infinitesimally.

%TODO ntoes for the next section.
