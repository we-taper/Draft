
\subsection{Homotopic maps}
\label{sec:Homotopic-maps}

By a \nomen{loop} we mena a continuous map $\alpha:I\to X$ such that
$\alpha(0)=\alpha(1)$. We can view is also as a continuous map
$\alpha:S^1 \to X$. It is said to be based at the point
$\alpha(0)$. Two loops $\alpha$ and $\beta$ with the same \nomen{base
point} can be multiplied, and their product is defined on page 87.
A visualization is here:
% cd 1 todo

But this product is not sufficient to become a group. At least, the
multiplication is not associative:
% cd 2 todo

But clearly the two result are exactly if we do care how long they
occupy on the interval $I$, if the interval $I$ is considered as a
time parameter. So we define the following homotopy relation between
loops. If we can find a family $\{f_r\}$ of maps, one for each $r\in
[0,1]$, such that $f_0=\alpha$, $f_1=\beta$, then we say that the
loops $\alpha$ and $\beta$ are homotopic. Schematically,
% cd 3 todo

This relation can be generalized to any continuous maps:
\begin{defi}[Homotopic]
\nomenclature{Homotopic}{\nomrefpage.}
    Let $f,g: X \to Y$ be continuous maps. Then $f$ is homotopic to
    $g$ if there exists a map $F:X \times I\to Y$ such that $F(x,O) =
    f(x)$ and $F(x,1) = g(x)$ for alt points $x\in X$.
\end{defi}

The map $F$ is called a \nomen{homotopy} from $f$ to $g$, and we write
\nomen{$f\simeq_F g$}. In addition, if $f$ and $g$ agree on some
$A\subset X$, we may wish to deform $f$ to $g$ without altering the
values of $f$ on $A$. In this case we ask for a homotopy $F$ from $f$
to $g$ with the additional property that 
\begin{equation}
    F(a,t)=f(a) \text{ for all $a\in A$, for all $t\in I$}
\end{equation}
when such a homotopy exists, we say the \nomen{$f$ is homotopic to $g$
relative to $A$} and write \nomen{$f\simeq_F g$ rel $A$}.
% cd 4 todo


When $f$ and $g$ are loops, then the homotopic relation for loops are
just saying that $f\simeq g$ rel $\{0,1\}$.

% fig 5.1 p 88 todo

\begin{ex}
    The author shows on page 88 of \cite{book} that: when $C$ is a
    convex subset of a euclidean space, let $f,g:X\to C$ be continuous
    maps, then $f\simeq_F g$, where $F$ is $F(x,t)=(1-t)f(x)+tg(x)$.
    Note that if $f$ and $g$ agree on a subset $A$ of $X$, then this
    homotopy is a homotopy relative to $A$. This $F$ is called a
    \nomen{straight-line homotopy}.
\end{ex}
\begin{ex}
    Let $f,g:X\to S^n$ be continuous maps. We can take $S^n$ to be the
    unit sphere in $\mathbb{E}^{n+1}$, and think of $f,g$ as
    continuous maps into $\mathbb{E}^{n+1}$, then we may form a
    straight-line homotopy from $f$ to $g$ by:
    \begin{equation}
        F(x,t) = \frac{(1-t)f(x)+tg(x)}{||(1-t)f(x)+tg(x)||}
    \end{equation}
    But why we are  % yes why! TODO
\end{ex}
\begin{ex}
    This is example is best illustrated by pictures:
    % fig 5.2 todo
    Geometrieally, $\alpha$ winds eaeh of the segments
    $[O,\frac{1}{2}]$, $[\frac{1}{2},\frac{3}{4}]$, $[\frac{3}{4}, 1]$
    once round the eirc1e, the first two being wound in an
    anticlockwise direetion, and the third clockwise. The loop $\beta$
    simply winds the whole interval $[0,1]$ once round the circle
    anticlockwise.

    The book\cite{book} gives a homotopy $F$ between $\alpha$ and
    $\beta$ on page 89. But it is best to imagine $\alpha$ and $\beta$
    being metal coils, and this $F$ just describes the process when
    one magically strach and unfold the coil from $\alpha$ to $\beta$.

    Notice that this coil is connected head to tail, so it is
    essential that there is not pole inside the coil in order that one
    can unfold the coil from $\alpha$ to $\beta$.
\end{ex}

I think we already feel this, but the book proves it on page 90, that

\begin{lemma}
    The relation of 'homotopy' is an equivalence relation on the set
    of all maps from $X$ to $Y$.
\end{lemma}

Also
\begin{lemma}
    The relation of 'homotopy relative to a subset $A$ of $X$' is an
    equivalence relation on the set of all maps from $X$ to $Y$ which
    agree with some give map on $A$.
\end{lemma}

The book also mentions that
\begin{lemma}
    Homotopy behaves well with respect to composition of maps
\end{lemma}
which means precisely that:
\begin{itemize}
    \item If $f\simeq_F g$ rel $A$, then $hf\simeq_{hF} hg$ rel $A$.
        $$ \begin{tikzcd}[]
            A\subset X \ar[r,"f",bend left]\ar[r,"g",bend right]
                & Y \ar[r,"h"] & Z
        \end{tikzcd}$$
    \item If $g\simeq_G h$ rel $B$, then $gf\simeq_F hf$ rel $f^{-1}B$
        via the homotopy $F(x,t)=G(f(x),t)$.
        $$ \begin{tikzcd}[]
            X\ar[r,"f"] & Y \ar[r,"g",bend left]\ar[r,"h",bend right]
             & Z \\
            f^{-1}B\ar[u,phantom,"\subset"
            {anchor=south,rotate=90,left=0,near end}]
                &
            B\ar[u,phantom,"\subset" {anchor=south,rotate=90,left=0,near
            end}] & \, 
        \end{tikzcd}$$
\end{itemize}




From this on, I will follow the lecture by professor Li Qin.

\subsection{Review}
\label{sec:Review}
From last lecture, we have shown:
\begin{ex}
    For $n\geq 2$,
    $$ \pi_1 (S^n) = \{e\} $$
\end{ex}
\begin{ex}
    $$ \pi_1 (\mathbb{R}^n) = \{e\}$$
\end{ex}

\begin{thm}
    $$\pi_1 (S^1) = \mathbb{Z} $$
\end{thm}

\begin{proof}
    Given any integer $n\in \mathbb{Z}$, we give a loop by associatng
    each $n$ 
    \begin{align*}
        \pi : \mathbb{R} \to S^1 \\
        t \mapsto e^{2\pi it}
    \end{align*}
    Define 
    \begin{align}
        \gamma_n : [0,1] \to \mathbb{R} \\
        s\mapsto n s
    \end{align}
    Then $\phi_n:= \pi\circ \gamma_n$ has the property that
    $\phi_n(0)=1$, $\phi_n(1)=1$. So we obtain 
    \begin{equation}
        \phi: \mathbb{Z} \to \pi_1  (S^1)
    \end{equation}
    Geometrically, we $\phi_n$ loops around $S^1$ in $n$ turns.

    We need to prove that $\phi$ is a isomorphism. This is done by:
    \begin{enumerate}
        \item Prove that $\phi$ is a homomorphism;
        \item Prove that $\phi$ is bijective.
    \end{enumerate}
    First,
    \begin{align*}
        \gamma_n: s \mapsto n s \\
        \gamma_m: s\mapsto m s\\
    \end{align*}
    We need
    \begin{align*}
        \braket{\pi\circ \gamma_{m+n}} =
        \braket{\pi\circ\gamma_m}\braket{\pi\circ \gamma_n}\\
    \end{align*}
    Define $\sigma:[0,1]\to \mathbb{R}$, $s\mapsto \gamma_n(s)+m$,
    this is a translation of real line. Then $\pi\circ \sigma =
    \pi\circ \gamma_n$. Then
    \begin{align*}
        \braket{\pi\circ\gamma_m}\braket{\pi\circ \gamma_n}
        = \braket{\pi\circ\gamma_m} \braket{\pi\circ\sigma}
        = \braket{\pi\circ(\gamma_m\circ \sigma)}
    \end{align*}
    $\gamma_{m+n}$ has the same domain and codomain of $\gamma_m \circ
    \sigma$, and they obviously share the same start and the same end
    point. Therefore these two path are homotopic relative to
    $\{0,1\}$.
    Therefore
    \begin{equation}
        \braket{\pi\circ \gamma_{m+n}} =
        \braket{\pi\circ(\gamma_m\circ\sigma)}
    \end{equation}
    Or
    \begin{equation}
        \phi_{m+n} = \phi_{m}\phi_n
    \end{equation}

    Second, we need to show that this map is surjective. Notice that
    $\pi:\mathbb{R}\to S^1$, $t\mapsto e^{2\pi i t}$, is like a
    projection of a circulatory path onto a circle $S^1$. This map is
    locally homeomorphic. We can find a cover of $S^1$ as the
    combination of
    \begin{align*}
        U = S^1\setminus \{-1\} \\
        V = S^1\setminus \{1\}
    \end{align*}
    Then $\pi^{-1}(V)$ are the intervals on $\mathbb{R}$ excluding the
    whole integer points. Similarly, $\pi^{-1}(U)$ are those intervals
    on $\mathbb{R}$ excluding those half-integer points. In each of those
    intervals the map $\pi$ is bijective.
    Now we need a lemma:
    \begin{lemma}[Path-lifting lemma]
        \begin{equation}
            \begin{tikzcd}[]
                \, & \mathbb{R}\ar[d,"\pi"] \\
                {[0,1]}\ar[r,"\sigma"]\ar[ru,"f"] & S^1
            \end{tikzcd}
        \end{equation}
        Assuming we have $\pi$ and $\sigma$, both are continuous maps.
        More specifically, $\sigma$ is a path in $S^1$ which begins at the
        point $1$. Then there is a unique path $\tilde\sigma$ in
        $\mathbb{R}$ which begins at $0$ and satisfies $\pi\circ
        \tilde\sigma = \sigma$.
    \end{lemma}
    \begin{proof}
        By Lebesgue lemma, we can divide the interval $[0,1]$ fine enough
        such that each part  is maped to only one of the cover $U$ or $V$.
        We thus break a path % TODO
    \end{proof}
    Note that $\tilde\sigma(0)=0$, $\tilde\sigma(1)$ is an integer.
    Now for any loop $\gamma:[0,1]\to S^1$ based at $1$, we can find a
    lifting $\tilde\gamma:[0,1]\to \mathbb{R}$ such that
    $\tilde\gamma(0)=0$,$\tilde\gamma(1)=n$, and $\gamma = \pi\circ
    \tilde\gamma$. Then $\tilde\gamma \cong \gamma_n$ rel $\{0,1\}$,
    also $\braket{\pi\circ\tilde\gamma}=\braket{\pi\circ\gamma_n}$.
    Hence for any path $\gamma$ we find a $n$ such that
    $\gamma=\phi(n)$. So the map is surjective.

    We need another lemma to prove that it is injective.
    \begin{lemma}[Homotopy-lifting lemma]
        If $F:[0,1]\times[0,1]\to S^1$ is a map such that
        $F(0,t)=F(1,t)=1$ for $0\leq t\leq 1$, then there exists a
        unique $\tilde F: [0,1]\times[0,1]\to \mathbb{R}$ such that
        \begin{align}
            \pi\circ \tilde F = F \\
            \tilde F(0,t) =0,\, 0\leq t\leq 1
        \end{align}
    \end{lemma}
    % TODO Graph here.
    \begin{proof}
        We need the Lebesgue lemma. Let $S^1=U\cup V$ as before.
    \end{proof}
    Now we proof that the map $\phi$ is injective. Suffice to prove
    that $\Ker(\phi)$ is trivial. Suppose $\phi(n)=\pi\circ \gamma_n$
    is homotopic to the constant loop. Then choose a homotopy $F$
    from $\pi\circ\gamma_n$ to the constant loop. By the
    homotopy-lifting lemma we can find $\tilde F: [0,1]\times[0,1]\to
    \mathbb{R}$ such that $\pi\circ \tilde F = F$. Also $\tilde
    F(0,t)=0$.
    % TODO  We can find the vertical bottom is 0 and verticl height is
    % gamma_n. Right line is integers and can only be 0 Hence gamma_n
    % starts at 0 and ends at 0
    One can find that $\gamma_n\cong 0$. This completes the proof of
    injectivity. Hence completes the whole proof.
\end{proof}

We have an application,
\begin{thm}[Brow Fixed Point theorem] % TODO brower? broer?
    A contiuous map $f:B^2\to B^2$ ($B^2:2D$-closed dicks) must have a
fixed point. That is, $\exists x\in B^2$ such that $f(x)=x$.
\end{thm}
\begin{proof}
    Assuming that this theorem is false, that is $\forall x\in B^2$,
    $x\neq f(x)$, then we have a straight path from $f(x)$ to $x$. We can
    extends this path to cuts the boundary of $B^2$ at $h(x)$. This is
    for all $x\in B^2$, hence we have a map $h: B^2 \to S^1$. Also,
    $h|_{S^1}$ is obvious an identity map. But $S^1$ can be included
    inside $B^2$, so we have:
    \begin{equation}
        S^1 \to B^2 \to S^1 % TODO add function labels
    \end{equation}
    Hence we have a series of homomorphism of fundamental groups:
    \begin{equation}
        \pi_1 (S^1) \to \pi_1 (B^2) \to \pi_1(S^1)
    \end{equation}
    and the composite is identity map. But observe that $B^2$ is a
    convex set and hence its fundamental group is trivial. But $S^1$
    has non-trivial fundamental group.
\end{proof}
\begin{remark}
    This theorem can be extended to higher dimensional case. But the
    proof cannot be the same because for higher dimension $\pi_1(S^n)$
    is no longer non-trivial.
\end{remark}

Another application, which we need a theorem to help:
\begin{thm}
    \begin{equation}
        \pi_1 (X\times Y, (x_0,y_0)) = \pi_1 (X,x_0) \otimes \pi_1
        (X,y_0)
    \end{equation}
\end{thm}
\begin{proof}
    We use the projection maps: $P_1$ and $P_2$.
    Then, the map
    \begin{align*}
        (P_1)_*: \pi_1 (X\times Y) \to \pi_1(X) \\
        (P_2)_*: \pi_1 (X\times Y) \to \pi_1(Y)
    \end{align*}
    and their composition formed into
    \begin{align*}
        \braket{\alpha}\mapsto
        (\braket{P_1\circ\alpha},\braket{P_2\circ\alpha})
    \end{align*}
    this map is surjective, injective, and is homomorphism.
    because any %TODO
\end{proof}

\begin{fact}
By this theorem, the two objects $S^2$ and $S^1\times S^1$ is not
homeomorphic, since their fundamental groups are not the same (the
former is trivial and the later is $\mathbb{Z}\times\mathbb{Z}$).
\end{fact}


% TODO homework: pro 13 of page 95.


