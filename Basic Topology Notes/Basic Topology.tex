% The entire content of this work (including the source code
% for TeX files and the generated PDF documents) by 
% Hongxiang Chen (nicknamed we.taper, or just Taper) is
% licensed under a 
% Creative Commons Attribution-NonCommercial-ShareAlike 4.0 
% International License (Link to the complete license text:
% http://creativecommons.org/licenses/by-nc-sa/4.0/).
\documentclass{article}

% My own physics package
% The following line load the package xparse with additional option to
% prevent the annoying warnings, which are caused by the package
% "physics" loaded in package "physicist-taper".
\usepackage[log-declarations=false]{xparse}
\usepackage{physicist-taper}

\makenomenclature

% Show the keys for labels, replace options with "final" when done
% with editing.
\usepackage[draft,notref]{showkeys}

\title{Notes of Basic Topolgy}
\date{\today}
\author{Taper}


\begin{document}


\maketitle
\abstract{
    A note of Basic Topology, based on \textit{Basic Topology} by M.A.
    Armstrong.
}
\tableofcontents
There are several parts that I will skipped for convenience. Those
include chapter 1 - Introduction, chapter 2 - Continuity, chapter 3 -
Compactness and Connectedness, and chapter 4 - Identification Spaces.
Below is some especially confusing part that I would like to note:

\section{Special Notes}
\label{sec:Special-Notes}
\paragraph{Basic facts about maps}
Assuming domain $f=X$, codomain $f=Y$.
\begin{align}
    f(U\cup V) &= f(U)\cup f(V) \\
    f(U\cap V) &= f(U)\cap f(V) \\
    f(U^c) &\supseteq f(U)^c,\,\text{i.e. } f(U)^c \subseteq f(U^c) \\
    f^{-1}(U\cup V) &= f^{-1}(U)\cup f^{-1}(V) \\
    f^{-1}(U\cap V) &= f^{-1}(U)\cap f^{-1}(V) \\
    f^{-1}(U^c) &= [f^{-1}(U)]^c
\end{align}
\paragraph{Smallest the Largest Topolgy}
The set of all possible topolgies on $X$ is partially ordered by
inclusion. For a certain characteristics $\mathcal{C}$, it is possible
to have the smallest or the largest one. 

The \nomen{smallest topolgy} $\mathcal{T}_\text{min}$ is the one such
that, for any $\mathcal{T}'$ satisfying $\mathcal{C}$,
$\mathcal{T}_\text{min}\subseteq \mathcal{T}'$. The \nomen{largest
topolgy} $\mathcal{T}_\text{max}$ is the one such that, for any
$\mathcal{T}'$ satisfying $\mathcal{C}$, $\mathcal{T}'\subseteq
\mathcal{T}_\text{max}$.

For example, assuming we have
\begin{equation}
    f: X\to Y
\end{equation}
where $f$ is any function.

If $X$ has topolgy $\mathcal{T}_X$, we ask then what kind of topolgy on
$Y$ will make $f$ a continuous function. First, all $f^{-1}(V)$, with
$V\in \mathcal{T}_Y$ should be open in $X$. So, the easiest choice is to
make $\mathcal{T}_{Y,\text{min}}=\{\varnothing,Y\}$, this is the
smallest topolgy.  Also, any set $V\in Y$ such that $f^{-1}(V)\notin
\mathcal{T}_X$ should not be in $\mathcal{T}_Y$. Then the largest
topolgy is $\mathcal{T}_{Y,\text{max}}=\{ V\subset Y| f^{-1}(V)\in
\mathcal{T}_X\}$.

If $Y$ has topolgy $\mathcal{T}_Y$, we also ask what kind of topolgy
on $X$ will make $f$ a continuous function. First, all $V\in
\mathcal{T}_Y$, their preimage $f^{-1}(V)$ must be in $\mathcal{T}_X$.
So the smallest topolgy is
$\mathcal{T}_{X,\text{min}}=\{f^{-1}(V)|V\in\mathcal{T}_Y\}$.
\section{Anchor}
\label{sec:Anchor}

\begin{thebibliography}{1}
    \bibitem{book} M.A. Armstrong. Basic Topology. 2ed.
\end{thebibliography}
\printnomenclature
\section{License}
The entire content of this work (including the source code
for TeX files and the generated PDF documents) by 
Hongxiang Chen (nicknamed we.taper, or just Taper) is
licensed under a 
\href{http://creativecommons.org/licenses/by-nc-sa/4.0/}{Creative 
Commons Attribution-NonCommercial-ShareAlike 4.0 International 
License}. Permissions beyond the scope of this 
license may be available at \url{mailto:we.taper[at]gmail[dot]com}.
\end{document}
