% The entire content of this work (including the source code
% for TeX files and the generated PDF documents) by 
% Hongxiang Chen (nicknamed we.taper, or just Taper) is
% licensed under a 
% Creative Commons Attribution-NonCommercial-ShareAlike 4.0 
% International License (Link to the complete license text:
% http://creativecommons.org/licenses/by-nc-sa/4.0/).
\documentclass{article}

% My own physics package
% The following line load the package xparse with additional option to
% prevent the annoying warnings, which are caused by the package
% "physics" loaded in package "physicist-taper".
\usepackage[log-declarations=false]{xparse}
\usepackage{physicist-taper}

\title{Simons's Algorithm}
\date{\today}
\author{Hongxiang Chen}

\begin{document}

\maketitle
\abstract{
  Based on Watrous's notes.
}

\section{Why the probability looks in that way}
\label{sec:Why the probability looks in that way}

We are asked of the probability of finding $n-1$ linear independent vectors
$\vec{y}_1,\cdots,\vec{y}_{n-1}$ in the vectors space of dimension $n$ over
binary numbers satisfying the condition:
\begin{equation}
  \vec{s}\vdot\vec{y_i} = 0, \forall i=1,\cdots,n-1
\end{equation}
We could assume that all vectors satisfying the above condition appear with equal
probability.

Since the binary numbers are quite limited, this is a discrete probability and
we could count this easily. Assuming we have founded $m$ linearly independent
vectors $\vec{y}_1,\cdots,\vec{y}_m$, without loss of generality, we could
assume:
\begin{align}
  \vec{y}_1 &= (1,0,\cdots,0), \\
  \vec{y}_2 &= (0,1,0,\cdots,0), \\
            &\cdots, \nonumber\\
  \vec{y}_m &= (0,\cdots,0,1,0,\cdots,0)
\end{align}

Then over all $2^{n-1}$\footnote{Because the condition $\vec{y}\vdot\vec{s}=0$
reduces the dimension by $1$. Formally, this is the result of Rank–nullity
theorem over the field $\mathbb{Z}_2$.} possible vectors, $2^m$ vectors can be linearly expressed by the
above founded vectors. Therefore, the probability of finding the next linearly
independent vector is $\left(1-\frac{2^m}{2^{n-1}}\right)$. Therefore, overall the probability
of finding $n-1$ linearly independent vectors are:
\begin{equation}
  \prod_{k=1}^{n-2}\left(1-\frac{2^k}{2^{n-1}}\right) = \prod_{k=1}^{n-2}
  \left(1-\frac{1}{2^k}\right)
\end{equation}
This is clearly decreasing number w.r.t $n$, hence in the asymptotic limit the
probability is:
\begin{equation}
  \prod_{k=1}^{\infty}\left(1-\frac{1}{2^k}\right).
\end{equation}
\end{document}
