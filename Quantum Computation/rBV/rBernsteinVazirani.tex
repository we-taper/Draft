% The entire content of this work (including the source code
% for TeX files and the generated PDF documents) by 
% Hongxiang Chen (nicknamed we.taper, or just Taper) is
% licensed under a 
% Creative Commons Attribution-NonCommercial-ShareAlike 4.0 
% International License (Link to the complete license text:
% http://creativecommons.org/licenses/by-nc-sa/4.0/).
\documentclass{article}

% for lwarp package: converting latex to html.
\usepackage{iftex}

% --- LOAD FONT SELECTION AND ENCODING BEFORE LOADING LWARP ---

\ifPDFTeX
\usepackage{lmodern}            % pdflatex
\usepackage[T1]{fontenc}
\usepackage[utf8]{inputenc}
\else
\usepackage{fontspec}           % XeLaTeX or LuaLaTeX
\fi

% --- LWARP IS LOADED NEXT ---
\usepackage[
%   latexmk,                    % Use latexmk to compile.
%   mathjax,                    % Use MathJax to display math.
]{lwarp}

% --- LATEX AND HTML CUSTOMIZATION ---
\setcounter{tocdepth}{2}        % Include subsections in the \TOC.
\setcounter{secnumdepth}{2}     % Number down to subsections.
\setcounter{FileDepth}{-5}       % To place the entire file into one html page
\booltrue{CombineHigherDepths}  % Combine parts/chapters/sections
\setcounter{SideTOCDepth}{1}    % Include subsections in the side\TOC
\HTMLLanguage{en-UK}            % Sets the HTML meta language.
\CSSFilename{lwarp_formal.css}

% --- OTHER PACKAGES ARE LOADED AFTER LWARP ---
% My own physics package
% The following line load the package xparse with additional option to
% prevent the annoying warnings, which are caused by the package
% "physics" loaded in package "physicist-taper".
% \usepackage[log-declarations=false]{xparse}
\usepackage[final]{physicist-taper}

% for lwarp to work with tikz
\tikzset{every picture/.style={ampersand replacement=\&}}

\title{The Recursive Bernstein-Vazirani Algorithm}
\date{\today}
\author{Hongxiang Chen}


\begin{document}


\maketitle
\abstract{
Based on notes available at:
\url{https://courses.cs.washington.edu/courses/cse599d/06wi/lecturenotes7.pdf}
}
\tableofcontents

\section{Notation}
\label{sec:Notation}

A $\vec{x}$ is a vector of $n$ binary numbers. A summation $\sum_{\vec{x}}$ is a
summation over all possible $n$ binary numbers: $\sum_{\vec{x}}=\sum_{x\in
  \{0,1\}^n}$. A dot product between two vectors is $\vec{x}\vdot\vec{y} =
  \sum_{i=0}^{n-1} x_i y_i \text{ mod }2$.

\section{Tricks}
\label{sec:Tricks}

\subsection{Hadamard gate}
\label{sec:Hadamard gate}

The Hadamard gate on $n$ qubits $H^{\otimes n}$ can be written in the following notation:

\begin{equation}
  H^{\otimes n} = \frac{1}{\sqrt{2^n}}\sum_{\vec{x},\vec{y}} (-1)^{\vec{x}\vdot\vec{y}}
  \ket{\vec{x}}\bra{\vec{y}}.
\end{equation}

Define $\ket{P_{\vec{x}}}$ as
\begin{equation}
  \ket{P_{\vec{x}}} = \frac{1}{\sqrt{2^n}} \sum_{\vec{y}}
  (-1)^{\vec{x}\vdot\vec{y}} \ket{\vec{y}}.
\end{equation}
One can verify that
\begin{equation}
  \braket{P_{\vec{x}} |P_{\vec{y}}} = \delta_{\vec{x},\vec{y}},\,
  \sum_{\vec{x}}\ketbra{P_{\vec{x}}} = \id.
\end{equation}
And the Hadamard gate could be written as
\begin{equation}
  H^{\otimes n} = \sum_{\vec{x}} \ket{\vec{x}}\bra{P_{\vec{x}}} 
  = \sum_{\vec{x}} \ket{P_{\vec{x}}} \bra{\vec{x}}.
\end{equation}
Therefore the Hadamard gate transfer between the basis where information is
recorded in qubit register ($\ket{\vec{x}}$) and the basis where information is
recorded in the phase ($\ket{P_{\vec{x}}}$). A special state which will be used
often is
\begin{equation}
  \ket{P_0} = \frac{1}{\sqrt{2^n}} \sum_{\vec{x}}\ket{\vec{x}}.
\end{equation}

\subsection{Phase kickback}
\label{sec:Phase kick}
For a classical function $f:\{0,1\}^n\to \{0,1\},\, \vec{x}\mapsto f(\vec{x})$,
its analogous quantum gate is $U_f:\ket{\vec{x},y} \mapsto \ket{\vec{x},y\oplus
f(x)}$. Quantum computation can calculate this in a parallel manner as a phase
gate by
\begin{equation}
  U_f \left(\sum_{\vec{x}} \phi(\vec{x})  \ket{\vec{x}}
  \otimes \frac{1}{\sqrt{2}} (\ket{0}-\ket{1}) \right)
  = \sum_{\vec{x}} (-1)^{f(\vec{x})} \phi(\vec{x})\ket{\vec{x}} 
  \otimes \frac{1}{\sqrt{2}} (\ket{0}-\ket{1}),
\end{equation}
where $\phi(\vec{x})$ are some arbitrary non-zero phase. If we ignore the
ancilla qubit, then $U_f$ is acting as a phase $(-1)^{f(\vec{x})}$ on every
$\ket{\vec{x}}$ on ``parallel".


\section{The recursive Bernstein-Vazirani}
\label{sec:The recursive Bernstein-Vazirani}

The recursive Bernstein-Vazirani problem (rBV) of level $k$ asks us to find an n-bit number
$\vec{s}$ through a series of oracles provided. The first oracle provided is
$f:\{0,1\}^{nk}\to \{0,1\},\, \{\vec{x}_1,\cdots, \vec{x}_k\}\mapsto
\vec{x}_k\vdot \vec{s}_{\vec{x}_1,\cdots,\vec{x}_{k-1}}$, where we need to find
out the $n$ number $\vec{s}_{\vec{x}_1,\cdots,\vec{x}_{k-1}}$. The following
oracles are 
\begin{equation}
  g_j: \{0,1\}^n\to \{0,1\},\, \vec{s}_{\vec{x}_1,\cdots,\vec{x}_{k-1}}
  \mapsto \vec{x}_j \vdot \vec{s}_{\vec{x}_1,\cdots,\vec{x}_{j-1}},\,
  j=k-1,k-2,\cdots,1.
\end{equation}
The final $g_1$ is slightly different because
$g_1(\vec{s}_{\vec{x}_1})=\vec{s}\vdot\vec{x}_1$. Here we would step by step
find out the numbers $\vec{s}_{\vec{x}_1,\cdots,\vec{x}_j}$, until finally we
have $\vec{s}$. Classically the query complexity is lowered bounded as
$\Omega(n^k)$.

\subsection{A quantum algorithm}
\label{sec:A quantum algorithm}

Here we present a simple quantum algorithm to the $k$th level rBV problem. The
quantum register are initialised in the state:
\begin{equation}
  \ket{P_0}^{\otimes k}\otimes \ket{-} \otimes \cdots \otimes \ket{-} 
  = \left(\frac{1}{\sqrt{2^n}}\right)^k
  \sum_{\vec{x}_1,\cdots,\vec{x}_{k}} \ket{\vec{x}_1,\cdots,\vec{x}_k}
  \ket{-} \otimes \cdots \otimes \ket{-}.
\end{equation}

Here the first $k\times n$ qubits are grouped into $k$ qubit registers, each
having $n$ qubits. The rest of qubits are ancilla qubits prepared in the
$\ket{-} = (\ket{0}-\ket{1})/\sqrt{2} $ state. Although for simplicity we will
ignore them in the following calculation, it should be noted that the ancilla
qubits are important for the phase kickback trick mentioned before. Also for
simplicity, we will ignore all normalisation constant for for the following
discussion. Therefore the initial state of qubit register is written as:
\begin{equation}
  \sum_{\vec{x}_1,\cdots,\vec{x}_{k}} \ket{\vec{x}_1,\cdots,\vec{x}_k}
\end{equation}

Define the quantum version of oracles as
\begin{align}
  & f\ket{\vec{x}_1,\cdots,\vec{x}_k} =
  (-1)^{\vec{x}_k\vdot\vec{s}_{\vec{x}_1,\cdots,\vec{x}_{k-1}}} 
  \ket{\vec{x}_1,\cdots,\vec{x}_k},\\
  & g_j \ket{\vec{x}_1,\cdots, \vec{x}_j, 
  \vec{s}_{\vec{x}_1,\cdots,\vec{x}_j} ,\vec{x}_{j+2},\cdots,\vec{x}_k}\nonumber\\
  & = (-1)^{\vec{x}_j \vdot \vec{s}_{\vec{x}_1,\cdots,\vec{x}_{j-1}}}
    \ket{\vec{x}_1,\cdots, \vec{x}_j, 
    \vec{s}_{\vec{x}_1,\cdots,\vec{x}_j} ,\vec{x}_{j+2},\cdots,\vec{x}_k},\nonumber\\
  & \;\text{for } j=k-1,k-2,\cdots ,2,\\
  & g_1 \ket{\vec{x}_1, \vec{s}_{\vec{x}_1}, \vec{x}_3, \cdots,\vec{x}_k} =
  (-1)^{\vec{x}_1 \vdot \vec{s}} \ket{\vec{x}_1, \vec{s}_{\vec{x}_1}, \vec{x}_3, \cdots,\vec{x}_k},
\end{align}
where we have used the phase kickback trick mentioned in section~\ref{sec:Phase
kick} and ignored the ancilla qubits. Notice that all quantum oracles are
meaningless phase gate unless a superposition of basis
$\ket{\vec{x}_1,\cdots,\vec{x}_k}$ are present. In the following, we will always
ensure a superposition of the form 
\[
  \sum_{\vec{x}_1,\cdots,\vec{x}_j} \phi(\vec{x}_1,\cdots,\vec{x}_j,\text{(something)})
  \ket{\vec{x}_1,\cdots,\vec{x}_j,\text{(something)}},
\]
is present for $g_j$ to act on. Here $\phi$ are some non-zero phases.

A quantum algorithm to solve the rBV problem depends on the following operators
$D_j(j=1,\cdots,k)$, defined recursively by:

\begin{align}
  &D_k = H^{\otimes n}_{k} f, \\
  &D_{k-1} = H^{\otimes n}_{k-1} D^\dagger_k g_{k-1}, \\
  &D_{k-2} = H^{\otimes n}_{k-2} D^\dagger_k D^\dagger_{k-1} g_{k-2}, \\
  &\cdots \\
  &D_1 = H^{\otimes n}_1 D^\dagger_k D^\dagger_{k-1}\cdots D^\dagger_2 g_1.
\end{align}


Here the $H^{\otimes n}_j$ is simply the $n$ bit Hadamard gate $H^{\otimes n}$ acting
on the $j$th register. Each operator $D_{k-j}$ solve the $j$th level rBV
problem. Specifically, the first operator $D_k$ does the following:
\begin{align}
  & D_k \sum_{\vec{x}_1,\cdots,\vec{x}_k} \ket{\vec{x}_1,\cdots,\vec{x}_k}\nonumber\\ 
  = & H^{\otimes n}_k \sum_{\vec{x}_1,\cdots,\vec{x}_k}
  (-1)^{\vec{x}_k\vdot\vec{s}_{\vec{x}_1,\cdots,\vec{x}_{k-1}}} 
  \ket{\vec{x}_1,\cdots,\vec{x}_k} \\
  = & H_k^{\otimes n} \sum_{\vec{x}_1,\cdots,\vec{x}_{k-1}}
  \ket{\vec{x}_1,\cdots,\vec{x}_{k-1},P_{\vec{s}_{\vec{x}_1,\cdots,\vec{x}_{k-1}}}}
  \\
  = & \sum_{\vec{x}_1,\cdots,\vec{x}_{k-1}}
  \ket{\vec{x}_1,\cdots,\vec{x}_{k-1},\vec{s}_{\vec{x}_1,\cdots,\vec{x}_{k-1}}}.
\end{align}

Notice that if there are some arbitrary non-zero phases
$\phi(\vec{x}_1,\cdots,\vec{x}_k)$ in front of
$\ket{\vec{x}_1,\cdots,\vec{x}_k}$ on the first line, the final result will not
change except from this arbitrary phase. $D_k$ can be understood as the following
diagram:
\displaymathother
\begin{equation}
\begin{tikzcd}
  \ket{\cdots,\vec{x}_k} \arrow[r, bend left, "D_k"] \& 
   \arrow[l, bend left, "D^\dagger_k"] 
   \ket{\cdots,\vec{s}_{\vec{x}_1,\cdots,\vec{x}_{k-1}}}.
\end{tikzcd}
\end{equation}
\displaymathnormal

The second operator $D_{k-1}$ does the following next:
\begin{align}
  & D_{k-1} \sum_{\vec{x}_1,\cdots,\vec{x}_{k-1}}
  \ket{\vec{x}_1,\cdots,\vec{x}_{k-1},\vec{s}_{\vec{x}_1,\cdots,\vec{x}_{k-1}}}
  \\
  =& H^{\otimes n}_{k-1} D^\dagger_k \sum_{\vec{x}_1,\cdots,\vec{x}_{k-1}} 
  (-1)^{\vec{x}_{k-1}\vdot \vec{s}_{\vec{x}_1,\cdots,\vec{x}_{k-2}}}
  \ket{\vec{x}_1,\cdots,\vec{x}_{k-1},\vec{s}_{\vec{x}_1,\cdots,\vec{x}_{k-1}}}
  \\
  =& H^{\otimes n}_{k-1} \sum_{\vec{x}_1,\cdots,\vec{x}_{k}}
  (-1)^{\vec{x}_{k-1}\vdot \vec{s}_{\vec{x}_1,\cdots,\vec{x}_{k-2}}}
  \ket{\vec{x}_1,\cdots,\vec{x}_{k-1},\vec{x}_{k}}
  \\
  =& H^{\otimes n}_{k-1} \sum_{\vec{x}_1,\cdots,\vec{x}_{k-2}}
  \ket{\vec{x}_1,\cdots,\vec{x}_{k-2},
  P_{\vec{s}_{\vec{x}_1,\cdots,\vec{x}_{k-2}}},P_0}
  \\
  =& \sum_{\vec{x}_1,\cdots,\vec{x}_{k-2}}
  \ket{\vec{x}_1,\cdots,\vec{x}_{k-2}, \vec{s}_{\vec{x}_1,\cdots,\vec{x}_{k-2}},P_0}
\end{align}

Similarly, we have the following diagram for $D_{k-1}$:
\displaymathnormal
\begin{equation}
\begin{tikzcd}
  \ket{\cdots,\vec{x}_{k-1},\vec{s}_{\vec{x}_1,\cdots,\vec{x}_{k-1}}} 
  \arrow[r, "\text{load phase}", "g_{k-1}" below,]
  \dar[bend left, "D_{k-1}"]
  \& 
  (-1)^{\cdots} \ket{\vec{x}_1,\cdots,\vec{x}_{k-1},\vec{s}_{\vec{x}_1,\cdots,\vec{x}_{k-1}}}
  \dar["D^\dagger_k"]
  \\
  \ket{\cdots,\vec{s}_{\vec{x}_1,\cdots,\vec{x}_{k-2}},P_0}
  \uar[bend left, "D^\dagger_{k-1}"]
  \&
  (-1)^{\cdots} \ket{\vec{x}_1,\cdots,\vec{x}_{k-1},\vec{x}_{k}}
  \lar["\text{unload phase}" above, "H"]
\end{tikzcd}
\end{equation}
\displaymathother

Similarly for $D_{k-j}$, we use $g_j$ to load the phase 
$(-1)^{\vec{x}_j\vdot \vec{s}_{\cdots}}$, and use $D^\dagger_k\cdots
D^\dagger_{k-j+1}$ to solve the lower level rBV problem, and use Hadamard gate
to unload the phase into basis:

\displaymathother
\begin{equation}
\begin{tikzcd}[row sep=large, column sep = large]
  \ket{\cdots,\vec{x}_{k-j},\vec{s}_{\vec{x}_1,\cdots,\vec{x}_{k-j}},P_0^{\otimes j-1}} 
  \arrow[r, "\text{load phase}", "g_j" below,]
  \dar[bend left, "D_{k-j}"]
  \& 
  (-1)^{\cdots} 
  \ket{\cdots,\vec{x}_{k-j},\vec{s}_{\vec{x}_1,\cdots,\vec{x}_{k-j}},P_0^{\otimes j-1}} 
  \dar["D^\dagger_k\cdots D^\dagger_{k-j+1}"]
  \\
  \ket{\cdots,\vec{s}_{\vec{x}_1,\cdots,\vec{x}_{k-j-1}},P_0,P_0^{\otimes j-1}}
  \uar[bend left, "D^\dagger_{k-j}"]
  \&
  (-1)^{\cdots} \ket{\vec{x}_1,\cdots,\vec{x}_{k-j}, \vec{x}_{k-j+1},\cdots,\vec{x}_{k}}
  \lar["\text{unload phase}" above, "H"]
\end{tikzcd}
\end{equation}
\displaymathnormal

The operators $D^\dagger_k\cdots D^\dagger_{k-j+1}$ solves the lower level
problem by gradually moving the $s$ number to the end:
\displaymathother
\begin{equation}
\begin{tikzcd}
  \ket{\cdots,\vec{s}_{\vec{x}_1,\cdots,\vec{x}_{k-j}},P_0,P_0^{\otimes j-2}}
  \rar["D^\dagger_{k-j+1}"]
  \&
  \ket{\cdots,\vec{x}_{k-j+1}, \vec{s}_{\vec{x}_1,\cdots,\vec{x}_{k-j+1}},P_0^{\otimes j-2}}
  \dar["\cdots"]
  \\
  \ket{\vec{x}_1,\cdots,\vec{x}_{k-1},\vec{x}_{k}}
  \&
  \ket{\cdots,\vec{x}_{k-j+1},\cdots,\vec{x}_{k-1},\vec{s}_{\vec{x}_{k-1}}}
  \lar["D_k"]
\end{tikzcd}
\end{equation}
\displaymathnormal

And finally we note that:
\begin{equation}
  D_1 \sum_{\vec{x}_1}\ket{\vec{x}_1, \vec{s}_{\vec{x}_1}, P_0^{\otimes k-2}} =
  \ket{\vec{s}, P_0^{\otimes k-1}}.
\end{equation}
With a measurement in the computational basis, one can with certainty find the
number $\vec{s}$. Therefore, the combined operators $D_1D_2\cdots D_k$ solves
the $k$th level rBV problem.

\subsection{Complexity of quantum algorithm}
\label{sec:Complexity of quantum algorithm}
The number of gates ($n$ bit Hadamard and quantum queries) taken in $D_k$ is
$2$. In $D_{k-1}$ is $2+2=2^2$, in $D_{k-2}$ is $2+2^2+2=2^3$. In general, in
$D_{k-j}$ there are
\begin{equation}
  2^1 + 2^2 + \cdots + 2^{j-1} + 2 = 2^{j+1},
\end{equation}
$n$ bit Hadamard gates and quantum query gates. Hence, the total complexity is:
\begin{equation}
  2 + 2^2 + \cdots + 2^k = 2^{k+1}-2 \approx O(2^k)
\end{equation}

If $k=\log_2 n$, then the quantum complexity is $O(n)$ whereas the classical
complexity is lower bounded as $\Omega(n^{\log_2 n})$.

\end{document}
