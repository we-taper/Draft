% The entire content of this work (including the source code
% for TeX files and the generated PDF documents) by 
% Hongxiang Chen (nicknamed we.taper, or just Taper) is
% licensed under a 
% Creative Commons Attribution-NonCommercial-ShareAlike 4.0 
% International License (Link to the complete license text:
% http://creativecommons.org/licenses/by-nc-sa/4.0/).
\documentclass{article}

% My own physics package
% The following line load the package xparse with additional option to
% prevent the annoying warnings, which are caused by the package
% "physics" loaded in package "physicist-taper".
\usepackage[log-declarations=false]{xparse}
\usepackage{physicist-taper}
\makenomenclature % For an index of symbols.

\title{Dirac Hamiltonian}
\date{\today}
\author{Taper}


\begin{document}


\maketitle
\abstract{
This aims for recapitulate important information about Dirac
Hamiltonian as is written in Sakurai's text \cite{Sakurai2011},
chapter 8.\footnote{Be careful that some version of this book is
missing chapter 8... I don't understand why...} This is intended to
learn to help me deal with a Lattice Dirac Hamiltonian.
}
\tableofcontents

The starting point is still the Schrodinger equation:

\begin{equation}
    i\hbar \pdv{t}\ket{\psi(t)} = H\ket{\psi(t)}
\end{equation}

We will be using \textbf{Natural Units} from now on.

\section{Klein-Gorden Equations}
\label{sec:Klien-Gorden Equations}

The Klein-Gorden equation starts with the the relativistic energy (of
a particle with momentum $\vb{p}$ and mass $m$):
\begin{equation}
    E_p = +\sqrt{p^2+m^2}
\end{equation}
The plus sign here hints that we may have a negative energy solution.

We do not quantize this equation directly, because the square root is
hard to represent in operators. Instead, Klein-Gorden equation starts
with $H^2$:
\begin{equation}
    H^2 = p^2 + m^2 \overset{\text{quantized as}}{\to}
    -\grad^2 + m^2
\end{equation}
On the other hand, the Schrodinger equation applying twice gives:
\begin{equation}
    -\pdv[2]{t} = H^2
\end{equation}
Hence we have the Klein-Gorden equation:
\begin{equation}
    \label{eq:kg-eq}
    \left(\pdv[2]{t} - \grad^2 + m^2\right) \psi(x,t) = 0
\end{equation}
It can be simplified in $4$-vector notation as:
\begin{equation}
    \label{eq:kg-eq-4vec}
    \left(\partial_\mu \partial^\mu + m^2\right) \psi(x,t) = 0
\end{equation}
or most succinctly as,
\begin{equation}
    \left(\partial^2 + m^2\right) \psi(x,t) = 0
\end{equation}
The author looked at this equation from three different aspects.

\paragraph{Producing the free particle solution} It is first checked
that the free-particle state:
\begin{equation}
    \psi \propto \exp(-i(Et-\vb{p}\vdot\vb{x})) = e^{-ip^\mu x_\mu}
\end{equation}
is a solution to the Klein-Gorden equation, provided that:
\begin{equation}
    E^2 = \vb{p}^2+m^2
\end{equation}

\paragraph{Probability density} By analogy with non-relativistic case,
the four-vector current is defined as:
\begin{equation}
    j^\mu = \frac{i}{2m}
    [\psi^*\partial^\mu\psi - (\partial^\mu\psi)^*\psi]
\end{equation}
It is easy to show $\partial_\mu j^\mu=0$\footnote{Just note that
normally we have $A^\mu B_\mu = A_\mu B^\mu$}. Then, this is a
conserved current. The time-component of this is the density:
\begin{equation}
    \rho \equiv j^0 = \frac{i}{2m}
    \left[\psi^*\partial_t\psi - (\partial_t\psi)^*\psi\right]
\end{equation}
If one check this equation, one find that $\rho$ is proportionally to
the imaginary of some number which, one has no clear reason to say
whether it is positive or negative. Therefore, the Klein-Gorden
equation leaves one in a bad position to interprate the probability
nature of quantum mechanics (what is a "negative probability"
anyway?).

\paragraph{Coupling to electromagnetic field} The author discusses a
little bit about why, when there is electromagnetic field, we have the
replacement
\begin{equation}
    p\to p+eA
\end{equation}
in page 490, footnote (be careful that the author uses the notation
$e=|e|$, whereas I used the notation $e=-|e|$, i.e. $e$ is just $q$,
the charge).

This coupling effectively gives the replacement:
\begin{equation}
    \partial_\mu \quad\to\quad D_\mu
\end{equation}
where $D_\mu\equiv\partial_\mu - ieA_\mu$.

The Klein-Gorden equation now becomes:
\begin{equation}
    \label{eq:kg-emfield}
    [D_\mu D^\mu+m^2] \psi(x,t) = 0
\end{equation}
This equation revels a connection between $e\to-e$ and
$\psi\to\psi^*$\footnote{You will find this if you complex conjugate
the Klein-Gorden equation.}.

\paragraph{Two component interpretation} The author also mentions a
way to split the equation into two equations. If we define $\phi(x,t)$
and $\chi(x,t)$ as in eq.(8.1.15) in p.491, then it is straightforward
to check that, the Klein-Gorden equation is equivalent to:
\begin{equation}
    iD_t \Upsilon = \left[
    -\frac{1}{2m}\vb{D}^2(\tau_3+i\tau_2)+m\tau_3 \right] \Upsilon
\end{equation}
where $(D_t,\vb{D})=D_\mu$, $\tau_i$ are pauli matrices, and $\Upsilon$ is the
two component wave function:
\begin{equation}
    \Upsilon \equiv \begin{pmatrix}
        \phi(x,t) \\ \chi(x,t)
    \end{pmatrix}
\end{equation}

This formulation enables a mysterious interpretation of negative energy,
negative probability, and particles and antiparticles. For more information,
please look at pp.491~494 of \cite{Sakurai2011}.

\section{Dirac Equation}
\label{sec:Dirac Equation}

The Dirac equation aims to change the Klein-Gorden equation into a first order
PDE. There are two approaches to derive the equation. The first can be found in
section 8.2 of \cite{Sakurai2011}, and section II.1 of \cite{zee2010quantum}.
The second approach is found in page 99 (chapter 2) of \cite{Greiner1997}, and
is prefered in my opinion. After all, they lead to the same equation:

\begin{equation}
    (i\gamma^\mu \partial_\mu - m) \psi(x,t) =0
    \label{eq:dirac-eq}
\end{equation}
Where $\gamma^\mu$ are matrices satisfying:
\begin{equation}
    \frac{1}{2}\{ \gamma^\mu,\gamma^\nu\} = \eta^{\mu\nu}
\end{equation}

In other words, $\gamma^\mu$ is a representation of the Clifford algebra
$\mathrm{Cl}_{1,3}(\R)$. We can cast this equation into another form:
\begin{equation}
    i\pdv{t} ket{\psi} = H \ket{\psi} = \bm{\alpha}\vdot\vb{p}+\beta m
\end{equation}
where $\vb{p}$ is the momentum, $\bm{\alpha}$ and $\beta$ satisfy a similar
relation:
\begin{align}
    \frac{1}{2}\{\alpha_i,\alpha_j\} = \delta_{ij} \\
    \{\alpha_i,\beta\} &= 0 \\
    \alpha_i^2 = \beta^2 &= 0
\end{align}
They are related to $\gamma^\mu$ matrices by:
\begin{equation}
    \alpha_i = \gamma^0 \gamma^i \quad\text{and}\quad
    \beta = \gamma^0
\end{equation}
A detailed formula for $\gamma$ matrices can be found in p.94 of
\cite{zee2010quantum}. A detailed formula for $\alpha_i,\beta$
matrices can be found in p.496 of \cite{Sakurai2011}.

\paragraph{The density problem} The dirac formulation has a appearent advantage.
The current defined by
\begin{equation}
    j^\mu \equiv \bar{\psi}\gamma^\mu\psi
\end{equation}
where
\begin{equation}
    \bar{\psi} \equiv \psi^\dagger \beta
\end{equation}
is conserved. And the probability density $\rho$ is
\begin{equation}
    \rho \equiv j^0 = \psi^\dagger \beta \gamma^0 \psi = \psi^\dagger \psi
\end{equation}
It is positive-definite.

\paragraph{Coupled with a charge} When coupled with electromagnetic
field, and with a the vector potential $\vb{A}=0$, it becomes
\begin{equation}
    H = \bm{\alpha}\vdot\vb{p} + \beta m - e \Phi
\end{equation}
where $\Phi$ is the scalar potential.

\bibliography{cite}{}
\bibliographystyle{alphaurl}

\printnomenclature
\section{License}
The entire content of this work (including the source code
for TeX files and the generated PDF documents) by 
Hongxiang Chen (nicknamed we.taper, or just Taper) is
licensed under a 
\href{http://creativecommons.org/licenses/by-nc-sa/4.0/}{Creative 
Commons Attribution-NonCommercial-ShareAlike 4.0 International 
License}. Permissions beyond the scope of this 
license may be available at \url{mailto:we.taper[at]gmail[dot]com}.
\end{document}
